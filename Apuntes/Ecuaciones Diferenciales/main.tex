\documentclass[a4paper,12pt]{article}

% --- Idioma y codificación ---
\usepackage[utf8]{inputenc}   % Acentos directos
\usepackage[T1]{fontenc}      % Codificación de salida
\usepackage[spanish]{babel}   % Español
\usepackage{lmodern}          % Fuente moderna
\usepackage{ragged2e}         % en el preámbulo

% --- Matemáticas ---
\usepackage{amsmath}          % Entornos matemáticos
\usepackage{amssymb}          % Símbolos
\usepackage{amsthm}           % Teoremas
\usepackage{mathtools}        % Extras de amsmath
\usepackage{bm}               % Negrita en símbolos
\usepackage{esint}            % Integrales de contorno (∮)

\usepackage{graphicx}         % Imágenes
% Rutas de búsqueda para imágenes
\graphicspath{{./}{imagenes/}}
\DeclareGraphicsExtensions{.pdf,.png,.jpg,.jpeg}
\usepackage{xcolor}           % Colores
\usepackage{enumitem}         % Listas
\usepackage{titlesec}         % Títulos
\usepackage[hidelinks]{hyperref} % Enlaces (cargar al final)

% --- Esquemas y Dibujos ---
\usepackage{tikz}
\usetikzlibrary{arrows.meta}

% subsecciones más pequeñas
\titleformat{\subsection}
  {\normalfont\small\bfseries}   % formato más pequeño y en negrita
  {\thesubsection}{1em}{} 

\title{Apuntes Ecuaciones Diferenciales}
\author{Luis López}
\date{September 2025}

\begin{document}
\maketitle % Genera la portada

\newpage 

\tableofcontents % Índice automático
\newpage 
\section*{Introducción a las ecuaciones diferenciales.}
\noindent
Una \textbf{ecuación diferencial} es una relación matemática en la que intervienen una función desconocida y sus derivadas. 
Su objetivo es describir cómo varía una magnitud en función de otra, estableciendo vínculos entre la propia magnitud y su tasa de cambio.  

\medskip
\noindent
Las ecuaciones diferenciales son fundamentales porque numerosos fenómenos naturales y tecnológicos se formulan de esta manera. 
En física, por ejemplo, permiten estudiar el movimiento de partículas (leyes de Newton), la propagación del calor, las ondas o el comportamiento de los circuitos eléctricos. 
En otras disciplinas también son esenciales: en biología modelan el crecimiento de poblaciones y la difusión de sustancias, 
en economía describen la evolución de sistemas dinámicos, 
y en ingeniería sirven para analizar vibraciones, fluidos o sistemas de control.  

\medskip
\noindent
En definitiva, las ecuaciones diferenciales constituyen una herramienta matemática clave que conecta las teorías con la realidad dinámica, 
pues gran parte de los procesos que nos rodean se fundamentan en cambios que pueden expresarse mediante estas ecuaciones.

\clearpage
\section{Ecuaciones diferenciales ordinarias}
\noindent
Se denomina \textbf{ecuación diferencial ordinaria (EDO)} a toda ecuación
\[
F(x,y,y',y'',\ldots,y^{(n)}) = 0
\]
donde $x$ es la variable independiente, $y$ la variable dependiente y $y',y'',\ldots$ son derivadas de la variable dependiente.  
Llamamos \textbf{orden de la ecuación} al grado de la derivada más grande que aparece en la ecuación.  

\medskip
\noindent
Se denomina \textbf{solución} de una EDO en $(a,b)$ a toda función definida en $(a,b)$ junto a sus derivadas de orden $n$, tal que al sustituirla en la EDO da una identidad respecto a $x$ en $(a,b)$.

\medskip
\noindent
\textbf{Ejemplo:}
\[
y' = x^5 \;\;\Rightarrow\;\; y = \tfrac{x^6}{6} + C
\]
Ecuación de primer orden $\Rightarrow$ 1 constante arbitraria.
\[
y'' = y' \;\;\Rightarrow\;\; y = C_1 e^x + C_2
\]
\noindent
Si una EDO involucra varias variables independientes, se denomina \textbf{ecuación en derivadas parciales (EDP)}:
\[
F(x_1,x_2,\ldots,x_n, \tfrac{\partial y}{\partial x_1}, \tfrac{\partial y}{\partial x_2}, \ldots, \tfrac{\partial^2 y}{\partial x_i \partial x_j}, \ldots ) = 0
\]
A la gráfica de una solución de una EDO también se le llama \textbf{integral}.  

\subsection*{EDOs de primer orden}
\noindent
La forma más general de una EDO de primer orden es:
\[
F(x,y,y')=0 \quad \text{(forma implícita)}
\]
\[
y' = f(x,y) \quad \text{(forma explícita)}.
\]
\noindent
Se dice que una EDO es \textbf{autónoma} si no depende explícitamente de la variable independiente, es decir:
\[
y' = f(y) \quad \Leftrightarrow \quad F(y,y')=0.
\]
\subsection{Problema de Cauchy}
\noindent
Consiste en la búsqueda de la solución de la EDO
\[ 
y' = f(x,y) \quad \text{tal que} \quad\; y(x_0) = y_0
\]
es decir, se añade una \textbf{condición inicial}:
\[
\begin{cases}
y' = f(x,y) \\
y(x_0) = y_0
\end{cases}
\]
\noindent
\textbf{Ejemplo:}
\[
y' = y \;\;\Rightarrow\;\; y = C e^x, \qquad y=e^x \text{ si } y(0)=1
\]

\medskip
\noindent
Ya que $1 = C e^0 \Rightarrow C=1$.

\subsection{Teorema de Picard–Lindelöf (existencia y unicidad)}
\noindent
El problema de Cauchy tiene una única solución si, y solo si, $f(x,y)$ y
$\dfrac{\partial f}{\partial y}$ son continuas en $(x,y)$.
\noindent
Es decir, si $f(x,y)$ y $\dfrac{\partial f}{\partial y}$ son continuas en una cierta región del plano $x,y$,  
entonces $\exists! \; y(x)$ que pasa por cada punto de dicha región.

\medskip
\noindent
\textbf{Ejemplo:}
\[
y' = \frac{y}{x} \;\;\Rightarrow\;\; \frac{dy}{y} = \frac{dx}{x}
\;\;\Rightarrow\;\; \int \frac{dy}{y} = \int \frac{dx}{x}
\;\;\Rightarrow\;\; \ln y = \ln x + C \;\;\Rightarrow\;\; y = Cx
\]
\noindent
Si $y(x_0)=y_0$ entonces $y_0 = Cx_0 \;\;\Rightarrow\;\; C = \dfrac{y_0}{x_0}$.

\medskip
\noindent
Por definición, la \textbf{solución general} de una EDO es la familia de curvas $y(x,C)$ que satisfacen $y'=f(x,y)$ para distintos valores admisibles de $C$.  

\medskip
\noindent
Una \textbf{solución particular} es aquella que además cumple la condición inicial $y(x_0)=y_0$,  
es decir, es una curva concreta dentro de la familia general.

\[
y' = y, \quad y = C e^x \;\; \Rightarrow \;\; \text{solución general}
\]
\[
y_p = e^x \;\; \Rightarrow \;\; \text{solución particular}
\]

\section{Tipos de EDOs}

\subsection{EDOs de variables separables}
\noindent
Las EDOs de variables separadas son aquellas en las que se puede reescribir la ecuación diferencial 
de forma que cada variable quede en un lado distinto de la igualdad. Esto permite integrar ambas 
partes por separado para encontrar la solución. Suelen ser sencillas de resolver y aparecen con 
frecuencia en problemas físicos y de crecimiento poblacional. Además, siempre requieren una condición 
inicial para determinar la constante de integración y obtener la solución particular.

\medskip
\noindent
Una EDO de variables separables tiene la forma:
\[
\frac{dy}{dx} = f(x)g(y),
\]
Que se puede reescribir como:
\[
\frac{1}{g(y)}\,dy = f(x)\,dx.
\]
\noindent
\textbf{Ejemplo:}
\[
3 e^x \tan(y)\, dx = -(2 - e^x)\sec^2(y)\, dy
\]

\medskip
\noindent
Integramos en ambos lados:
\[
\int \frac{3e^x}{2-e^x}\,dx = \int \frac{-\sec^2(y)}{\tan y}\,dy
\]
Resolviendo:
\[
-3 \ln|2-e^x| = -\ln|\tan y| + C
\]
Despejando:
\[
\tan(y) = D (e^x-2)^3 \;\;\Rightarrow\;\; y = \arctan\!\big(D(e^x-2)^3\big).
\]
\newpage
\subsection{EDOs reducibles a separables}
\noindent
Las EDOs reducibles a variables separadas son aquellas que inicialmente no están 
en forma separable, pero pueden transformarse mediante un cambio de variable o 
manipulación algebraica para lograrlo. Una vez separadas, se resuelven igual que 
las de variables separadas, integrando cada lado por separado.

\medskip
\noindent
Son del tipo:
\[
\frac{dy}{dx} = f(ax+by+c),
\qquad a,b,c \;\; \text{son constantes definidas}.
\]
Se resuelven mediante el cambio de variable:
\[
z = ax + by + c.
\]
Entonces:
\[
z' = a + b\,y' 
\;\;\Rightarrow\;\; 
y' = \frac{z'-a}{b}.
\]
Sustituyendo en la ecuación original:
\[
z' = \frac{dz}{dx} = a + b\,f(z),
\]
Integrando:
\[
\int \frac{dz}{a + b\,f(z)} = x + D.
\]

\medskip
\noindent
\textbf{Ejemplo:}

\[
(x+y)^2\,y' = a^2, \qquad a = \text{cte}.
\]
Reordenamos:
\[
y' = \frac{a^2}{(x+y)^2}.
\]
Hacemos el cambio de variable:
\[
z = x+y \quad\Rightarrow\quad z' = 1 + y' \quad\Rightarrow\quad y' = z'-1.
\]
Sustituyendo:
\[
z'-1 = \frac{a^2}{z^2} \quad\Rightarrow\quad z' = 1 + \frac{a^2}{z^2}
\quad\Rightarrow\quad \int \frac{dz}{1 + \frac{a^2}{z^2}} = x + D.
\]
Integramos y operamos:
\[
\int \frac{dz}{1 + \frac{a^2}{z^2}} = \int \frac{z^2}{z^2+a^2}\,dz = \int \frac{z^2+a^2-a^2}{z^2+a^2}\ dz.
\]
\[
\int \frac{z^2+a^2}{z^2+a^2}\,dz - \int \frac{a^2}{z^2+a^2}\,dz
= z-a^2 \int \frac{1}{1+ (\frac{z}{a})^2}\,dz = z - a\arctan\!\left(\frac{z}{a}\right).
\]

\medskip
\noindent
Deshacemos el cambio de variable $z = x+y$
\[
z - \arctg\!\left(\tfrac{z}{a}\right) = x + D
\]
\[
x+y - \arctg\!\left(\tfrac{x+y}{a}\right) = x + D
\;\;\Rightarrow\;\; 
\text{Si} \quad y(0)=0 \;\;\Rightarrow\;\; D=0
\]

\[
y_g = \arctg\!\left(\tfrac{x+y}{a}\right) + D 
\;\;\Rightarrow\;\;
y_p = \arctg\!\left(\tfrac{x+y}{a}\right)
\]

\subsection{EDO’s homogéneas}
\noindent
Las EDOs homogéneas son aquellas en las que la función y su 
derivada dependen de la relación entre las variables y no de 
ellas por separado. Se suelen resolver mediante un cambio de 
variable que simplifica la ecuación y permite separarla.

\medskip
\noindent
Se dice que $f(x,y)$ es homogénea de grado $n$ si:
\[
f(tx,ty) = t^n f(x,y)
\]
Se dice que $y' = f(x,y)$ es homogénea si $f(x,y)$ es homogénea de grado 0:
\[
f(tx,ty) = f(x,y), 
\qquad 
f(x,y) = \tfrac{y}{x},
\qquad
f(tx,ty) = \tfrac{ty}{tx} = \tfrac{y}{x}
\]
\[
f(x,y) = xy \;\;\Rightarrow\;\; f(tx,ty) = t^2 xy
\]
\noindent
\textbf{Ejemplo:}

\[
f(x,y) = \left(\tfrac{x}{y}\right)^2 \;\;\Rightarrow\;\; \text{grado 0}
\]

\[
x y' = \sqrt{x^2 - y^2} + y 
\;\;\Rightarrow\;\;
y' = \frac{\sqrt{x^2 - y^2}}{\sqrt{x^2}} + \frac{y}{x}
\]

\[
\;\;\Rightarrow\;\; y' = \sqrt{1 - \left(\tfrac{y}{x}\right)^2} + \frac{y}{x}
\]

\newpage
\noindent
Hacemos el cambio de variable $z = \frac{y}{x}$
\[
z' = \frac{y'x - y}{x^2} \quad \Rightarrow \quad y' = x z' + y
\]
Operando:
\[
y' = \sqrt{1 - z^2} + z 
\quad \Rightarrow \quad 
z + z'x = \sqrt{1 - z^2} + z
\]
\[
\Rightarrow \quad z' = \frac{\sqrt{1-z^2}}{x} = \frac{dz}{dx}
\]
Integrando:
\[
\int \frac{dz}{\sqrt{1-z^2}} = \int \frac{dx}{x} \;\Rightarrow\; \arcsin(z) = \ln|x| + C \;\Rightarrow\; z = \sin(\ln|x| + C)
\]
Desacemos el cambio de variable $z = \frac{y}{x}$
\[
y = x \sin(\ln|x| + C)
\]
Familia de soluciones:
\[
y' = \frac{dy}{dx} = f\left( \frac{ax + by + c}{a_1 x + b_1 y + c_1} \right)
\]
Si \( c = c_1 = 0 \):

\[
y' = \frac{dy}{dx} = f\left( \frac{ax + by}{a_1 x + b_1 y} \right) =
f\left( \frac{a + b \frac{y}{x}}{a_1 + b_1 \frac{y}{x}} \right) = 
f\left( \frac{2 + 3 \frac{y}{x}}{5 + 4 \frac{y}{x}} \right)
\]
Cambio de variable:

\[
z = \frac{y}{x} \quad \Rightarrow \quad y' = z' x + z
\]

\[
z'x + z = h\left( \frac{2 + 3z}{5 + 4z} \right)
\]

\medskip
\noindent
Si \( c \neq 0 \) o \( c_1 \neq 0 \), no es homogénea, entonces:
\[
\begin{vmatrix}
a & b \\
a_1 & b_1
\end{vmatrix} = 0
\quad \Rightarrow \quad
\frac{a_1}{a} = \frac{b_1}{b} = d
\quad \Rightarrow \quad
\left[
\begin{aligned}
a_1 &= da \\[6pt]
b_1 &= db
\end{aligned}
\right.
\]

\newpage
\noindent
El cambio de variable sería:
\[
z = ax + by
\]
\[
y' = f\left( \frac{ax + by + c}{a_1 x + b_1 y + c_1} \right) = f\left( \frac{ax + by + c}{d(ax + by) + c_1} \right) = g(ax + by) = g(z)
\]

\subsection*{Ejemplo:}
\[
y' = \frac{x + y + 1}{-2x - 2y + 1} = \frac{(x + y) + 1}{-2(x + y) + 1}
\]
Hacemos el cambio:

\[
z = x + y
\quad \Rightarrow \quad
z' - 1 = \frac{z + 1}{-2z + 1}
\]
Operando:

\[
z' = \frac{z + 1}{-2z + 1} + 1 = \frac{-2z + 1 + z + 1}{-2z + 1} = \frac{-z + 2}{-2z + 1} =
\frac{dz}{dx} = z'
\]

\[
\frac{-2z + 1}{-z + 2} dz = dx
\]
Integrando:

\[
C_1 + x = \int \frac{-2z + 1}{-z + 2} dz = \int \frac{-2z}{-z + 2} dz + \int \frac{1}{-z + 2} dz
\]

\[
= -2z + 2 + \int \frac{1}{-z + 2} dz = -\ln|2 - z|
\]
Si $c$ o $c_1 \neq 0$ y el determinante $\neq 0$:

\[
\begin{vmatrix}
a & b \\
a_1 & b_1
\end{vmatrix} \neq 0
\;\;\Rightarrow\;\;
\begin{cases}
x = \xi + h \\
y = \eta + k
\end{cases}
\]

\medskip
\noindent
Entonces:
\[
y' = \frac{dy}{dx} = \frac{d\eta}{d\xi}\cdot \frac{d\xi}{dx} 
= \frac{d\eta}{d\xi} \;\;\Rightarrow\;\; y' = \frac{d\eta}{d\xi}
\]

\[
\frac{d\eta}{d\xi} = 
\frac{a\xi + b\eta + ah + bk + c}{a_1\xi + b_1\eta + a_1 h + b_1 k + c_1}
= \frac{a + b\frac{\eta}{\xi}}{a_1 + b_1\frac{\eta}{\xi}}
= \frac{d\eta}{d\xi}
\]
\newpage
\noindent
\subsection*{Ejemplo:}
\vspace{-1.2em}
\[
(x+y-2)\,dx + (x-y+4)\,dy = 0
\]
Hacemos el cambio de variable:
\[
y' = \frac{x+y-2}{-x+y-4}
\quad \Rightarrow \quad
\begin{cases}
x = \xi + h \\
y = \eta + k
\end{cases}
\]
Si sustituimos quedaría:
\[
\frac{d\eta}{d\xi} = \frac{\xi + h + \eta + k - 2}{-\xi - h + \eta + k - 4}
\quad \Rightarrow \quad
\begin{cases}
h+k=2 \\
-h+k=4
\end{cases}
\quad \Rightarrow \quad
\begin{cases}
h=-1 \\
k=3
\end{cases}
\]
Entonces:

\[
\frac{d\eta}{d\xi} = \frac{\xi + \eta}{-\xi + \eta}
= \frac{1+\tfrac{\eta}{\xi}}{-1+\frac{\eta}{\xi}}
\]
Cambio de variable:

\[
z = \frac{\eta}{\xi} \;\;\Rightarrow\;\; \eta = z\xi
\quad \Rightarrow \quad
\frac{d\eta}{d\xi} = \frac{d}{d\xi}(\xi z) = \xi \frac{dz}{d\xi} + z
\]
Aplicando la sustitución:

\[
\frac{1+z}{-1+z} = \frac{dz}{d\xi}\,\xi + z
\;\;\Rightarrow\;\;
\frac{1+z+z+z^2}{-1+z} = -\frac{dz}{d\xi}\,\xi
\]
Integramos:
\[
\int \frac{z-1}{-z^2+2z+1}\,dz = \int \frac{d\xi}{\xi}
\;\;\Rightarrow\;\;
\ln|1+2z-z^2| = \ln|\xi^{-2}| + C = \ln\left|\frac{D}{\xi^2}\right|
\]

\medskip
\noindent
Aplicamos exponencial en ambos lados:

\[
1+2z-z^2 = \frac{D}{\xi^2}
\;\;\Rightarrow\;\;
\xi^2 + 2z\xi^2 - z^2\xi^2 = D
\]
Deshacemos el cambio de variable:
\[
(x+1)^2 + 2(x+1)(y-3) - (y-3)^2 = D
\]

\newpage
\noindent
\subsection{EDOs exactas}
\noindent
Las ecuaciones diferenciales exactas son aquellas en las que existe una función 
potencial cuyo diferencial total coincide con la ecuación dada. Su solución se 
obtiene hallando dicha función potencial f(x,y) tal que sus derivadas parciales 
coincidan con los coeficientes de la ecuación.\\
Son de la forma:
\[
f(x,y)=c \;\;\Rightarrow\;\; y'=0
\]
\medskip
\[
df = \frac{\partial f}{\partial x}\,dx + \frac{\partial f}{\partial y}\,dy = 0
\;\;\Rightarrow\;\;
\frac{dy}{dx} = -\frac{\dfrac{\partial f}{\partial x}}{\dfrac{\partial f}{\partial y}}
\;\;\Rightarrow\;\;
\begin{cases}
M(x,y) = \dfrac{\partial f}{\partial x} \\[6pt]
N(x,y) = \dfrac{\partial f}{\partial y}
\end{cases}
\]

\[
M(x,y)\,dx + 
N(x,y)\,dy
\;\;\Leftrightarrow\;\; 
\exists f \;\; \text{tal que } f(x,y)=\text{constante}.
\]

\[
\frac{\partial f}{\partial x}=M \;\;\Rightarrow\;\; 
f(x,y) = \int M\,dx + g(y)
\]

\[
\frac{\partial f}{\partial y}=N \;\;\Rightarrow\;\;
\int \frac{\partial M}{\partial y}\,dx + g'(y)
\]
Por lo tanto,
\[
g'(y) = N - \int \frac{\partial M}{\partial y}\,dx 
\;\;\Rightarrow\;\;
g(y) = \int\left[N - \int \frac{\partial M}{\partial y}\,dx\right]dy + C
\]

\subsubsection*{“Demostración”}
\noindent
Podemos hacer esto ya que si $f(x,y)=c \;\;\Rightarrow\;\; df=0$,  
entonces
\[
\frac{\partial f}{\partial x}\,dx + \frac{\partial f}{\partial y}\,dy = 0.
\]
Entonces:
\[
\frac{\partial^2 f}{\partial x \partial y} = \frac{\partial^2 f}{\partial y \partial x}
\;\;\Rightarrow\;\;
\frac{\partial}{\partial x}\!\left(\frac{\partial f}{\partial y}\right)
= \frac{\partial}{\partial y}\!\left(\frac{\partial f}{\partial x}\right)
\;\;\Rightarrow\;\;
\frac{\partial N}{\partial x} = \frac{\partial M}{\partial y}.
\]

\newpage
\noindent
\subsection*{Ejemplo:}
\vspace{-1.2em}
\[
(x+y-2)\,dx + (x-y+4)\,dy = 0
\]
Llamamos \(M\) a la primera parte y \(N\) a la segunda:
\[
M=(x+y-2)\,dx, \qquad N=(x-y+4)\,dy
\]
Comprobamos que sea exacta:
\[
\begin{cases}
\frac{\partial M}{\partial y}=1 \\[6pt]
\frac{\partial N}{\partial x}=1
\end{cases}
\;\;\Rightarrow\;\; \text{Es exacta.}
\]

\[
\frac{\partial f}{\partial x}=x+y-2
\]
Integramos:
\[
f=\int (x+y-2)\,dx + g(y)
= \frac{x^{2}}{2}+xy-2x+g(y)
\]
\[
N=\frac{\partial f}{\partial y}=x+g'(y)=x-y+4
\;\Rightarrow\;
g(y)=\int(-y+4)\,dy=-\frac{y^{2}}{2}+4y+C
\]

\medskip
\noindent
Resultando en:
\[
f(x,y)=\frac{x^{2}}{2}-2x-\frac{y^{2}}{2}+4y+xy+C=0
\]

\subsection*{Ejemplo:}
\vspace{-1.5em}
\[
(\sen(xy)+xy\cos(xy))\,dx+x^{2}\cos(xy)\,dy=0
\]
Donde:
\[
\begin{cases}
(\sen(xy)+xy\cos(xy))\,dx \;\;\longrightarrow\; M\\[4pt]
x^{2}\cos(xy)\,dy \;\;\longrightarrow\; N
\end{cases}
\]
Comprobamos que sea exacta.
\[
\begin{cases}
\frac{\partial M}{\partial y}
= x\cos(xy) + x\cos(xy) + xy(-x\sen(xy)) \\[6pt]
\frac{\partial N}{\partial x}
= 2x\cos(xy) - x^{2}y\sen(xy)
\end{cases}
\]
Queda comprobado que es una EDO exacta ya que son iguales.

\newpage
\noindent
Resolvemos:
\[
f=\int \big[x^{2}\cos(xy)\big]\,dy + g(x) = -x\,\sen(xy) + g(x)
\]
\[
\frac{\partial f}{\partial x}=M
\;\Rightarrow\;
-\sen(xy)-xy\cos(xy)+g'(x) = \sen(xy)+xy\cos(xy)
\]

\[
g'(x)=2\sen(xy)+2xy\cos(xy)
\]

\[
g(x)=\int \big[\,2\sen(xy)+2xy\cos(xy)\,\big]\,dx
\]
\subsection{EDO's cuasi-exactas}
\noindent
Las ecuaciones diferenciales cuasiexactas son aquellas que no son exactas inicialmente, 
pero pueden convertirse en exactas al multiplicarlas por un factor integrante adecuado. 
Una vez transformadas, se resuelven como una ecuación exacta encontrando la función 
potencial cuya derivada total reproduce la ecuación original.
\[
M\,dx + N\,dy = 0 \;\;\Rightarrow\;\; \frac{\partial M}{\partial y} \neq \frac{\partial N}{\partial x}
\qquad
\mu(x,y)=\text{factor integrante.}
\]
Si $\;\exists\,\mu(x,y)$ tal que $\;\mu(x,y)M\,dx+\mu(x,y)N\,dy=0\;$ es una EDO exacta, entonces la EDO original es cuasi-exacta.

\medskip
\noindent
Supongamos que $\;\exists\,f(x,y)=c,\;\; \frac{\partial f}{\partial x}=\mu M\;\; \text{y}\;\; \frac{\partial f}{\partial y}=\mu N$\\
Entonces:

\[
\frac{\partial^2 f}{\partial y\,\partial x}=\frac{\partial^2 f}{\partial x\,\partial y}
\;\;\Rightarrow\;\;
\frac{\partial}{\partial y}(\mu M)=\frac{\partial}{\partial x}(\mu N)
\]

\[
N\,\frac{\partial \mu}{\partial x}+\mu\,\frac{\partial N}{\partial x}
=
M\,\frac{\partial \mu}{\partial y}+\mu\,\frac{\partial M}{\partial y}
\;\;\Rightarrow\;\;
N\,\frac{\partial \mu}{\partial x}-M\,\frac{\partial \mu}{\partial y}
=
\mu\!\left(\frac{\partial M}{\partial y}-\frac{\partial N}{\partial x}\right)
\]
\newpage
\begin{itemize}
\item Si: $\displaystyle \frac{\partial_y M-\partial_x N}{N}$ solo depende de $x$ $\Rightarrow \mu=\mu(x)$
\end{itemize}

\[
\begin{gathered}
\mu = \mu(x) \Rightarrow\quad \frac{\partial \mu}{\partial y} = 0 
\Rightarrow\quad N\,\frac{\partial \mu}{\partial x} 
= \mu\!\left(\frac{\partial M}{\partial y} - \frac{\partial N}{\partial x}\right) \\[4pt]
\frac{d\mu}{\mu} =
\frac{\tfrac{\partial M}{\partial y} - \tfrac{\partial N}{\partial x}}{N}dx
\end{gathered}
\]
Resolvemos:
\[
\int \frac{d\mu}{\mu}=\int g(x)\,dx
\;\Rightarrow\;
\ln\mu=\int g(x)\,dx+\alpha
\;\Rightarrow\;
\mu(x)=c\,e^{\int g(x)\,dx}
\]
\medskip
\begin{itemize}
\item Si: $\displaystyle \frac{\partial_x N-\partial_y M}{M}$ solo depende de $y$ $\Rightarrow \mu=\mu(y)$
\end{itemize}

\[
\mu(y)=e^{\int h(y)\,dy}
\qquad\text{con}\qquad
h(y)=\frac{\partial_x N-\partial_y M}{M}
\]

\bigskip
\subsection*{Ejemplo:}
\[
(x+y^2)\,dx - 2xy\,dy = 0
\;\;\Rightarrow\;\;
\begin{cases}
\frac{\partial M}{\partial y} = 2y, \\[6pt]
\frac{\partial N}{\partial x} = -2y
\end{cases}
\;\;\Rightarrow\;\; \text{No es exacta.}
\]
Reslvemos:
\[
\frac{\partial_y M - \partial_x N}{N} = \frac{2y - (-2y)}{-2xy} = \frac{4y}{-2xy} = -\frac{2}{x}
= g(y)=0
\;\;\Rightarrow\;\; \mu=\mu(x)
\]
Integrando:
\[
\int \frac{d\mu}{\mu} = \int -\frac{2}{x}\,dx
\;\;\Rightarrow\;\;
\ln\mu = -2\ln x + C = \ln\left(\frac{C}{x^2}\right)
\;\;\Rightarrow\;\;
\mu(x)=\frac{C}{x^2}
\]
\newpage
\noindent
Multiplicamos la EDO por el factor integrante:
\[
\mu M\,dx + \mu N\,dy = 0
\;\;\Rightarrow\;\;
\frac{1}{x^2}\big[(x+y^2)\,dx - 2xy\,dy\big]=0
\]

\[
\Big(\frac{1}{x}+\frac{y^2}{x^2}\Big)\,dx - \frac{2y}{x}\,dy=0
\]

\[
\frac{\partial f}{\partial x}=\frac{y^2}{x^2} + g'(x) = \frac{1}{x} + \frac{y^2}{x^2}
\;\;\Rightarrow\;\;
g'(y)=\frac{-1}{x}
\;\;\Rightarrow\;\;
g(x)=\ln x+C
\]

\[
\frac{\partial f}{\partial y}=\frac{-2y}{x}
\;\;\Rightarrow\;\;
f=\int \frac{-2y}{x}\,dy + g(x) = \frac{-y^2}{x} + g(x)
\]

\[
f(x,y)=\frac{-y^2}{x}+\ln x+C
\]

\subsection{EDOs lineales de primer orden}
\noindent
Las ecuaciones diferenciales lineales de primer orden se resuelven usando un 
factor integrante que permite escribir la ecuación como una derivada total, 
facilitando así la integración directa para encontrar la solución general.

\medskip
\noindent
La forma general es:
\[
\frac{dy}{dx}+p(x)y=q(x)
\]
Una EDO es lineal si la combinación lineal de soluciones también es solución.
\medskip
\noindent
Si $y_1,y_2$ son soluciones $\;\;\Rightarrow\;\;\alpha y_1+\beta y_2$ también es solución.

\medskip
\noindent
Hay dos tipos:
\[
\text{Si } q(x)=0 \;\;\Rightarrow\;\; \text{EDO homogénea}
\]
\[
\text{Si } q(x)\neq 0 \;\;\Rightarrow\;\; \text{EDO inhomogénea}
\]
Es decir:
\[
\frac{d}{dx}(\alpha y_1 + \beta y_2) + p(x)(\alpha y_1 + \beta y_2) 
= \alpha \frac{dy_1}{dx} + \beta \frac{dy_2}{dx} + \alpha p(x)y_1 + \beta p(x)y_2
\]

\[
= \alpha\left(\frac{dy_1}{dx}+p(x)y_1\right)+\beta\left(\frac{dy_2}{dx}+p(x)y_2\right)
= (\alpha+\beta)q(x)
\]

\subsection{Homogénea q(x)=0}

\[
\frac{dy}{dx}+p(x)y=0
\;\;\Rightarrow\;\;
\frac{dy}{y}=-p(x)\,dx
\;\;\Rightarrow\;\;
\ln y = -\int p(x)\,dx
\;\;\Rightarrow\;\;
y=Ce^{-\int p(x)\,dx}
\]
\subsection{Inhomogénea ($q(x)\neq 0$)}
\[
\frac{dy}{dx}+p(x)y=g(x)
\]

\[
y_g(x)=y_h(x)+y_p(x)
\]

\[
\frac{d}{dx}(y_h+y_p)+p(x)(y_h+y_p)=\frac{dy}{dx}+p(x)y=q(x)
\]
Primero resuelvo la homogénea:
\[
y_h = C e^{-\int p(x)\,dx}
\]
Luego la particular:
\[
y_p = C(x)y_h \;\;\Rightarrow\;\; y_p = C(x)e^{-\int p(x)\,dx}
\]
\[
\frac{dy_p}{dx}+p(x)y_p = C'(x)y_h + C(x)+y'_h+p(x)y_h=q(x)
\]

\[
C(x)\int\frac{q(x)}{y_h(x)}\,dx+D 
\;\;\Rightarrow\;\;
y_p=Dy_h+y_h\int \frac{q(x)}{y_h}\,dx
\]

\subsection*{Ejemplo:}
\vspace{-1.5em}
\[
y' + 2y = e^{-x}
\]
Primero resolvemos la homogénea:
\[
y' + 2y = 0 \;\;\Rightarrow\;\; \int \frac{dy}{y} = \int -2dx 
\;\;\Rightarrow\;\; \ln y = -2x
\]
\[
y_h = Ce^{-2x}
\]
\newpage
\noindent
Ahora resolvemos la particular:
\[
y_p = C(x)e^{-2x} \;\;\Rightarrow\;\; 
y_p' = C'(x)e^{-2x} - 2C(x)e^{-2x}
\]
\[
C'(x)e^{-2x} - 2C(x)e^{-2x} + 2C(x)e^{-2x} = e^{-x} 
\;\;\Rightarrow\;\; C''(x) = e^{x}
\]
\[
C'(x) = e^x \;\;\Rightarrow\;\; C(x) = \int e^x dx + D = e^x + D
\]
La solución particular quedaría:
\[
y_p = (e^x + D)e^{-2x} = e^{-x} + De^{-2x}
\]
La solución general quedaría:
\[
y_g = y_h + y_p = Ce^{-2x} + e^{-x} + De^{-2x}
= Ae^{-2x} + e^{-x}
\]

\subsection*{Ejemplo:}
\vspace{-1.0em}
\[
\frac{dy}{dx}=\frac{1}{x\cos y+\sen 2y}
\;\;\Rightarrow\;\;
\frac{dx}{dy}=x\cos y+\sen 2y
\]

\[
\frac{dx}{dy}-x\cos y=\sen 2y
\qquad
\text{donde }\;
\begin{cases}
\cos y = p(y)\\[2pt]
\sen 2y = q(y)
\end{cases}
\qquad
\begin{aligned}
&y'+p(x)\,y=g(x)\\
&x'+p(y)\,x=g(y)
\end{aligned}
\]
Resolvemos la homogénea:
\[
x_h'-x_h\cos y=0
\;\Rightarrow\;
\int \frac{dx_h}{x_h}=\int \cos y\,dy
\;\Rightarrow\;
\ln x_h=\sen y + D
\]

\[
x_h=e^{\sen y}
\]
Resolvemos la particular:

\[
x_p=C(y)e^{\sen y}
\;\Rightarrow\;
x_p'=C'(y)e^{\sen y}+C(y)\cos y\,e^{\sen y}
\]

\[
x_p'-y_p\cos y=\sen(2y)
\]

\[
C'(y)e^{\sen y}+C(y)\cos y\,e^{\sen y}-C(y)e^{\sen y}\cos y=\sen 2y
\]

\[
 C'(y) = e^{\sen y}
 \;\Rightarrow\; C(y) = \int \sen 2y\,e^{-\sen y}\,dy  = 2\!\int \cos y\,\sen y\,e^{-\sen y}\,dy
\]

\newpage
\noindent
Hacemos un cambio de variable:
\[
z = \sen y \;\Rightarrow\; dz = \cos y\, dy
\]
\[
2\!\int \cos y\,\sen y\,e^{-\sen y}\,dy
=2\!\int z\,e^{-z}\,dz
\]

\subsection{Método de los coeficientes indeterminados}
\noindent
Las ecuaciones diferenciales con coeficientes indeterminados se 
resuelven encontrando primero la solución general de la ecuación 
homogénea asociada y luego una solución particular para la parte no 
homogénea. Esta última se obtiene proponiendo una forma adecuada 
según el término independiente y determinando los coeficientes desconocidos.

\medskip
\noindent
Ecuaciones de la forma:
\[
y' + p y = g(x) \quad \rightarrow \quad p = \text{cte}
\]
Donde $p(x)$ es “sencilla”, es decir, polinómica, exponencial o aritmética.

\medskip
\noindent
La solución general es:
\[
y_g(x)=y_h(x)+y_p(x)
\]
\subsection*{Ejemplo:}
\vspace{-1.2em}
\[
y' + 2y = e^{-x}
\;\Rightarrow\;
\begin{cases}
p(x) = 2 \\[4pt]
q(x) = e^{-x}
\end{cases}
\]
Ecuación homogénea:

\[
y' + 2y = 0 \quad \Rightarrow \quad y_h = C e^{-2x}
\]
Solución particular:

\[
y_p = A e^{-x} 
\quad \Rightarrow \quad y_p' = -A e^{-x}
\quad \Rightarrow \quad y'_p+2y_p = e^{-x}
\]
Sustituyendo en la ecuación:

\[
-A e^{-x} + 2A e^{-x} = e^{-x}
\quad \Rightarrow \quad
Ae^{-x} = e^{-x} 
\quad \Rightarrow \quad
A = 1
\]
Solución general:
\[
y = y_h + y_p = C e^{-2x} + e^{-x}
\]
\subsection*{Ejemplo:}
\vspace{-1.2em}
\[
y' + y = x^2 + x + 1
\]
Resolvemos la homogénea:

\[
y' + y = 0 \quad \Rightarrow \quad y_h = C e^{-x}
\]
Buscamos la particular:

\[
y_p = Ax^2 + Bx + C \quad \Rightarrow \quad y_p' = 2Ax + B
\]
Sustituyendo:

\[
2Ax + B + A x^2 + Bx + C = x^2 + x + 1
\]
\[
A x^2 + (2A + B)x + B + C = x^2 + x + 1
\]
Igualando coeficientes:

\[
\begin{cases}
A = 1\\ 
2A + B = 1 \quad \Rightarrow \quad B = 0\\ 
B + C = 1 \quad \Rightarrow \quad C = 1
\end{cases}
\]
De aquí saco la ecuación particular:

\[
y_p = x^2 + 2
\]
Y la solución general:

\[
y_g = y_h + y_p = C e^{-x} + x^2 + 2
\]
\subsection*{Ejemplo:}
\vspace{-1.2em}
\[
y' + y = \sin x
\]
Primero resolvemos la homogénea:

\[
y' + y = 0 \quad \Rightarrow \quad y_h = C e^{x}
\]
Ahora buscamos la particular:

\[
y_p = A \sin x + B \cos x
\quad \Rightarrow \quad
y_p' = A \cos x - B \sin x
\]

\newpage
\noindent
Sustituyendo:

\[
A \cos x - B \sin x + A \sin x + B \cos x = \sin x
\]

\[
(A - B) \sin x + (A + B) \cos x = \sin x
\]
Igualando coeficientes:

\[
\begin{cases}
A - B = 1 \\[4pt]
A + B = 0
\end{cases}
\]
Resolvemos el sistema:
\[
\begin{cases}
2A = 1 \quad \Rightarrow \quad A = \frac{1}{2} \\[4pt]
A = -B \quad \Rightarrow \quad B = -\frac{1}{2}
\end{cases}
\]
La ecuación particular queda:

\[
y_p = \tfrac{1}{2}\sin x - \tfrac{1}{2}\cos x
\]
Y la ecuación general queda:

\[
y = y_h + y_p = C e^{-x} + \tfrac{1}{2}\sin x - \tfrac{1}{2}\cos x
\]

\subsection{EDO's de Bernoulli}
\noindent
Las ecuaciones de Bernoulli son no lineales pero pueden transformarse en 
lineales mediante un cambio de variable adecuado.

\medskip
\noindent
Son EDOs de la forma:
\[
\frac{dy}{dx} + p(x)y = q(x) y^n \quad \text{donde } n \neq 0,1
\]
Se puede reducir a una EDO lineal haciendo un cambio de variable:

\[
z = y^{1-n} \quad \Rightarrow \quad z' = (1 - n) y^{-n} y'
\]

\[
y' = \frac{1}{1 - n} y^n z' = \frac{1}{1-n} z' z^{\frac{n}{1-n}}
\]

\newpage
\noindent
Sustituyendo en la ecuación original:
\[
\frac{dy}{dx} + p(x)y = q(x) y^n
\;\;\Rightarrow\;\;
\frac{1}{1 - n} z'z^{\frac{n}{1-n}} + p(x) z^{\frac{1}{1-n}}  = q(x)z^{\frac{n}{1-n}}
\]
Multiplicando todo por $z^{(1 - n)}$:

\[
\frac{1}{1-n}z' + p(x) z = q(x)
\]
Resultado: una EDO lineal.

\subsection*{Ejemplo}
\vspace{-1.5em}
\[
y' - xy = -x y^3 
\quad\Rightarrow \quad
\text{donde}
\quad\Rightarrow\quad
\begin{cases}
p(x) = -x \\[4pt]
q(x) = -x \\
\end{cases}
\]
Hacemos el cambio de variable:

\[
z = \frac{1}{y^2} = y^{-2}
\quad \Rightarrow \quad
y = \frac{1}{\sqrt{z}}
\quad \Rightarrow \quad
y' = -\frac{1}{2} \frac{1}{\sqrt{z^3}} z'
\]
Desarrollando:
\[
-\frac{1}{2} \frac{1}{\sqrt{z^3}} z' - x \frac{1}{\sqrt{z}} = -x \frac{1}{\sqrt{z^3}}
\]
Multiplicando por $\sqrt{z^3}$:
\[
-\frac{1}{2} z' - xz = -x
\;\;\Rightarrow\;\;
z' + 2xz = 2x
\]
Resolviendo la lineal que nos queda

\[
z' + 2xz = 2x
\]
Resolviendo la homogénea:

\[
z'_h + 2xz_h = 0 \quad \Rightarrow \quad \frac{dz}{z} = -2x\, dx
\]
Integrando:

\[
\int \frac{dz}{z} = \int -2x\, dx
\quad\Rightarrow\quad
\ln z = -x^2
\]
\[
z_h = C e^{-x^2}
\]

\newpage
\noindent
Resolviendo la particular:

\[
z_p = C e^{-x^2} \quad \Rightarrow \quad z'_p = C'(x) e^{-x^2} - 2x C e^{-x^2}
\]
Sustituyendo y resolviendo:

\[
C'(x)e^{-x^2} - 2xC(x)e^{-x^2} + 2xC(x)e^{-x^2} = 2x
\]

\[
C(x) = \int 2x e^{x^2}\, dx = e^{x^2} 
\]
\[
z_p = e^{x^2} e^{-x^2} = \frac{e^{x^2}}{e^{x^2}} = 1
\]
\subsection{EDO's de solución paramétrica}
\noindent
Las ecuaciones diferenciales paramétricas se resuelven expresando 
la solución en función de un parámetro auxiliar. Este parámetro 
permite transformar la ecuación original en una o varias más sencillas, 
que al resolverse proporcionan la solución en forma paramétrica.

\medskip
\noindent
Son del tipo:
\[
\begin{cases}
x = x(p)\\ 
y = y(p)
\end{cases}
\quad \Rightarrow \quad
\text{es una curva de forma paramétrica.}
\]

\subsection*{Ejemplo:}
\vspace{-1.2em}
\[
y' = p \quad 
\]
\[
y' = 2x \quad \Rightarrow \quad y = x^2
\]
Si pasamos a forma paramétrica:

\[
x(p) = \frac{p}{2}, \quad y(p) = \frac{p^2}{4}
\]

\newpage
\subsection*{1. EDO's del tipo F(y, y') = 0}

\subsubsection*{1.1}
\vspace{-1.2em}
\[
y' = g(y) = \frac{dx}{dx} \quad \Rightarrow \quad \int \frac{1}{g(y)}\, dy = \int dx
\]

\subsubsection*{1.2 (Las importantes)}
\vspace{-1.2em}
\[
y = g(y') \quad \Rightarrow \quad y = g(p)
\]

\[
\begin{cases}
p = y' \quad \Rightarrow \quad \dfrac{dy}{dx} = p \quad \Rightarrow \quad dy = p\, dx \\[8pt]
y = g(p) \quad \Rightarrow \quad \dfrac{d}{dp} \quad \Rightarrow \quad dy = g'(p)\, dp
\end{cases}
\]

\medskip
\noindent
Lo que resulta en:

\[
x = \int \frac{g'(p)}{p}\, dp + C
\]

\subsection*{Ejemplo:}
\vspace{-1.2em}
\[
y = (y')^2 + (y')^3 \quad \Rightarrow \quad \text{donde} \quad \Rightarrow \quad y' = p = dy/dx
\]

\[
y = p^2 + p^3 \quad \Rightarrow \quad dy = (2p + 3p^2)\, dp = pdx
\]

\[
\int (2 + 3p)\, dp = \int dx \quad \Rightarrow \quad
\begin{cases}
x(p) = 2p + \tfrac{3}{2} p^2 \\[6pt]
y(p) = p^2 + p^3
\end{cases}
\]

\subsection*{2. EDO's del tipo F(x, y') = 0} 

\subsubsection*{2.1}
\vspace{-1.2em}
\[
y' = g(x) \quad \Rightarrow \quad y = \int g(x)\, dx + C
\]

\subsubsection*{2.2}
\vspace{-1.2em}
\[
x = g(y') \quad \Rightarrow \quad p = y' \quad \Rightarrow \quad \frac{dy}{dx} = p
\quad \Rightarrow \quad \frac{dy}{p} = dx
\]

\[
x = g(p) \quad \Rightarrow \quad dx = g'(p)dp = \frac{dy}{p}
\]

\[
y = \int p\, g'(p)\, dp + C \quad ; \quad x(p) = g(p)
\]
\subsection*{Ejemplo:}

\[
x = ay' + by'^2 \quad ; \quad y' = p \quad \Rightarrow \quad x = ap + bp^2 
\]
donde \(a, b\) son constantes.

\[
dx = (a + 2bp)\, dp = \frac{dy}{p} \quad \Rightarrow \quad y = \int (ap + 2bp^2)\, dp
\]

\[
\begin{cases}
y(p) = \frac{a}{2}p^2 + \frac{2b}{3} p^3 + C \\[6pt]
x(p) = ap + bp^2
\end{cases}
\]

\subsection{EDO´s de Lagrange}
\noindent
Las ecuaciones de Lagrange tienen la forma $y = x\,f(p) + g(p)$, donde $p = y'$. 
Se resuelven tratando $p$ como variable independiente y expresando $x$ y $y$ en función de $p$. 
Esto permite reducir la ecuación a una forma paramétrica más sencilla que puede integrarse directamente.
\[
y = x f(y') + g(y') \quad\text{donde establecemos que} \quad y' = p.
\]
\[
y = x f(p) + g(p) \quad \Rightarrow \quad 
\begin{cases}
dy = dxf(p) + x f'(p)\, dp + g'(p)\, dp \\[4pt]
dy = p\, dx
\end{cases}
\]

\[
\frac{dx}{dp} [f(p) - p] + x f'(p) + g'(p) = 0
\]

\newpage
\noindent
Multiplico todo por \(dp\):

\[
dxf(p) + x f'(p)\, dp + g'(p)\, dp = pdx
\]

\[
dx[f(p) - p] + x f'(p)\, dp + g'(p)\, dp
\]

\[
\frac{dx}{dp} [f(p) - p] + x f'(p) + g'(p) = 0
\]

\medskip
\noindent
Divido todo por \(f(p) - p\):

\[
\frac{dx}{dp} + \frac{f'(p)}{f(p) - p} x = - \frac{g'(p)}{f(p) - p} 
\quad \Rightarrow \quad \text{EDO lineal para } x(p)
\]
Donde:

\[
\begin{cases}
P(p) = \frac{f'(p)}{f(p) - p} \\[10pt]
Q(p) = -\frac{g'(p)}{f(p) - p}
\end{cases}
\]

\[
y(p,c) = x(p,c) f(p) + g(p)
\]
\subsection*{Ejemplo:}
\vspace{-1.2em}
\[
y = 2xy' + \ln y
\]
Hacemos el cambio:

\[
p = y' \quad ; \quad y = 2xp + \ln p \quad \Rightarrow \quad dy = 2p\, dx + 2x\, dp + \frac{dp}{p} = pdx
\]

\[
p\, dx + (2x + \frac{1}{p})\, dp = 0 \quad \Rightarrow \quad p\frac{dx}{dp} + 2x + \frac{1}{p} = 0
\]

\[
\frac{dx}{dp} + \frac{2x}{p} = -\frac{1}{p^2}
\]

\newpage
\noindent
Resolvemos la EDO homogénea:

\[
\frac{dx}{dp} + \frac{2x}{p} = 0
\]

\[
\frac{dx}{x} = -\frac{2\, dp}{p} 
\quad \Rightarrow \quad 
\int \frac{dx}{2x} = \int -\frac{dp}{p}
\quad \Rightarrow \quad 
\ln x = -2 \ln p + a 
\]

\[
x(p) = \frac{C}{p^2}
\]
Resolvemos la solución particular:

\[
x(p) = \frac{C(p)}{p^2} \quad \Rightarrow \quad x'(p) = \frac{C'(p)}{p^2} - \frac{2C(p)}{p^3}
\]

\[
\frac{C'(p)}{p^2} - \frac{2C(p)}{p^3} + \frac{2C(p)}{p^3} = -\frac{1}{p^2}
\]

\[
C'(p) = 1 \quad \Rightarrow \quad C(p) = p
\]

\[
x_p = \frac{p}{p^2} = \frac{1}{p}
\]
Solución general:
\[
y_g = \frac{C}{p^2} + \frac{1}{p} \quad \rightarrow \quad x(p) = \frac{C}{p^2} + \frac{1}{p}
\quad ; \quad y = 2xy' + \ln y'
\]

\[
y = 2 \left( \frac{C}{p^2} + \frac{1}{p} \right) p + \ln p = 2 \left( \frac{C}{p} + 1 \right) + \ln p
\]

\subsection{EDO's de Clairaut}
\noindent
Las ecuaciones de Clairaut tienen la forma $y = x\,y' + f(y')$. 
Se resuelven derivando la ecuación y eliminando la variable $y'$ para obtener la solución general. 
Además, pueden presentar una solución singular que se obtiene al resolver simultáneamente 
la ecuación original y su derivada respecto a $y'$.

\newpage
\noindent
Se hace el cambio de variable:
\[
y' = p
\]
\[
y = xp + g(p) = xp + g(p)
\]
Entonces:

\[
\begin{cases}
dy = p\, dx \\[6pt]
dy = p\, dx + x\, dp + g'(p)\, dp
\end{cases}
\]
\medskip
\[
p\, dx = p\, dx + x\, dp + g'(p)\, dp = 0 \quad \Rightarrow \quad (x + g'(p)) dp = 0
\]

\medskip
\noindent
Posibles soluciones:
\medskip
\begin{itemize}
  \item Si $x + g'(p) = 0$:
  \[
  x = g'(p), \text{despejo p} \quad\Rightarrow\quad y = y(x) \quad \text{(solución singular)}.
  \]
  \item Si $dp = 0$: 
  \[
  p = \text{cte} = C \quad\Rightarrow\quad y = Cx + g(C) \quad \text{(solución general)}.
  \]
\end{itemize}

\subsection{EDO's de Riccatti}
\noindent
Las EDO’s de Riccati son ecuaciones diferenciales ordinarias de primer orden y no lineales.
Generalmente no se pueden resolver de forma directa, salvo que se conozca una solución particular.
Mediante un cambio de variable pueden transformarse en una ecuación de Bernoulli o lineal.

\noindent
\medskip
Formulación general: 
\[
y' + a(x)y^2 + b(x)y + c(x) = 0
\]
\begin{itemize}
    \item Si $a, b, c$ son constantes $\longrightarrow$ ecuaciones de variables separables:
\end{itemize}

\[
\frac{dy}{dx} = ay^2 + by + c 
\quad \Longrightarrow \quad 
\int \frac{dy}{ay^2 + by + c} = \int dx
\]
\newpage
\noindent
\begin{itemize}
    \item Si $a, b, c$ son funciones de $x$, en general no se puede integrar salvo que se conozca una solución particular $y_1$
\end{itemize}
\[
y_g(x) = y_1(x) + z(x) \quad \Longrightarrow \quad y_g' = y_1' + z'
\]
\vspace{-0,2 em}
\[
y_g^2(x) = y_1^2(x) + z^2(x) + 2y_1z \quad \Rightarrow \quad (y' + a(x)y^2 + b(x)y + c(x) = 0)
\]
\vspace{-0,2 em}
\[
y_1' + z' + a(x)(y_1^2 + z^2 + 2y_1z) + b(x)(y_1 + z) + c(x) = 0
\]
\vspace{-0,2 em}
\[
\big[y_1' + a(x)y_1^2 + b(x)y_1 + c(x)\big] + \big[z' + 2a(x)y_1 z + a(x)z^2 + b(x)z\big] = 0
\]

\noindent
La primera parte es 0, por lo tanto queda:
\[
z' + [2a(x)y_1 + b(x)]z = -a(x)z^2
\]
Ecuación de Bernoulli pura $z(x)$

\[
\frac{dy}{dx} + p(x)y = q(x)y^n
\]
\textbf{Ejemplo:}

\[
y' + y^2 - 2y\sen x + \sen^2 x - \cos x = 0 
\quad ; \quad y_1 = \sen x
\]
Entonces:
\[
\begin{cases}
a(x) = 1 \\
b(x) = -2\sen x \\
c(x) = \sen^2 x - \cos x
\end{cases}
\]
Resolvemos:
\[
y = \sen x + z(x) \quad \Longrightarrow \quad y' = \cos x + z'
\]
\vspace{-0,5 em}
\[
y' = \sen x + z'(x) + 2\sen x \, z(x)
\]
\vspace{-0,5 em}
\[
\cos x + z' + \sen^2 x + 2\sen x \, z - 2(\sen x + z)\sen x + \sen^2 x - \cos x = 0
\]
\vspace{-0,5 em}
\[
z' + z^2 + 2\sen x \, z - 2\sen x \, z = 0 \quad \Longrightarrow \quad z' + z^2 = 0 \quad \Longrightarrow \quad \frac{dz}{dx} = -z^2
\]
\newpage
\noindent
Integramos:

\[
\int \frac{dz}{z^2} = - \int dx 
\quad \Longrightarrow \quad 
-\frac{1}{z} = -x + c
\]

\[
z = \frac{1}{x + c}
\quad \Longrightarrow \quad 
y_g = \sen x + z = \sen x + \frac{1}{x + c}
\]

\subsection{Familias de curvas:}
\noindent
Las familias de curvas son conjuntos de funciones que dependen de un parámetro, expresadas como: \quad $y = y(x, c)$

\medskip
\noindent
Cada valor de c genera una curva diferente dentro de la familia.
Para obtener su ecuación diferencial, se elimina dicho parámetro mediante derivación y sustitución.

\medskip
\noindent
\textbf{Ejemplo:}
\[
y = c x \quad \Longrightarrow \quad y' = c \quad \Longrightarrow \quad y = y' x \quad \Longrightarrow \quad \frac{dy}{dx} = \frac{y}{x}
\]
\vspace{-0,8 em}
\[
y = a(1 - e^{-x/a})
\]
\vspace{-0,8 em}
\[
y' = e^{-x/a}\left( \frac{1}{a} \right)a = e^{-x/a} \quad \Longrightarrow \quad \ln y' = -\frac{x}{a} \quad \Longrightarrow \quad a = -\frac{x}{\ln y'}
\]

\[
y = -\frac{x}{\ln y'} \left(1 - e^{\frac{-x}{-x \ln y'}}\right) = \frac{-x}{ln y'}(1 - y') \quad \Longrightarrow \quad y \ln y' = - x(1 - y')
\]
\vspace{-0,2 em}
\[
x(1 - y') + y \ln y' = 0 \quad \Longrightarrow \quad \text{Ecuación diferencial de familia de curvas}
\]

\medskip
\noindent
Si tengo una ecuación diferencial de la forma $F(x, y, y', \ldots, y^{(n)}) = 0$, se encuentra la familia derivando $n$ veces la ecuación y eliminando los parámetros entre todas las ecuaciones.
\vspace{0,4 em}
\[
y = c_1 x + \frac{c_2}{x} + c_3 
\quad \Longrightarrow \quad y' = c_1 - \frac{c_2}{x^2}
\quad \Longrightarrow \quad y'' = \frac{2c_2}{x^3}
\quad \Longrightarrow \quad y''' = -\frac{6c_2}{x^4}
\]

\[
\frac{x^3 y''}{2} = \frac{-x^4 y'''}{6}
\quad \Longrightarrow \quad \frac{x y''}{3} + y''' = 0
\]

\[
c_2 = \frac{x^3 y''}{2}, \quad c_2 = \frac{-x^4 y'''}{6}
\]

\newpage
\subsection{Trayectorias ortogonales:}
\noindent
Dada una familia de curvas con ecuaciones $h(x, y, a) = 0$, toda otra curva perteneciente a otra familia que corte a cada una de éstas formando un mismo ángulo $\alpha = \frac{\pi}{2}$, se denomina \textit{trayectoria ortogonal}.  
Dos familias son ortogonales si todas lo son.

\[
y_1 = y_1(x, c_1) \quad \Longrightarrow \quad F(x, y, y_1') = 0
\]
La EDO de la segunda familia será:
\vspace{0,1 em}
\[
F\left(x, y, -\frac{1}{y'}\right) = 0 
\quad \Longrightarrow \quad y_2 = y_2(x, c) 
\quad \text{tal que} \quad y_1' \cdot y_2' = 1
\]

\[
h(x, y, c) = 0 
\quad \xrightarrow{\text{hallar las EDO's}} \quad 
F(x, y, y') = 0
\]
\vspace{0,1 em}
\[
\quad \xrightarrow{\text{planteo la EDO forma ortogonal}} \quad 
F\left(x, y, -\frac{1}{y'}\right) = 0 
\quad \xrightarrow{\text{resolver}} \quad 
g(x, y, k) = 0
\]

\newpage
\section{EDO's de segundo orden y superior}
\noindent
Se denomina EDO de orden $n$: 
\[
F(x, y, y', y'', \ldots, y'^{(n)}) = 0 
\quad \text{con} \quad y^{(n)} = \frac{d^n y}{dx^n}
\]
La solución general es una familia $n$–paramétrica: 
\[
y = y(x, c_1, c_2, \ldots, c_n)
\]
Un problema de valor inicial requiere $n$ condiciones iniciales.

\subsection*{Sistema de ecuaciones:}
\noindent
En las EDOs de segundo orden y superior, un sistema de ecuaciones establece las condiciones necesarias para determinar una solución única.  
Cada condición inicial, como \( y(x_0) \), \( y'(x_0) \), \( y''(x_0) \), etc., fija los valores de las constantes de integración.  
Por lo tanto, un problema de orden \( n \) requiere \( n \) condiciones iniciales para definir completamente la solución.
\[
y(x_1) = y_1, \quad y(x_2) = y_2, \quad y(x_3) = y_3, \ldots
\]
\[
y(x_0) = y_0, \quad y'(x_0) = y'_0, \quad y''(x_0) = y''_0, \ldots
\]

\subsection{EDOs lineales de segundo orden}
\noindent
Las EDOs lineales de segundo orden tienen la forma:
\[
y'' + P(x)y' + Q(x)y = R(x)
\]
Diferenciamos dos tipos:
\[
\begin{cases}
R(x) = 0 & \text{homogénea (EDOLH)} \\
R(x) \neq 0 & \text{inhomogénea (EDOLI)}
\end{cases}
\]

\subsection{Teorema de existencia y unicidad}
\noindent
El teorema de existencia y unicidad establece que una EDO lineal de segundo orden tiene una única solución si las funciones \( P(x) \), \( Q(x) \) y \( R(x) \) son continuas en un intervalo que contiene el punto inicial.  
Esto garantiza que el problema de Cauchy definido por \( y(x_0) \) y \( y'(x_0) \) posee una sola solución válida en dicho intervalo.

\medskip
\noindent
Dado un problema de Cauchy:
\[
\begin{cases}
y'' + P(x)y' + Q(x)y = R(x) \\
y(x_0) = y_0 \\
y'(x_0) = y'_0
\end{cases}
\]
$\exists !$ solución si $P(x)$, $Q(x)$ y $R(x)$ son continuas en $x \approx x_0$! 

\subsection{EDO equivalente}

\noindent
Las EDOs equivalentes se expresan en la forma operatorial \( L[y] = R(x) \), donde \( L \) es un operador diferencial que actúa sobre \( y \).  
La solución general se obtiene como la suma de la solución homogénea \( y_h \) (cuando \( R(x) = 0 \)) y una solución particular \( y_p \) (cuando \( R(x) \neq 0 \)).  
Por lo tanto, toda solución cumple \( y_g = y_h + y_p \).

\[
L[y] = R(x) \quad \text{con} \quad L = \frac{d^2}{dx^2} + P(x)\frac{d}{dx} + Q(x)
\]
Si $y_h$ es una solución de la EDOLH y $y_p$ es una solución de la EDOLI:

\[
y_g = y_h + y_p
\]

\[
L[y_h] = 0, \quad L[y_p] = R(x) \qquad \Rightarrow \qquad 
\begin{cases}
L[y_h] = 0 \\[6pt]
L[y_p] = R(x)
\end{cases}
\]
Resultando en:
\[
L[y_h + y_p] = L[y_h] + L[y_p] =0 + R(x) = R(x)
\]

\subsection{EDOLH}
\vspace{-1 em}
\[
y = 0 \quad \text{siempre es solución:} \quad L[0] = 0
\]
Si \( y_1 \) e \( y_2 \) son soluciones 
\[
C_1 y_1 + C_2 y_2 \quad\text{ también lo es.}
\]
\[
L(C_1 y_1 + C_2 y_2) = C_1 L(y_1) + C_2 L(y_2) = 0
\]

\newpage
\noindent
La combinación lineal de dos soluciones independientes es la solución general.  
Dos funciones son linealmente dependientes si una es múltiplo de la otra.

\[
\frac{y_2}{y_1 } = C  \quad \Longrightarrow \text{para no independientes.}
\]

\[
\text{Solución general} \Rightarrow y_g = C_1 y_1 + C_2 y_2
\]
Entonces: 
\[
y_g = C_1 y_1 + C_2 y_2 
\]
\[
\text{ deben satisfacer cualquier condición inicial:} 
\quad 
\begin{cases}
y(x_0) = y_0 \\
y'(x_0) = y'_0
\end{cases}
\]


\[
\begin{cases}
y_g(x_0) = C_1 y_1(x_0) + C_2 y_2(x_0) = y_0 \\
y'_g(x_0) = C_1 y'_1(x_0) + C_2 y'_2(x_0) = y'_0
\end{cases}
\quad \Rightarrow \text{saco } C_1 \text{ y } C_2
\]

\bigskip
\noindent
Aplico el Wronskiano:

\[
W = 
\begin{vmatrix}
y_1(x_0) & y_2(x_0) \\
y'_1(x_0) & y'_2(x_0)
\end{vmatrix}
\neq 0
\quad \text{para poder obtener } C_1 \text{ y } C_2
\]
Si \( y_1 \) e \( y_2 \) son soluciones, entonces \( W \) se anula siempre o nunca se anula.

\medskip
\noindent
Si:
\[
L[y_1] = L[y_2] = 0 \quad \text{en } [a,b] 
\quad \Rightarrow \quad 
W(x) = 
\begin{cases}
0 & \forall x \in [a,b] \\
\neq 0 & \forall x \in [a,b]
\end{cases}
\]

\[
W(x) = y_1 y'_2 - y'_1 y_2
\]

\[
W'(x) = y'_1 y'_2 + y_1 y''_2 - y''_1 y_2 - y'_1 y'_2 = y_1 y''_2 - y''_1 y_2
\]

\[
\begin{cases}
y''_1 + P(x)y'_1 + Q(x)y_1 = 0 \\
y''_2 + P(x)y'_2 + Q(x)y_2 = 0
\end{cases}
\Rightarrow 
y_1 y''_2 - y''_1 y_2 + P(y_1 y'_2 - y'_1 y_2) = 0
\]

\[
W'(x) + P(x)W(x) = 0
\]



















\newpage
\section{Ejercicios hoja 1}

\subsection{Ejercicio 1}
\vspace{-1.2em}
\[
\begin{aligned}
y' \sin x - y \cos x &= 0 
\quad ;\quad 
y'\!\left( \frac{\pi}{2} \right) = 1 \quad \rightarrow \quad \text{condición}
\end{aligned}
\]
Identificamos que se trata de una edo de variables separables, por lo tanto resolvemos 
separando las variables:

\[
y' = y \frac{\cos x}{\sin x}
\]

\[
\frac{dy}{y} = \frac{\cos x}{\sin x}dx
\quad \Rightarrow \quad
\int \frac{dy}{y} = \int \frac{\cos x}{\sin x}\, dx
\]

\medskip
\noindent
Hacemos el cambio de variable para resolver la integral con u = sin x:
\[
u = \sin x
\]
\[
du = \cos x\, dx
\]

\[
\ln |y| = \int \frac{du}{u} = \ln |u| + C
\]

\[
y = C \sin x \quad \rightarrow \quad y = \sin x
\]

\subsection{Ejercicio 11}
\vspace{-1.2em}
\[
x + y - 2 + (1 - x) y' = 0 \quad ; \quad \forall x \neq 0
\]

\[
y' + \frac{y}{1 - x} = \frac{2 - x}{1 - x}
\]

Ecuación homogénea:

\[
y_h' + \frac{y}{1 - x} = 0
\]

\[
\frac{dy}{dx} = \frac{-y}{1 - x}
\]
\newpage
\noindent
Integrando:

\[
\int \frac{dy}{y} = \int \frac{dx}{-1 + x}
\]

\[
\ln y = \ln(x - 1) + \ln C
\]

\[
y_h = C (x - 1)
\]
Sacamos una solución particular:

\[
y_p = C(x)(x - 1) \quad \Rightarrow \quad y_p' = C'(x)(x - 1) + C(x)
\]

\[
C'(x)(x - 1) + C(x) = \frac{2 - x}{1 - x}
\]

\[
C'(x)(x - 1)^2 = \frac{2 - x}{1 - x} (1 - x)^2
\]

\[
C'(x) = \frac{2 - x}{(1 - x)^2}
\]

\[
C(x) = \int \frac{2 - x}{(1 - x)^2}\, dx
\]
Hacemos el cambio de variable \(u = 1 - x\):
\[
u = 1 - x \quad du = -dx
\]

\[
C(x) = \int \frac{u + 1}{u^2}\, du = \int \frac{du}{u} + \int \frac{du}{u^2} = \ln u - \frac{1}{u} + D
\]

\[
C(x) = \ln(1 - x) + \frac{1}{x - 1} + D
\]

\[
y_p = C(x)(x - 1) = \ln(1 - x) + \frac{1}{x - 1} + D
\]
\newpage
\noindent
\subsection*{12.)}
\[
8x + 4y + 1 + (4x + 2y + 1)y' = 0
\]
Reescribimos la ecuación:

\[
\frac{dy}{dx} = \frac{-8x - 4y - 1}{4x + 2y + 1}
\]

\[
y' = \frac{-8x - 4y - 1}{4x + 2y + 1} \quad \Rightarrow \quad z = 4x + 2y + 1
\]

\[
\frac{dz}{dx} = 4 + 2y'
\]

\[
y' = \frac{1}{2} \frac{dz}{dx} - 2
\]

\[
\frac{1}{2}\frac{dz}{dx} - 2 = \frac{-2z + 1}{z}
\]

\[
\frac{dz}{dx} = \frac{-4z + 2}{z} + 4
\]

\[
\frac{dz}{dx} = \frac{-4z + 2 + 4z}{z} = \frac{2}{z}
\]

\[
z\, dz = 2 dx
\]

\[
\int z\, dz = \int 2 dx
\]

\[
\frac{z^2}{2} = 2x + C
\]

Deshacemos el cambio de variable:

\[
(4x + 2y + 1)^2 = 4x + C
\]

\[
16x^2 + 16xy + 4x + 4y^2 + 4y + 1 = 4x + C
\]

\[
4x^2 + 4xy + y^2 + 2y + \frac{1}{4} = C
\]

---

\subsection*{13.}
\[
\frac{x}{\sqrt{x^2 + y^2}}\, dx + \frac{y}{\sqrt{x^2 + y^2}}\, dy + \frac{1}{y} dx + \frac{x}{y^2} dy = 0
\]

Comprobamos que sea exacta:

\[
M = \frac{x}{\sqrt{x^2 + y^2}} + \frac{1}{y}
\]
\[
N = \frac{y}{\sqrt{x^2 + y^2}} + \frac{x}{y^2}
\]

\[
\frac{\partial M}{\partial y} = \dots \quad \frac{\partial N}{\partial x} = \dots
\]

Si es exacta:

\[
g(x,y) = \int M\, dx + C
\]

\[
\int M\, dx = \int \frac{x}{\sqrt{x^2 + y^2}} dx + \int \frac{1}{y} dx
\]

\[
u = x^2 + y^2 \quad du = 2x\, dx
\]

\[
\int \frac{x}{\sqrt{x^2 + y^2}} dx = \sqrt{x^2 + y^2}
\]

\[
\int \frac{1}{y} dx = \frac{x}{y}
\]

\[
g(x,y) = \sqrt{x^2 + y^2} + \frac{x}{y} + C
\]

\[
|x| + \frac{x}{y} + \sqrt{x^2 + y^2} + C = 0
\]
\subsection*{15.}
\[
(x - y)dx + x\, dy = 0
\]

\[
M = x - y \quad N = x
\]

Comprobamos si es exacta:

\[
\frac{\partial M}{\partial y} = -1 \quad \frac{\partial N}{\partial x} = 1 \quad \Rightarrow \quad \text{No es exacta}
\]

Buscamos un factor integrante:

\[
\mu': \frac{\partial_y M - \partial_x N}{N} = \frac{-1 - 1}{x} = \frac{-2}{x} \quad \Rightarrow \quad \mu = \mu(x)
\]

\[
\int d\mu = \int \frac{-2}{x} dx \quad \Rightarrow \quad \ln|\mu| = -2 \ln|x| + C \quad \Rightarrow \quad \mu = \frac{C}{x^2}
\]

Multiplicamos la ecuación original por $\mu$:

\[
\frac{C}{x^2}(x - y) dx + \frac{C}{x^2} x\, dy = 0
\]

\[
\frac{\partial M}{\partial y} = -\frac{1}{x^2} \quad \frac{\partial N}{\partial x} = -\frac{1}{x^2}
\]

Ahora es exacta:

\[
\frac{\partial f}{\partial x} = -\frac{y}{x^2} + g'(y)
\]

\[
M = \frac{\partial f}{\partial x} = \frac{-y}{x^2} + g'(y) = \frac{1}{x} - \frac{y}{x^2}
\]

\[
g'(y) = \frac{1}{x} \quad g(y) = \int \frac{1}{x} dx = \ln|x| + D
\]

\[
f(x,y) = \frac{y}{x} + \ln|x| + D
\]


\subsection*{16.}
\[
(x^4 \ln x - 2x y^3) dx + 3x^2 y^2 dy = 0
\]

Comprobamos si es exacta:

\[
\frac{\partial M}{\partial y} = -6xy^2 \quad \frac{\partial N}{\partial x} = 6xy^2 \quad \Rightarrow \quad \text{No es exacta}
\]

\[
\frac{\partial_y M - \partial_x N}{N} = \frac{-6xy^2 - 6xy^2}{3x^2 y^2} = -\frac{4}{x}
\]

\[
\mu = \mu(x) \quad \ln \mu = -4 \ln x \quad \Rightarrow \quad \mu = \frac{C}{x^4}
\]

Multiplicamos por $\mu$:

\[
\frac{\ln x}{x^4} dx - \frac{2y^3}{x^3} dx + \frac{3y^2}{x^2} dy = 0
\]
Comprobamos si ahora es exacta:

\[
M = \ln x - \frac{2y^3}{x^3} 
\quad\quad 
N = \frac{3y^2}{x^2}
\]

Hacemos las parciales:

\[
\frac{\partial M}{\partial y} = -\frac{6y^2}{x^3} 
\quad 
\frac{\partial N}{\partial x} = -\frac{6y^2}{x^3}
\]

Es exacta.

\[
\int M\, dx = \int \ln x\, dx - \int \frac{2y^3}{x^3}\, dx
\]

\[
g'(y) = \int \frac{3y^2}{x^2}\, dy = \frac{y^3}{x^2}
\]

\[
g(y) = \int \ln x\, dx
\]

\[
g(y) = x (\ln x - 1)
\]

\[
f(x,y) = \frac{y^3}{x^2} + x(\ln x - 1) + D
\]

---

\subsection*{22.}
\[
\frac{dy}{dx} = \frac{y}{x}, \quad y(0) = 0
\]

\[
\frac{dy}{y} = \frac{dx}{x}
\]

\[
\int \frac{dy}{y} = \int \frac{dx}{x}
\]

\[
\ln |y| = \ln |x| + C
\]

\[
y = Cx
\]

$\Rightarrow$ Familia de rectas que pasan por $(0,0)$.

En el caso de $y(0) = 1$:

\[
1 = C \cdot 0 = 0
\]
No existe solución.

$\Rightarrow$ No está en la familia de curvas.

---



\end{document}