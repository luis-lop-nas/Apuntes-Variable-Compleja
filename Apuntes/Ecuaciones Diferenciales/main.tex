% --- Idioma y codificación ---
\usepackage[utf8]{inputenc}   % Acentos directos
\usepackage[T1]{fontenc}      % Codificación de salida
\usepackage[spanish]{babel}   % Español
\usepackage{lmodern}          % Fuente moderna

% --- Matemáticas ---
\usepackage{amsmath}          % Entornos matemáticos
\usepackage{amssymb}          % Símbolos
\usepackage{amsthm}           % Teoremas
\usepackage{mathtools}        % Extras de amsmath
\usepackage{bm}               % Negrita en símbolos
\usepackage{esint}            % Integrales de contorno (∮)

% --- Utilidades ---
\usepackage{graphicx}         % Imágenes
\usepackage{xcolor}           % Colores
\usepackage{enumitem}         % Listas
\usepackage{titlesec}         % Títulos
\usepackage[hidelinks]{hyperref} % Enlaces (cargar al final)

% --- Esquemas y Dibujos ---
\usepackage{tikz}
\usetikzlibrary{arrows.meta}

% subsecciones más pequeñas
\titleformat{\subsection}
  {\normalfont\small\bfseries}   % formato más pequeño y en negrita
  {\thesubsection}{1em}{} 

\title{Apuntes Ecuaciones Diferenciales}
\author{Luis López}
\date{September 2025}

\begin{document}

\maketitle % Genera la portada

\newpage % Salto de página después de la portada

\section*{Introducción a las Ecuaciones Diferenciales}

Una \textbf{ecuación diferencial} es una ecuación matemática que relaciona una función con sus derivadas. 
En matemáticas aplicadas, las funciones suelen representar \textit{cantidades físicas}, las derivadas 
indican sus \textit{tasas de cambio}, y la ecuación define la relación entre ambas. 
Debido a que estas relaciones aparecen de forma natural en múltiples contextos, 
las ecuaciones diferenciales juegan un rol esencial en la \textit{ingeniería}, 
la \textit{física}, la \textit{química}, la \textit{economía} y la \textit{biología}.

\subsection*{Tipos principales de ecuaciones diferenciales}

De forma general, se distinguen dos grandes clases de ecuaciones diferenciales:

\begin{itemize}
    \item \textbf{Ecuaciones Diferenciales Ordinarias (EDO):} son aquellas en las que interviene una 
    única variable independiente. Suelen escribirse en la forma
    \[
        F\left(x, y, y', y'', \dots, y^{(n)}\right) = 0,
    \]
    donde $y = y(x)$ es la función incógnita.
    
    \item \textbf{Ecuaciones Diferenciales en Derivadas Parciales (EDP):} aparecen cuando la función 
    incógnita depende de dos o más variables independientes. En ellas intervienen derivadas parciales, por ejemplo:
    \[
        F\left(x, y, u, \frac{\partial u}{\partial x}, \frac{\partial u}{\partial y}, \dots \right) = 0,
    \]
    donde $u = u(x,y)$ es la función desconocida.
\end{itemize}

\subsection*{Usos de las ecuaciones diferenciales}

Las ecuaciones diferenciales permiten modelar una gran variedad de fenómenos, tales como:

\begin{itemize}
    \item El crecimiento poblacional en biología.
    \item La transferencia de calor y difusión en física e ingeniería.
    \item El movimiento de partículas bajo fuerzas externas.
    \item La dinámica de circuitos eléctricos.
    \item Modelos económicos de inversión y consumo.
\end{itemize}

En estos apuntes estudiaremos principalmente las \textbf{Ecuaciones Diferenciales Ordinarias (EDO)}, 
ya que constituyen la base fundamental para comprender el análisis de sistemas dinámicos.  
Más adelante, daremos una primera aproximación a las \textbf{Ecuaciones en Derivadas Parciales (EDP)}, 
que amplían el marco teórico y permiten describir fenómenos de mayor complejidad.

\newpage

\section*{Ecuaciones Diferenciales}

Se denomina \textbf{ecuación diferencial ordinaria (EDO)} a toda ecuación de la forma
\[
    F\left(x, y, y', y'', \dots, y^{(n)}\right) = 0,
\]
donde $x$ es la \textit{variable independiente}, $y$ es la \textit{variable dependiente} e 
$y', y'', \dots, y^{(n)}$ son las derivadas de $y$ respecto a $x$.  

El \textbf{orden de la ecuación} es el grado de la derivada de mayor orden que aparece en ella.  

\subsection*{Solución de una EDO}
Se denomina \textbf{solución de una EDO en $(a,b)$} a toda función $y(x)$ definida en $(a,b)$, 
junto con sus derivadas hasta el orden necesario, tal que al sustituirla en la ecuación diferencial 
se obtiene una identidad respecto de $x$ en el intervalo $(a,b)$.  

\subsection*{Ejemplos}
\begin{itemize}
    \item $y' = x \quad \Rightarrow \quad y = \tfrac{1}{2}x^2 + C$ \\
    Ecuación de primer orden con una constante arbitraria.
    
    \item $y' = y \quad \Rightarrow \quad y = Ce^{x}$ \\
    Ecuación de primer orden con solución exponencial.
\end{itemize}

\subsection*{EDP vs EDO}
Si una ecuación diferencial involucra más de una variable independiente, se denomina 
\textbf{ecuación en derivadas parciales (EDP)}.  

A la gráfica de una solución de una EDO también se le llama \textbf{integral de la ecuación}.  

\subsection*{EDO de primer orden}
La forma más general de una EDO de primer orden es:
\[
    F(x, y, y') = 0 \quad \text{(forma implícita)},
\]
o bien,
\[
    y' = f(x,y) \quad \text{(forma explícita)}.
\]

Se dice que una EDO es \textbf{autónoma} si no depende explícitamente de la variable independiente $x$, 
es decir:
\[
    F(y, y') = 0 \quad \text{o bien} \quad y' = f(y).
\]

\subsection*{Problema de Cauchy}
El problema de Cauchy (también llamado problema de valor inicial) consiste en resolver una ecuación diferencial ordinaria junto con una o más condiciones iniciales, que son valores específicos de la solución y sus derivadas en un punto dado, usualmente el instante inicial.\\

Las condiciones iniciales en las ecuaciones diferenciales sirven para dar valores a las constantes que van apareciendo al resolverlas.
\[
    y' = f(x,y),
\]
tal que 
\[
    y(x_0) = y_0,
\]
es decir, se especifica una \textit{condición inicial} en el punto $(x_0,y_0)$.

\end{document}