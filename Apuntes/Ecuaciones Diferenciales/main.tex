\documentclass[a4paper,12pt]{article}
% --- Márgenes y espaciado global ---
\usepackage[a4paper,margin=3.5cm]{geometry}
\usepackage{setspace}
\setstretch{1.3}
\raggedbottom % Evita que LaTeX estire verticalmente el contenido entre páginas
\setlength{\parindent}{0pt}
\setlength{\parskip}{0.8em}
\usepackage{microtype} % Mejora la justificación y kerning
% --- Idioma y codificación ---
\usepackage[utf8]{inputenc}   % Acentos directos
\usepackage[T1]{fontenc}      % Codificación de salida
\usepackage[spanish]{babel}   % Español
\usepackage{lmodern}          % Fuente moderna

% --- Matemáticas ---
\usepackage{amsmath}          % Entornos matemáticos
\usepackage{amssymb}          % Símbolos
\usepackage{amsthm}           % Teoremas
\usepackage{mathtools}        % Extras de amsmath
\usepackage{bm}               % Negrita en símbolos
\usepackage{esint}            % Integrales de contorno (∮)

% --- Utilidades ---
\usepackage{graphicx}         % Imágenes
\usepackage{xcolor}           % Colores
\usepackage{enumitem}         % Listas
\usepackage{titlesec}         % Títulos
% Espaciado consistente de secciones y subsecciones
\titlespacing*{\section}{0pt}{1.2em}{0.6em}
\titlespacing*{\subsection}{0pt}{1.0em}{0.3em}
\titlespacing*{\subsubsection}{0pt}{0.8em}{0.4em}

% --- Esquemas y Dibujos ---
\usepackage{tikz}
\usetikzlibrary{arrows.meta}

% subsecciones más pequeñas
\titleformat{\subsection}
  {\normalfont\small\bfseries}   % formato más pequeño y en negrita
  {\thesubsection}{1em}{} 

\title{Apuntes Ecuaciones Diferenciales}
\author{Luis López}
\date{September 2025}

% --- Enlaces (cargar al final del preámbulo) ---
\usepackage[hidelinks]{hyperref}

\begin{document}

\maketitle % Genera la portada
\clearpage

\tableofcontents % Índice automático
\clearpage
\section*{Introducción a las ecuaciones diferenciales}


Una \textbf{ecuación diferencial} es una relación matemática en la que intervienen una función desconocida y sus derivadas. 
Su objetivo es describir cómo varía una magnitud en función de otra, estableciendo vínculos entre la propia magnitud y su tasa de cambio.  

Las ecuaciones diferenciales son fundamentales porque numerosos fenómenos naturales y tecnológicos se formulan de esta manera. 
En física, por ejemplo, permiten estudiar el movimiento de partículas (leyes de Newton), la propagación del calor, las ondas o el comportamiento de los circuitos eléctricos. 
En otras disciplinas también son esenciales: en biología modelan el crecimiento de poblaciones y la difusión de sustancias, 
en economía describen la evolución de sistemas dinámicos, 
y en ingeniería sirven para analizar vibraciones, fluidos o sistemas de control.  

En definitiva, las ecuaciones diferenciales constituyen una herramienta matemática clave que conecta las teorías con la realidad dinámica, 
pues gran parte de los procesos que nos rodean se fundamentan en cambios que pueden expresarse mediante estas ecuaciones.

\clearpage
\section{Ecuaciones diferenciales ordinarias}

Se denomina \textbf{ecuación diferencial ordinaria (EDO)} a toda ecuación
\[
F(x,y,y',y'',\ldots,y^{(n)}) = 0
\]
donde $x$ es la variable independiente, $y$ la variable dependiente y $y',y'',\ldots$ son derivadas de la variable dependiente.  
Llamamos \textbf{orden de la ecuación} al grado de la derivada más grande que aparece en la ecuación.  

Se denomina \textbf{solución} de una EDO en $(a,b)$ a toda función definida en $(a,b)$ junto a sus derivadas de orden $n$, tal que al sustituirla en la EDO da una identidad respecto a $x$ en $(a,b)$.

\textbf{Ejemplo:}
\[
y' = x^5 \;\;\Rightarrow\;\; y = \tfrac{x^6}{6} + C, \quad \text{ecuación de primer orden $\Rightarrow$ 1 constante arbitraria}
\]
\[
y'' = y' \;\;\Rightarrow\;\; y = C_1 e^x + C_2
\]

Si una EDO involucra varias variables independientes, se denomina \textbf{ecuación en derivadas parciales (EDP)}:
\[
F(x_1,x_2,\ldots,x_n, \tfrac{\partial y}{\partial x_1}, \tfrac{\partial y}{\partial x_2}, \ldots, \tfrac{\partial^2 y}{\partial x_i \partial x_j}, \ldots ) = 0
\]

A la gráfica de una solución de una EDO también se le llama \textbf{integral}.  

\subsection*{EDOs de primer orden}

La forma más general de una EDO de primer orden es:
\[
F(x,y,y')=0 \quad \text{(forma implícita)}
\]
\[
y' = f(x,y) \quad \text{(forma explícita)}.
\]

Se dice que una EDO es \textbf{autónoma} si no depende explícitamente de la variable independiente, es decir:
\[
y' = f(y) \quad \Leftrightarrow \quad F(y,y')=0.
\]
\subsection{Problema de Cauchy}

Consiste en la búsqueda de la solución de la EDO $y' = f(x,y)$ tal que $y(x_0)=y_0$,
es decir, se añade una \textbf{condición inicial}:
\[
\begin{cases}
y' = f(x,y) \\
y(x_0) = y_0
\end{cases}
\]

\textbf{Ejemplo:}
\[
y' = y \;\;\Rightarrow\;\; y = C e^x, \qquad y=e^x \text{ si } y(0)=1
\]
Ya que $1 = C e^0 \Rightarrow C=1$.

\subsection{Teorema de Picard–Lindelöf (existencia y unicidad)}
El problema de Cauchy tiene una única solución si, y solo si, $f(x,y)$ y
$\dfrac{\partial f}{\partial y}$ son continuas en $(x,y)$.

Es decir, si $f(x,y)$ y $\dfrac{\partial f}{\partial y}$ son continuas en una cierta región del plano $x,y$,  
entonces $\exists! \; y(x)$ que pasa por cada punto de dicha región.

\textbf{Ejemplo:}
\[
y' = \frac{y}{x} \;\;\Rightarrow\;\; \frac{dy}{y} = \frac{dx}{x}
\;\;\Rightarrow\;\; \int \frac{dy}{y} = \int \frac{dx}{x}
\;\;\Rightarrow\;\; \ln y = \ln x + C \;\;\Rightarrow\;\; y = Cx
\]

Si $y(x_0)=y_0$ entonces $y_0 = Cx_0 \;\;\Rightarrow\;\; C = \dfrac{y_0}{x_0}$.

\medskip

Por definición, la \textbf{solución general} de una EDO es la familia de curvas $y(x,C)$ que satisfacen $y'=f(x,y)$ para distintos valores admisibles de $C$.  

Una \textbf{solución particular} es aquella que además cumple la condición inicial $y(x_0)=y_0$,  
es decir, es una curva concreta dentro de la familia general.

\[
y' = y, \quad y = C e^x \;\; \Rightarrow \;\; \text{solución general}
\]
\[
y_p = e^x \;\; \Rightarrow \;\; \text{solución particular}
\]

\section{Tipos de EDOs}

\subsection{EDOs de variables separables}

Una EDO de variables separables tiene la forma:
\[
\frac{dy}{dx} = f(x)g(y),
\]
que se puede reescribir como:
\[
\frac{1}{g(y)}\,dy = f(x)\,dx.
\]

\textbf{Ejemplo:}
\[
3 e^x \tan(y)\, dx = -(2 - e^x)\sec^2(y)\, dy
\]

Integramos en ambos lados:
\[
\int \frac{e^x}{2-e^x}\,dx = \int -\sec^2(y)\,dy
\]

Resolviendo:
\[
-3 \ln|2-e^x| = -\ln|\tan y| + C
\]

\[
\tan(y) = D (e^x-2)^3 \;\;\Rightarrow\;\; y = \arctan\!\big(D(e^x-2)^3\big).
\]

\subsection{EDOs reducibles a separables}

Son del tipo:
\[
\frac{dy}{dx} = f(ax+by+c),
\qquad a,b,c \;\; \text{constantes definidas}.
\]

Se resuelven mediante el cambio de variable:
\[
z = ax + by + c.
\]

Entonces:
\[
z' = a + b\,y' 
\;\;\Rightarrow\;\; 
y' = \frac{z'-a}{b}.
\]

Sustituyendo en la ecuación original:
\[
\frac{dz}{dx} = a + b\,f(z),
\]
que es una ecuación separable en $z$ y $x$.

\bigskip

\textbf{Ejemplo:}

\[
(x+y^2)\,y' = a^2, \qquad a = \text{cte}.
\]

Reordenamos:
\[
y' = \frac{a^2}{(x+y)^2}.
\]

Hacemos el cambio de variable:
\[
z = x+y \quad\Rightarrow\quad z' = 1 + y' \quad\Rightarrow\quad y' = z'-1.
\]

Sustituyendo:
\[
z'-1 = \frac{a^2}{z^2} \quad\Rightarrow\quad z' = 1 + \frac{a^2}{z^2}.
\]

\[
\frac{dz}{dx} = \frac{z^2+a^2}{z^2}.
\]

Integramos:
\[
\int \frac{z^2}{z^2+a^2}\,dz = \int dx.
\]

Descomponemos:
\[
\int \frac{z^2}{z^2+a^2}\,dz 
= \int \left(1 - \frac{a^2}{z^2+a^2}\right)\,dz
= z - a^2\int \frac{1}{z^2+a^2}\,dz.
\]

La integral se resuelve con la identidad:
\[
\int \frac{1}{z^2+a^2}\,dz = \frac{1}{a}\arctan\!\left(\frac{z}{a}\right).
\]

Entonces:
\[
x + D = z - a\arctan\!\left(\frac{z}{a}\right).
\]
\subsection*{Deshacemos el cambio de variable $z = x+y$}

\[
z - \arctg\!\left(\tfrac{z}{a}\right) = x + D
\]

\[
x+y - \arctg\!\left(\tfrac{x+y}{a}\right) = x + D
\;\;\Rightarrow\;\; 
Si: y(0)=0 \;\;\Rightarrow\;\; D=0
\]

\[
y_g = \arctg\!\left(\tfrac{x+y}{a}\right) + D 
\;\;\Rightarrow\;\;
y_p = \arctg\!\left(\tfrac{x+y}{a}\right)
\]

\subsection{EDO’s homogéneas}

Se dice que $f(x,y)$ es homogénea de grado $n$ si:
\[
f(tx,ty) = t^n f(x,y)
\]

Se dice que $y' = f(x,y)$ es homogénea si $f(x,y)$ es homogénea de grado 0:
\[
f(tx,ty) = f(x,y), 
\qquad 
f(x,y) = \tfrac{y}{x},
\qquad
f(tx,ty) = \tfrac{ty}{tx} = \tfrac{y}{x}
\]

\[
f(x,y) = xy \;\;\Rightarrow\;\; f(tx,ty) = t^2 xy
\]

\textbf{Ejemplo:}

\[
f(x,y) = \left(\tfrac{x}{y}\right)^2 \;\;\Rightarrow\;\; \text{grado 0}
\]

\[
x y' = \sqrt{x^2 + y^2} + y 
\;\;\Rightarrow\;\;
y' = \frac{\sqrt{x^2 + y^2}}{x} + \frac{y}{x}
\]

\[
\;\;\Rightarrow\;\; y' = \sqrt{1 + \left(\tfrac{y}{x}\right)^2} + \tfrac{y}{x}
\]

Donde $z = \tfrac{y}{x} \;\;\Rightarrow\;\; z' = \tfrac{y'x - y}{x^2} \;\;\Rightarrow\;\; y' = x z' + y$

\subsection{Cambio de variable lineal}

Si $c \neq 0$ y 
\[
\begin{vmatrix}
a & b \\
a_1 & b_1
\end{vmatrix} \neq 0
\;\;\Rightarrow\;\;
\begin{cases}
x = \xi + h \\
y = \eta + k
\end{cases}
\]

\[
y' = \frac{dy}{dx} = \frac{d\eta}{d\xi}\cdot \frac{d\xi}{dx} 
= \frac{d\eta}{d\xi} \;\;\Rightarrow\;\; y' = \frac{d\eta}{d\xi}
\]

\[
\frac{d\eta}{d\xi} = 
\frac{a\xi + b\eta + ah + bk + c}{a_1\xi + b_1\eta + a_1 h + b_1 k + c_1}
= \frac{a + b\frac{\eta}{\xi}}{a_1 + b_1\frac{\eta}{\xi}}
= \frac{dw}{d\xi}
\]

\textbf{Ejemplo:}

\[
(x+y-2)\,dx + (x-y+4)\,dy = 0
\;\;\Rightarrow\;\;
y' = \frac{x+y-2}{-x+y-4}
\]

Hacemos el cambio de variable:
\[
\begin{cases}
x = \xi + h \\
y = \eta + k
\end{cases}
\]

\[
\frac{d\eta}{d\xi} = \frac{\xi + h + \eta + k - 2}{-\xi - h + \eta + k - 4}
\]

Condiciones:
\[
h+k=2, 
\qquad 
-h+k=4
\;\;\Rightarrow\;\;
h=-1, \; k=3
\]

\[
\frac{d\eta}{d\xi} = \frac{\xi + \eta + 2}{-\xi + \eta}
= 1 + \frac{2\xi}{-\xi + \eta}
\]

Cambio de variable:
\[
z = \frac{\eta}{\xi} \;\;\Rightarrow\;\; \eta = z\xi
\]

\[
\frac{d\eta}{d\xi} = \frac{d}{d\xi}(\xi z) = \xi \frac{dz}{d\xi} + z
\]

\[
\frac{1+z}{-1+z} = \frac{dz}{d\xi}\,\xi
\;\;\Rightarrow\;\;
\frac{1+z}{-1+z} = -\frac{dz}{d\xi}\,\xi
\]

Integramos:
\[
\int \frac{z-1}{-z^2+2z+1}\,dz = \int \frac{d\xi}{\xi}
\;\;\Rightarrow\;\;
\ln|1+2z-z^2| = \ln|\xi^{-2}| + C = \ln\left|\frac{D}{\xi^2}\right|
\]

\[
1+2z-z^2 = \frac{D}{\xi^2}
\;\;\Rightarrow\;\;
\xi^2 + 2z\xi^2 - z^2\xi^2 = D
\]

\[
(x+1)^2 + 2(x+1)(y-3) - (y-3)^2 = D
\]

\subsection{EDOs exactas}

Son de la forma: $f(x,y)=c \;\;\Rightarrow\;\; y'=0$

\[
df = \frac{\partial f}{\partial x}\,dx + \frac{\partial f}{\partial y}\,dy = 0
\;\;\Rightarrow\;\;
\frac{dy}{dx} = -\frac{\frac{\partial f}{\partial x}}{\frac{\partial f}{\partial y}}
\]

\[
\underbrace{M}_{\frac{\partial f}{\partial x}}(x,y)\,dx + 
\underbrace{N}_{\frac{\partial f}{\partial y}}(x,y)\,dy
\;\;\Leftrightarrow\;\; 
\exists f \;\; \text{tal que } f(x,y)=\text{constante}.
\]

\[
\frac{\partial f}{\partial x}=M \;\;\Rightarrow\;\; 
f(x,y) = \int M\,dx + g(y)
\]

\[
\frac{\partial f}{\partial y}=N \;\;\Rightarrow\;\;
\int \frac{\partial M}{\partial y}\,dx + g'(y)
\]

Por lo tanto,
\[
g'(y) = N - \int \frac{\partial M}{\partial y}\,dx 
\;\;\Rightarrow\;\;
g(y) = \int\left[N - \int \frac{\partial M}{\partial y}\,dx\right]dy + C
\]

\subsubsection*{“Demostración”}

Podemos hacer esto ya que si $f(x,y)=c \;\;\Rightarrow\;\; df=0$,  
entonces
\[
\frac{\partial f}{\partial x}\,dx + \frac{\partial f}{\partial y}\,dy = 0.
\]

Entonces:
\[
\frac{\partial^2 f}{\partial x \partial y} = \frac{\partial^2 f}{\partial y \partial x}
\;\;\Rightarrow\;\;
\frac{\partial}{\partial x}\!\left(\frac{\partial f}{\partial y}\right)
= \frac{\partial}{\partial y}\!\left(\frac{\partial f}{\partial x}\right)
\;\;\Rightarrow\;\;
\frac{\partial N}{\partial x} = \frac{\partial M}{\partial y}.
\]


\textbf{Ejemplo:}
\[
(x+y-2)\,dx + (x-y+4)\,dy = 0
\]

Llamamos \(M\) a la primera parte y \(N\) a la segunda:
\[
M=(x+y-2)\,dx, \qquad N=(x-y+4)\,dy
\]
\textbf{Comprobamos:}
\[
\frac{\partial M}{\partial y}=1  
\qquad 
\frac{\partial N}{\partial x}=1
\]

\[
\frac{\partial f}{\partial x}=x+y-2
\]

\[
\Rightarrow\;
f=\int (x+y-2)\,dx + g(y)
= \frac{x^{2}}{2}+xy-2x+g(y)
\]

\[
N=\frac{\partial f}{\partial y}=x+g'(y)=x-y+4
\;\Rightarrow\;
g(y)=\int(-y+4)\,dy=-\frac{y^{2}}{2}+4y+C
\]
\[
f(x,y)=\frac{x^{2}}{2}-2x-\frac{y^{2}}{2}+4y+xy+C=0
\]

\textbf{Ejemplo:}

\[
(\sen(xy)+xy\cos(xy))\,dx+x^{2}\cos(xy)\,dy=0
\]

\[
\begin{cases}
(\sen(xy)+xy\cos(xy))\,dx \;\;\longrightarrow\; M\\[2pt]
x^{2}\cos(xy)\,dy \;\;\longrightarrow\; N
\end{cases}
\]

Comprobamos que sea exacta.

\[
\frac{\partial M}{\partial y}
= x\cos(xy) + x\cos(xy) + xy(-x\sen(xy)),
\qquad
\frac{\partial N}{\partial x}
= 2x\cos(xy) - x^{2}y\sen(xy)
\]
\[
\longrightarrow\; \text{Queda comprobado que es una EDO exacta ya que son iguales.}
\]

\textbf{Resolvemos:}
\[
f=\int \big[x^{2}\cos(xy)\big]\,dy + g(x) = -x\,\sen(xy) + g(x)
\]

\[
\frac{\partial f}{\partial x}=M
\;\Rightarrow\;
-\sen(xy)-xy\cos(xy)+g'(x) = \sen(xy)+xy\cos(xy)
\]

\[
g'(x)=2\sen(xy)+2xy\cos(xy)
\]

\[
g(x)=\int \big[\,2\sen(xy)+2xy\cos(xy)\,\big]\,dx
\]
\subsection{EDO's cuasi-exactas}

\[
M\,dx + N\,dy = 0 \;\;\Rightarrow\;\; \frac{\partial M}{\partial y} \neq \frac{\partial N}{\partial x}
\qquad
\mu(x,y)=\text{factor integrante.}
\]

Si $\;\exists\,\mu(x,y)$ tal que $\;\mu(x,y)M\,dx+\mu(x,y)N\,dy=0\;$ es una EDO exacta, entonces la EDO original es cuasi-exacta.

\medskip
Supongamos que $\;\exists\,f(x,y)=c,\;\; \frac{\partial f}{\partial x}=\mu M\;\; \text{y}\;\; \frac{\partial f}{\partial y}=\mu N$

\[
\frac{\partial^2 f}{\partial y\,\partial x}=\frac{\partial^2 f}{\partial x\,\partial y}
\;\;\Rightarrow\;\;
\frac{\partial}{\partial y}(\mu M)=\frac{\partial}{\partial x}(\mu N)
\]

\[
N\,\frac{\partial \mu}{\partial x}+\mu\,\frac{\partial N}{\partial x}
=
M\,\frac{\partial \mu}{\partial y}+\mu\,\frac{\partial M}{\partial y}
\;\;\Rightarrow\;\;
N\,\frac{\partial \mu}{\partial x}-M\,\frac{\partial \mu}{\partial y}
=
\mu\!\left(\frac{\partial M}{\partial y}-\frac{\partial N}{\partial x}\right)
\]

\begin{itemize}
\item Si: $\displaystyle \frac{\partial_y M-\partial_x N}{N}$ solo depende de $x$ $\Rightarrow \mu=\mu(x)$
\end{itemize}

\[
\begin{gathered}
\mu = \mu(x) \Rightarrow\quad \frac{\partial \mu}{\partial y} = 0 
\Rightarrow\quad N\,\frac{\partial \mu}{\partial x} 
= \mu\!\left(\frac{\partial M}{\partial y} - \frac{\partial N}{\partial x}\right) \\[4pt]
\Rightarrow\quad 
\frac{d\mu}{\mu} =
\underbrace{\left(\frac{\partial M}{\partial y} - \frac{\partial N}{\partial x}\right)\!\Big/ N}_{g(x)}\, dx
\end{gathered}
\]
\[
\int \frac{d\mu}{\mu}=\int g(x)\,dx
\;\Rightarrow\;
\ln\mu=\int g(x)\,dx+\alpha
\;\Rightarrow\;
\mu(x)=c\,e^{\int g(x)\,dx}
\]

\begin{itemize}
\item Si: $\displaystyle \frac{\partial_x N-\partial_y M}{M}$ solo depende de $y$ $\Rightarrow \mu=\mu(y)$
\end{itemize}

\[
\mu(y)=e^{\int h(y)\,dy}
\qquad\text{con}\qquad
h(y)=\frac{\partial_x N-\partial_y M}{M}
\]

\textbf{Ejemplo:}

\[
(x+y^2)\,dx - 2xy\,dy = 0
\]

\[
\frac{\partial M}{\partial y} = 2y, 
\qquad 
\frac{\partial N}{\partial x} = -2y
\;\;\Rightarrow\;\; \text{No es exacta.}
\]

\[
\frac{\partial_y M - \partial_x N}{N} = \frac{2y - (-2y)}{-2xy} = \frac{4y}{-2xy} = -\frac{2}{x}
\;\;\Rightarrow\;\; g(y)=0
\;\;\Rightarrow\;\; \mu=\mu(x)
\]

\[
\int \frac{d\mu}{\mu} = \int -\frac{2}{x}\,dx
\;\;\Rightarrow\;\;
\ln\mu = -2\ln x + C
\;\;\Rightarrow\;\;
\mu(x)=\frac{C}{x^2}
\]

\[
\mu M\,dx + \mu N\,dy = 0
\;\;\Rightarrow\;\;
\frac{1}{x^2}\big[(x+y^2)\,dx - 2xy\,dy\big]=0
\]

\[
\Big(\tfrac{1}{x}+\tfrac{y^2}{x^2}\Big)\,dx - \tfrac{2y}{x}\,dy=0
\]

\[
\frac{\partial f}{\partial x}=\frac{1}{x}+\frac{y^2}{x^2}
\;\;\Rightarrow\;\;
f=\int \left(\frac{1}{x}+\frac{y^2}{x^2}\right)dx + g(y)
= \frac{y^2}{x}+\ln x+g(y)
\]

\[
\frac{\partial f}{\partial y}=\frac{2y}{x}+g'(y)=\frac{-2y}{x}
\;\;\Rightarrow\;\;
g'(y)=-\frac{2y}{x}+\frac{2y}{x}=0
\;\;\Rightarrow\;\;
g(y)=\ln x+C
\]

\[
f(x,y)=\frac{-y^2}{x}+\ln x+C
\]

\subsection{EDOs lineales de primer orden}

La forma general es:
\[
\frac{dy}{dx}+p(x)y=q(x)
\]

Una EDO es lineal si la combinación lineal de soluciones también es solución.

Si $y_1,y_2$ son soluciones $\;\;\Rightarrow\;\;\alpha y_1+\beta y_2$ también es solución.

\medskip
Hay dos tipos:
\[
\text{Si } q(x)=0 \;\;\Rightarrow\;\; \text{EDO homogénea}
\]
\[
\text{Si } q(x)\neq 0 \;\;\Rightarrow\;\; \text{EDO inhomogénea}
\]
\[
\frac{d}{dx}(\alpha y_1 + \beta y_2) + p(x)(\alpha y_1 + \beta y_2) 
= \alpha \frac{dy_1}{dx} + \beta \frac{dy_2}{dx} + \alpha p(x)y_1 + \beta p(x)y_2
\]

\[
= \alpha\left(\frac{dy_1}{dx}+p(x)y_1\right)+\beta\left(\frac{dy_2}{dx}+p(x)y_2\right)
= (\alpha+\beta)q(x)
\]

\subsection{Homogénea q(x)=0}

\[
\frac{dy}{dx}+p(x)y=0
\;\;\Rightarrow\;\;
\frac{dy}{y}=-p(x)\,dx
\;\;\Rightarrow\;\;
\ln y = -\int p(x)\,dx
\;\;\Rightarrow\;\;
y=Ce^{-\int p(x)\,dx}
\]
\subsection{Inhomogénea $q(x)\neq 0$}
\[
\frac{dy}{dx}+p(x)y=g(x)
\]

\[
y_g(x)=y_h(x)+y_p(x)
\]

\[
\frac{d}{dx}(y_h+y_p)+p(x)(y_h+y_p)=\frac{dy}{dx}+p(x)y=q(x)
\]

Primero resuelvo la homogénea:
\[
y_h = C e^{-\int p(x)\,dx}
\]
\[
y_p = C(x)y_h \;\;\Rightarrow\;\; y_p = C(x)e^{-\int p(x)\,dx}
\]
\[
\frac{dy_p}{dx}+p(x)y_p = C'(x)y_h + C(x)\left(\frac{dy_h}{dx}+p(x)y_h\right)=q(x)
\]

\[
C(x)\left(\frac{dy_h}{dx}+p(x)y_h\right)=0
\]
\[
C'(x)y_h=q(x)
\;\;\Rightarrow\;\;
C(x)=\int \frac{q(x)}{y_h}\,dx
\]
\[
y_p = D_1 y_h + y_h \int \frac{q(x)}{y_h}\,dx
\]

\textbf{Ejemplo:}

\[
y' + 2y = e^{-x}
\]

Primero resolvemos la homogénea:
\[
y' + 2y = 0 \;\;\Rightarrow\;\; \int \frac{dy}{y} = \int -2dx 
\;\;\Rightarrow\;\; \ln y = -2x
\]
\[
y_h = Ce^{-2x}
\]

Ahora resolvemos la particular:
\[
y_p = C(x)e^{-2x} \;\;\Rightarrow\;\; 
y_p' = C'(x)e^{-2x} - 2C(x)e^{-2x}
\]
\[
C'(x)e^{-2x} - 2C(x)e^{-2x} + 2C(x)e^{-2x} = e^{-x} 
\;\;\Rightarrow\;\; C'(x)e^{-2x} = e^{-x}
\]
\[
C'(x) = e^x \;\;\Rightarrow\;\; C(x) = \int e^x dx + D = e^x + D
\]

La solución particular quedaría:
\[
y_p = (e^x + D)e^{-2x} = e^{-x} + De^{-2x}
\]

La solución general quedaría:
\[
y_g = y_h + y_p = Ce^{-2x} + e^{-x} + De^{-2x}
= Ae^{-2x} + e^{-x}
\]

\textbf{Ejemplo:}

\[
\frac{dy}{dx}=\frac{1}{x\cos y+\sen 2y}
\;\;\Rightarrow\;\;
\frac{dx}{dy}=x\cos y+\sen 2y
\]

\[
\frac{dx}{dy}-x\cos y=\sen 2y
\qquad
\text{donde }\;
\begin{cases}
\cos y = p(y)\\[2pt]
\sen 2y = q(y)
\end{cases}
\qquad
\begin{aligned}
&y'+p(x)\,y=g(x)\\
&x'+p(y)\,x=g(y)
\end{aligned}
\]

\[
x_h'-x_h\cos y=0
\;\Rightarrow\;
\int \frac{dx_h}{x_h}=\int \cos y\,dy
\;\Rightarrow\;
\ln x_h=\sen y + D
\]
\[
x_h=e^{\sen y}
\]

\[
x_p=C(y)e^{\sen y}
\;\Rightarrow\;
x_p'=C'(y)e^{\sen y}+C(y)\cos y\,e^{\sen y}
\]
\[
x_p'-x_p\cos y=\sen(2y)
\]

\[
C'(y)e^{\sen y}+C(y)\cos y\,e^{\sen y}-C(y)e^{\sen y}\cos y=\sen 2y
\]
\[
\begin{gathered}
 C'(y)e^{\sen y} = \sen 2y \\[4pt]
 C'(y) = \sen 2y\,e^{-\sen y} \\[4pt]
 C(y) = \int \sen 2y\,e^{-\sen y}\,dy  = 2\!\int \cos y\,\sen y\,e^{-\sen y}\,dy
\end{gathered}
\]
\text{Hacemos cambio de variable } \(z=\sen y\) \(\Rightarrow\) \(dz=\cos y\,dy\):
\[
2\!\int \cos y\,\sen y\,e^{-\sen y}\,dy
=2\!\int z\,e^{-z}\,dz
\]
\end{document}