\documentclass[a4paper,12pt]{article}

% --- Idioma y codificación ---
\usepackage[utf8]{inputenc}
\usepackage[T1]{fontenc}
\usepackage[spanish, es-tabla, shorthands=off]{babel} % <-- clave para que TikZ no falle
\usepackage{lmodern}

% --- Matemáticas ---
\usepackage{amsmath}
\usepackage{amssymb}
\usepackage{amsthm}
\usepackage{mathtools}
\usepackage{bm}
\usepackage{esint}

% --- Utilidades ---
\usepackage{graphicx}
\usepackage{xcolor}
\usepackage{enumitem}
\usepackage{titlesec}
\usepackage[hidelinks]{hyperref}
\usepackage{caption}

% --- Esquemas y Dibujos ---
\usepackage{tikz}
\usetikzlibrary{arrows.meta}

% Formato de subtítulos
\titleformat{\subsection}{\normalfont\small\bfseries}{\thesubsection}{1em}{}

\title{Apuntes de Variable Compleja}
\author{Luis López}
\date{Septiembre 2025}

\begin{document}

\maketitle
\newpage
\tableofcontents
\newpage

\section*{Introducción}

Los \textbf{números complejos}, denotados por $\mathbb{C}$, constituyen una extensión de los números reales $\mathbb{R}$, cumpliéndose que $\mathbb{R} \subset \mathbb{C}$.
A diferencia de los reales, los complejos forman un \textit{cuerpo algebraicamente cerrado}, lo que significa que todo polinomio con coeficientes complejos admite todas sus raíces en $\mathbb{C}$.

Todo número complejo puede escribirse como
\[
z = x + iy,
\]
donde $x, y \in \mathbb{R}$ e $i$ es la unidad imaginaria ($i^2=-1$).
También pueden representarse en \textit{forma polar}, mediante su módulo y argumento.

El conjunto $\mathbb{C}$ no solo es fundamental en álgebra y análisis, sino que resulta indispensable en múltiples áreas de las matemáticas aplicadas y la física.
Asimismo, los números complejos son herramientas habituales en ingeniería.

\newpage
\section{Números complejos}

\subsection{Teoría y estructura elemental}

La imposibilidad de resolver ciertas ecuaciones con números reales nos obliga a introducir los
\textbf{números imaginarios}, definidos a partir de la unidad $i$ tal que
\[
    i^2 = -1.
\]

\subsection{Introducción elemental}

Denotamos los números complejos como
\[
    \mathbb{C} = \{ z = a + bi \;|\; a,b \in \mathbb{R} \}.
\]

Dado $z = a+bi \in \mathbb{C}$, se definen:
\begin{itemize}
    \item Parte real: $\Re(z) = a \in \mathbb{R}$.
    \item Parte imaginaria: $\Im(z) = b \in \mathbb{R}$.
    \item Módulo: $|z| = \sqrt{a^2+b^2}$.
    \item Conjugado: $\overline{z} = a - bi$.
\end{itemize}

\noindent\textbf{Ejemplo.} Sea \( z = 1 - 2i \). Entonces:
\[
\Re(1-2i) = 1, \; \Im(1-2i) = -2, \; \overline{1-2i} = 1 + 2i, \; |1-2i| = \sqrt{1^2 + (-2)^2} = \sqrt{5}.
\]

\subsection{Propiedades elementales}

\begin{enumerate}
    \item $\overline{\overline{z}} = z$. \; Demostración: si $z = a + bi \Rightarrow \overline{z} = a - bi \Rightarrow \overline{\overline{z}} = a + bi = z$.
    \item $z + \overline{z} = 2 \Re(z)$.
    \item $z - \overline{z} = 2i \Im(z)$.
    \item $|\overline{z}| = |z|$.
    \item $\overline{z + z'} = \overline{z} + \overline{z'}$.
    \item $\overline{z \cdot z'} = \overline{z} \cdot \overline{z'}$.
\end{enumerate}

Además, tenemos las siguientes propiedades asociadas al \textbf{módulo}:
\begin{enumerate}
    \item $|\Re(z)| \leq |z|$.
    \item $|\Im(z)| \leq |z|$. \; En efecto, $|z| = \sqrt{a^2 + b^2} \geq |b|$.
    \item \textbf{Desigualdad triangular:} \; $|z+w| \leq |z| + |w| \quad \forall z,w \in \mathbb{C}$.
    \item $z \cdot \overline{z} = |z|^2$.
    \item \textbf{Desigualdad triangular inversa:} \; $\big||z| - |z'|\big| \leq |z-z'|$.
\end{enumerate}

Todas estas propiedades, junto con la suma y producto de números complejos, generalizan las propiedades de los números reales:
\[
z+z' = (x+iy) + (x' + i y') = (x+x') + (y+y')i \in \mathbb{C},
\]
\[
z \cdot z' = (x+iy)(x'+iy') = (xx'-yy') + (xy'+x'y)i \in \mathbb{C}.
\]

En particular, si $z = a + bi$, se cumple que
\[
|z| = \sqrt{a^2 + b^2},
\]
es decir, el módulo de $z$ coincide con el valor absoluto en los reales.

\subsection{Forma polar y geometría de los números complejos}

El conjunto $\mathbb{C}$ se puede representar como $\mathbb{R}^2$ mediante la asignación
\[
z = a + bi \;\longmapsto\; (a, b) \in \mathbb{R}^2.
\]
De esta forma obtenemos el denominado \textbf{plano complejo}; por lo tanto, la interpretación geométrica de \emph{todo lo visto} sería:

\begin{figure}[h]
\centering

% ---- Figura 1: puntos 1 e i ----
\begin{minipage}{0.45\textwidth}
\centering
\begin{tikzpicture}[scale=1]
  % ejes
  \draw[->] (-0.4,0) -- (1.6,0) node[below right] {\(\Re\)};
  \draw[->] (0,-0.4) -- (0,1.6) node[left] {\(\Im\)};
  % puntos (con etiquetas ajustadas)
  \fill (1,0) circle (2pt) node[below] {$(1,0)$};
  \fill (0,1) circle (2pt) node[left] {$(0,1)$};
\end{tikzpicture}

\medskip
{\small Representación de $1$ e $i$}
\end{minipage}
\hfill
% ---- Figura 2: z y conjugado ----
\begin{minipage}{0.45\textwidth}
\centering
\begin{tikzpicture}[scale=1]
  % ejes
  \draw[->] (-0.4,0) -- (3.5,0) node[below right] {\(\Re\)};
  \draw[->] (0,-1.8) -- (0,2.0) node[left] {\(\Im\)};

  % puntos
  \coordinate (O) at (0,0);
  \coordinate (Z)  at (2.2,1.1);   % z
  \coordinate (Zc) at (2.2,-1.1);  % z conjugado

  % vectores OZ y OZc
  \draw[thick,->] (O) -- (Z);
  \draw[thick,->] (O) -- (Zc);

  % proyecciones punteadas
  \draw[dashed] (Z)  -- (2.2,0);
  \draw[dashed] (Z)  -- (0,1.1);
  \draw[dashed] (Zc) -- (2.2,0);

  % puntos con etiquetas
  \fill (Z)  circle (2pt) node[above right] {$z=a+bi$};
  \fill (Zc) circle (2pt) node[below right] {$\overline{z}=a-bi$};

  % arco pequeño para arg(z)
  \draw (0.5,0) arc (0:26:0.5);
\end{tikzpicture}

\medskip
{\small Geometría de $z$ y $\overline{z}$}
\end{minipage}

\end{figure}

\newpage

Si $z = \dfrac{|z|}{|z|}z = |z|\dfrac{z}{|z|} = |z|(\cos\theta + i\sin\theta)$,  
decimos que $z$ está en \textbf{forma polar}.\\

Sea:
\[
w = \frac{z}{|z|} \quad \Rightarrow \quad |w| = \frac{|z|}{|z|} = 1.
\]
Es decir, $\dfrac{z}{|z|}$ es un número complejo de módulo 1, luego existe $\theta \in \mathbb{R}$ tal que
\[
\frac{z}{|z|} = \cos\theta + i\sin\theta.
\]

\noindent\textbf{Ejemplo.} El número complejo $1+i$ en forma polar es:
\[
1+i = \sqrt{2}\left(\cos\frac{\pi}{4} + i\sin\frac{\pi}{4}\right).
\]

\begin{figure}[h]
\centering
\begin{tikzpicture}[scale=2]
  % Ejes
  \draw[->] (-1.2,0) -- (1.2,0) node[below] {\(\Re\)};
  \draw[->] (0,-1.2) -- (0,1.2) node[left] {\(\Im\)};

  % Circunferencia unitaria
  \draw (0,0) circle (1);

  % Ángulo (puedes cambiar este valor)
  \def\ang{45}

  % Punto en la circunferencia: w = (cosθ, sinθ)
  \coordinate (O) at (0,0);
  \coordinate (W) at ({cos(\ang)},{sin(\ang)});

  % Radio OW (longitud 1 en la circunferencia unitaria)
  \draw[thick,->] (O) -- (W);

  % Proyecciones a los ejes
  \draw[dashed] (W) -- ({cos(\ang)},0);
  \draw[dashed] (W) -- (0,{sin(\ang)});

  % Etiquetas de cos y sin
  \node[below] at ({cos(\ang)},0) {$\cos\theta$};
  \node[left]  at (0,{sin(\ang)}) {$\sin\theta$};

  % Arco del ángulo θ
  \draw (0.35,0) arc (0:\ang:0.35);
  \node at ({0.48*cos(\ang/2)},{0.48*sin(\ang/2)}) {$\theta$};
\end{tikzpicture}

\medskip
{\small Circunferencia unitaria: \(w=\cos\theta+i\sin\theta\)}
\end{figure}

Vemos que un mismo número complejo tiene infinitas representaciones polares por culpa del ángulo $\theta$, llamado \textbf{argumento de $z$}.  
Para solucionar este problema introducimos el \textbf{argumento principal} de $z$, que es aquel ángulo $-\pi < \theta \leq \pi$ que verifica:
\[
z = |z|(\cos\theta + i\sin\theta), \qquad \theta = \arg(z).
\]

Además, si $z = x+iy$, entonces:
\[
\arg(z) =
\begin{cases}
\arctan\!\left(\dfrac{y}{x}\right), & x > 0, \, y \geq 0, \\[1ex]
\arctan\!\left(\dfrac{y}{x}\right)+\pi, & x < 0, \, y \geq 0, \\[1ex]
\pi, & x < 0, \, y = 0, \\[1ex]
\arctan\!\left(\dfrac{y}{x}\right)-\pi, & x < 0, \, y < 0, \\[1ex]
-\dfrac{\pi}{2}, & x = 0, \, y < 0, \\[1ex]
\dfrac{\pi}{2}, & x = 0, \, y > 0, \\[1ex]
\arctan\!\left(\dfrac{y}{x}\right), & x > 0, \, y < 0.
\end{cases}
\]

\medskip
\noindent Nótese que se verifica que
\[
\arg(z) = \arg(z) + 2k\pi, \qquad k \in \mathbb{Z}.
\]

Así, geométricamente, un número complejo $z$ tendría esta información:
\[
z = |z| e^{i\arg(z)}.
\]

\begin{figure}[h]
\centering
\begin{tikzpicture}[scale=1.2]

  % ======= parámetros editables =======
  \def\angZ{30}    % ángulo de z   (grados)
  \def\rZ{2.2}     % módulo de z   (longitud del vector)

  % ======= ejes =======
  \draw[->] (-0.6,0) -- (3.2,0) node[below right] {\(\Re\)};
  \draw[->] (0,-0.6) -- (0,2.6) node[left] {\(\Im\)};

  % ======= origen y punto z =======
  \coordinate (O) at (0,0);
  \coordinate (Z) at ({\rZ*cos(\angZ)},{\rZ*sin(\angZ)});

  % ======= vector OZ =======
  \draw[thick,->] (O) -- (Z) node[pos=0.9, above left] {$z$};

  % ======= arco del argumento =======
  \draw (0.55,0) arc (0:\angZ:0.55);
  \node at (0.7,-0.25) {\(\arg(z)\)}; % debajo del eje real

  % ======= línea guía punteada (opcional) =======
  \draw[dashed] (Z) -- ({\rZ*cos(\angZ)},0);

  % punto en la punta
  \fill (Z) circle (2pt);

\end{tikzpicture}

\medskip
{\small Argumento de \(z\).}
\end{figure}

Al tener el módulo de cualquier complejo podemos hablar de la noción de distancia entre complejos,
dada por:
\[
d(z,w) = |z-w|.
\]

\noindent\textbf{Definición.} El disco centrado en $z_0 \in \mathbb{C}$ y de radio $\varepsilon > 0$ es:
\[
D(z_0,\varepsilon) = \{ z \in \mathbb{C} : |z - z_0| < \varepsilon \}.
\]

\noindent\textbf{Ejemplo.} 
\[
D(1+i,1) = \{ z \in \mathbb{C} : |z-(1+i)| < 1 \}.
\]

El disco cerrado se denota por:
\[
\overline{D}(z_0,\varepsilon) = \{ z \in \mathbb{C} : |z - z_0| \leq \varepsilon \}.
\]

\begin{figure}[h]
\centering

% ---- Disco abierto D(1+i,1) ----
\begin{minipage}{0.4\textwidth}
\centering
\begin{tikzpicture}[scale=1.1]
  % ejes
  \draw[->] (-0.2,0) -- (3.0,0) node[below right] {\(\Re\)};
  \draw[->] (0,-0.2) -- (0,3.0) node[left] {\(\Im\)};

  % centro y radio
  \coordinate (C) at (1,1);
  \def\R{1}

  % disco ABIERTO: círculo discontinuo, sin relleno
  \draw[dashed, very thick] (C) circle (\R);

  % centro (solo punto, sin coordenada)
  \fill (C) circle (2pt);
\end{tikzpicture}

\medskip
{\small Disco abierto \(D(1+i,1)\)}
\end{minipage}
\hfill
% ---- Flecha entre figuras ----
\begin{minipage}{0.1\textwidth}
\centering
\begin{tikzpicture}
  \draw[->, thick] (0,0) -- (1.6,0);
\end{tikzpicture}
\end{minipage}
\hfill
% ---- Disco cerrado \overline{D}(1+i,1) ----
\begin{minipage}{0.4\textwidth}
\centering
\begin{tikzpicture}[scale=1.1]
  % ejes
  \draw[->] (-0.2,0) -- (3.0,0) node[below right] {\(\Re\)};
  \draw[->] (0,-0.2) -- (0,3.0) node[left] {\(\Im\)};

  % centro y radio
  \coordinate (C) at (1,1);
  \def\R{1}

  % disco CERRADO: relleno suave + borde continuo
  \filldraw[fill=orange!25, draw=black, very thick] (C) circle (\R);

  % centro (solo punto, sin coordenada)
  \fill (C) circle (2pt);
\end{tikzpicture}

\medskip
{\small Disco cerrado \(\overline{D}(1+i,1)\)}
\end{minipage}

\end{figure}

\end{document}