\documentclass[a4paper,12pt]{article}

% --- Idioma y codificación ---
\usepackage[utf8]{inputenc}   % Acentos directos
\usepackage[T1]{fontenc}      % Codificación de salida
\usepackage[spanish]{babel}   % Español
\usepackage{lmodern}          % Fuente moderna

% --- Matemáticas ---
\usepackage{amsmath}          % Entornos matemáticos
\usepackage{amssymb}          % Símbolos
\usepackage{amsthm}           % Teoremas
\usepackage{mathtools}        % Extras de amsmath
\usepackage{bm}               % Negrita en símbolos
\usepackage{esint}            % Integrales de contorno (∮)

% --- Utilidades ---
\usepackage{graphicx}         % Imágenes
\usepackage{xcolor}           % Colores
\usepackage{enumitem}         % Listas
\usepackage{titlesec}         % Títulos
\usepackage[hidelinks]{hyperref} % Enlaces (cargar al final)

\titleformat{\subsection}
  {\normalfont\small\bfseries}   % formato más pequeño y en negrita
  {\thesubsection}{1em}{} 

\title{Apuntes de Variable Compleja}
\author{Luis López}
\date{Septiembre 2025}

\begin{document}

\maketitle
\newpage
\tableofcontents
\newpage

\section*{Introducción}

Los \textbf{números complejos}, denotados por $\mathbb{C}$, constituyen una extensión de los números reales $\mathbb{R}$, cumpliéndose que $\mathbb{R} \subset \mathbb{C}$. 
A diferencia de los reales, los complejos forman un \textit{cuerpo algebraicamente cerrado}, lo que significa que todo polinomio con coeficientes complejos admite todas sus raíces en $\mathbb{C}$.

Todo número complejo puede escribirse como 
\[
z = x + iy,
\]
donde $x, y \in \mathbb{R}$ e $i$ es la unidad imaginaria ($i^2=-1$). 
También pueden representarse en \textit{forma polar}, mediante su módulo y argumento.

El conjunto $\mathbb{C}$ no solo es fundamental en álgebra y análisis, sino que resulta indispensable en múltiples áreas de las matemáticas aplicadas y la física. 
Asimismo, los números complejos son herramientas habituales en ingeniería.

\newpage
\section{Números complejos}

\subsection{Teoría y estructura elemental}

La imposibilidad de resolver ciertas ecuaciones con números reales nos obliga a introducir los 
\textbf{números imaginarios}, definidos a partir de la unidad $i$ tal que
\[
    i^2 = -1.
\]

\subsection{Definiciones básicas}

Denotamos los números complejos como
\[
    \mathbb{C} = \{ z = a + bi \;|\; a,b \in \mathbb{R} \}.
\]

Dado $z = a+bi \in \mathbb{C}$, se definen:
\begin{itemize}
    \item Parte real: $\Re(z) = a \in \mathbb{R}$.
    \item Parte imaginaria: $\Im(z) = b \in \mathbb{R}$.
    \item Módulo: $|z| = \sqrt{a^2+b^2}$.
    \item Conjugado: $\overline{z} = a - bi$.
\end{itemize}

\subsection{Propiedades elementales}

Para $z=a+bi \in \mathbb{C}$:
\begin{align*}
    z + \overline{z} &= 2\Re(z), \\
    z - \overline{z} &= 2i \Im(z), \\
    z \cdot \overline{z} &= |z|^2.
\end{align*}

\subsection{Forma polar}

Si $z \neq 0$:
\[
    z = |z|(\cos \theta + i\sin \theta) = |z|e^{i\theta},
\]
donde $\theta = \arg(z)$.

\end{document}