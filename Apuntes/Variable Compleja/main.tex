\documentclass[a4paper,12pt]{article}

% --- Idioma y codificación ---
\usepackage[utf8]{inputenc}
\usepackage[T1]{fontenc}
\usepackage[spanish, es-tabla, shorthands=off]{babel} % <-- clave para que TikZ no falle
\usepackage{lmodern}

% --- Matemáticas ---
\usepackage{amsmath}
\usepackage{amssymb}
\usepackage{amsthm}
\usepackage{mathtools}
\usepackage{bm}
\usepackage{esint}

% --- Utilidades ---
\usepackage{graphicx}
\usepackage{xcolor}
\usepackage{enumitem}
\usepackage{titlesec}
\usepackage[hidelinks]{hyperref}
\usepackage{caption}

% --- Esquemas y Dibujos ---
\usepackage{tikz}
\usetikzlibrary{arrows.meta}

% Formato de subtítulos
\titleformat{\subsection}{\normalfont\small\bfseries}{\thesubsection}{1em}{}

\title{Apuntes de Variable Compleja}
\author{Luis López}
\date{Septiembre 2025}

\begin{document}

\maketitle
\newpage
\tableofcontents
\newpage

\section*{Introducción}

Los \textbf{números complejos}, denotados por $\mathbb{C}$, constituyen una extensión de los números reales $\mathbb{R}$, cumpliéndose que $\mathbb{R} \subset \mathbb{C}$.
A diferencia de los reales, los complejos forman un \textit{cuerpo algebraicamente cerrado}, lo que significa que todo polinomio con coeficientes complejos admite todas sus raíces en $\mathbb{C}$.

Todo número complejo puede escribirse como
\[
z = x + iy,
\]
donde $x, y \in \mathbb{R}$ e $i$ es la unidad imaginaria ($i^2=-1$).
También pueden representarse en \textit{forma polar}, mediante su módulo y argumento.

El conjunto $\mathbb{C}$ no solo es fundamental en álgebra y análisis, sino que resulta indispensable en múltiples áreas de las matemáticas aplicadas y la física.
Asimismo, los números complejos son herramientas habituales en ingeniería.

\newpage
\section{Números complejos}

\subsection{Teoría y estructura elemental}

La imposibilidad de resolver ciertas ecuaciones con números reales nos obliga a introducir los
\textbf{números imaginarios}, definidos a partir de la unidad $i$ tal que
\[
    i^2 = -1.
\]

\subsection{Introducción elemental}

Denotamos los números complejos como
\[
    \mathbb{C} = \{ z = a + bi \;|\; a,b \in \mathbb{R} \}.
\]

Dado $z = a+bi \in \mathbb{C}$, se definen:
\begin{itemize}
    \item Parte real: $\Re(z) = a \in \mathbb{R}$.
    \item Parte imaginaria: $\Im(z) = b \in \mathbb{R}$.
    \item Módulo: $|z| = \sqrt{a^2+b^2}$.
    \item Conjugado: $\overline{z} = a - bi$.
\end{itemize}

\noindent\textbf{Ejemplo.} Sea \( z = 1 - 2i \). Entonces:
\[
\Re(1-2i) = 1, \; \Im(1-2i) = -2, \; \overline{1-2i} = 1 + 2i, \; |1-2i| = \sqrt{1^2 + (-2)^2} = \sqrt{5}.
\]

\subsection{Propiedades elementales}

\begin{enumerate}
    \item $\overline{\overline{z}} = z$. \; Demostración: si $z = a + bi \Rightarrow \overline{z} = a - bi \Rightarrow \overline{\overline{z}} = a + bi = z$.
    \item $z + \overline{z} = 2 \Re(z)$.
    \item $z - \overline{z} = 2i \Im(z)$.
    \item $|\overline{z}| = |z|$.
    \item $\overline{z + z'} = \overline{z} + \overline{z'}$.
    \item $\overline{z \cdot z'} = \overline{z} \cdot \overline{z'}$.
\end{enumerate}

Además, tenemos las siguientes propiedades asociadas al \textbf{módulo}:
\begin{enumerate}
    \item $|\Re(z)| \leq |z|$.
    \item $|\Im(z)| \leq |z|$. \; En efecto, $|z| = \sqrt{a^2 + b^2} \geq |b|$.
    \item \textbf{Desigualdad triangular:} \; $|z+w| \leq |z| + |w| \quad \forall z,w \in \mathbb{C}$.
    \item $z \cdot \overline{z} = |z|^2$.
    \item \textbf{Desigualdad triangular inversa:} \; $\big||z| - |z'|\big| \leq |z-z'|$.
\end{enumerate}

Todas estas propiedades, junto con la suma y producto de números complejos, generalizan las propiedades de los números reales:
\[
z+z' = (x+iy) + (x' + i y') = (x+x') + (y+y')i \in \mathbb{C},
\]
\[
z \cdot z' = (x+iy)(x'+iy') = (xx'-yy') + (xy'+x'y)i \in \mathbb{C}.
\]

En particular, si $z = a + bi$, se cumple que
\[
|z| = \sqrt{a^2 + b^2},
\]
es decir, el módulo de $z$ coincide con el valor absoluto en los reales.

\subsection{Forma polar y geometría de los números complejos}

El conjunto $\mathbb{C}$ se puede representar como $\mathbb{R}^2$ mediante la asignación
\[
z = a + bi \;\longmapsto\; (a, b) \in \mathbb{R}^2.
\]
De esta forma obtenemos el denominado \textbf{plano complejo}; por lo tanto, la interpretación geométrica de \emph{todo lo visto} sería:

\begin{figure}[h]
\centering

% ---- Figura 1: puntos 1 e i ----
\begin{minipage}{0.45\textwidth}
\centering
\begin{tikzpicture}[scale=1]
  % ejes
  \draw[->] (-0.4,0) -- (1.6,0) node[below right] {\(\Re\)};
  \draw[->] (0,-0.4) -- (0,1.6) node[left] {\(\Im\)};
  % puntos (con etiquetas ajustadas)
  \fill (1,0) circle (2pt) node[below] {$(1,0)$};
  \fill (0,1) circle (2pt) node[left] {$(0,1)$};
\end{tikzpicture}

\medskip
{\small Representación de $1$ e $i$}
\end{minipage}
\hfill
% ---- Figura 2: z y conjugado ----
\begin{minipage}{0.45\textwidth}
\centering
\begin{tikzpicture}[scale=1]
  % ejes
  \draw[->] (-0.4,0) -- (3.5,0) node[below right] {\(\Re\)};
  \draw[->] (0,-1.8) -- (0,2.0) node[left] {\(\Im\)};

  % puntos
  \coordinate (O) at (0,0);
  \coordinate (Z)  at (2.2,1.1);   % z
  \coordinate (Zc) at (2.2,-1.1);  % z conjugado

  % vectores OZ y OZc
  \draw[thick,->] (O) -- (Z);
  \draw[thick,->] (O) -- (Zc);

  % proyecciones punteadas
  \draw[dashed] (Z)  -- (2.2,0);
  \draw[dashed] (Z)  -- (0,1.1);
  \draw[dashed] (Zc) -- (2.2,0);

  % puntos con etiquetas
  \fill (Z)  circle (2pt) node[above right] {$z=a+bi$};
  \fill (Zc) circle (2pt) node[below right] {$\overline{z}=a-bi$};

  % arco pequeño para arg(z)
  \draw (0.5,0) arc (0:26:0.5);
\end{tikzpicture}

\medskip
{\small Geometría de $z$ y $\overline{z}$}
\end{minipage}

\end{figure}

\newpage

Si $z = \dfrac{|z|}{|z|}z = |z|\dfrac{z}{|z|} = |z|(\cos\theta + i\sin\theta)$,  
decimos que $z$ está en \textbf{forma polar}.\\

Sea:
\[
w = \frac{z}{|z|} \quad \Rightarrow \quad |w| = \frac{|z|}{|z|} = 1.
\]
Es decir, $\dfrac{z}{|z|}$ es un número complejo de módulo 1, luego existe $\theta \in \mathbb{R}$ tal que
\[
\frac{z}{|z|} = \cos\theta + i\sin\theta.
\]

\noindent\textbf{Ejemplo.} El número complejo $1+i$ en forma polar es:
\[
1+i = \sqrt{2}\left(\cos\frac{\pi}{4} + i\sin\frac{\pi}{4}\right).
\]

\begin{figure}[h]
\centering
\begin{tikzpicture}[scale=2]
  % Ejes
  \draw[->] (-1.2,0) -- (1.2,0) node[below] {\(\Re\)};
  \draw[->] (0,-1.2) -- (0,1.2) node[left] {\(\Im\)};

  % Circunferencia unitaria
  \draw (0,0) circle (1);

  % Ángulo (puedes cambiar este valor)
  \def\ang{45}

  % Punto en la circunferencia: w = (cosθ, sinθ)
  \coordinate (O) at (0,0);
  \coordinate (W) at ({cos(\ang)},{sin(\ang)});

  % Radio OW (longitud 1 en la circunferencia unitaria)
  \draw[thick,->] (O) -- (W);

  % Proyecciones a los ejes
  \draw[dashed] (W) -- ({cos(\ang)},0);
  \draw[dashed] (W) -- (0,{sin(\ang)});

  % Etiquetas de cos y sin
  \node[below] at ({cos(\ang)},0) {$\cos\theta$};
  \node[left]  at (0,{sin(\ang)}) {$\sin\theta$};

  % Arco del ángulo θ
  \draw (0.35,0) arc (0:\ang:0.35);
  \node at ({0.48*cos(\ang/2)},{0.48*sin(\ang/2)}) {$\theta$};
\end{tikzpicture}

\medskip
{\small Circunferencia unitaria: \(w=\cos\theta+i\sin\theta\)}
\end{figure}

Vemos que un mismo número complejo tiene infinitas representaciones polares por culpa del ángulo $\theta$, llamado \textbf{argumento de $z$}.  
Para solucionar este problema introducimos el \textbf{argumento principal} de $z$, que es aquel ángulo $-\pi < \theta \leq \pi$ que verifica:
\[
z = |z|(\cos\theta + i\sin\theta), \qquad \theta = \arg(z).
\]

Además, si $z = x+iy$, entonces:
\[
\arg(z) =
\begin{cases}
\arctan\!\left(\dfrac{y}{x}\right), & x > 0, \, y \geq 0, \\[1ex]
\arctan\!\left(\dfrac{y}{x}\right)+\pi, & x < 0, \, y \geq 0, \\[1ex]
\pi, & x < 0, \, y = 0, \\[1ex]
\arctan\!\left(\dfrac{y}{x}\right)-\pi, & x < 0, \, y < 0, \\[1ex]
-\dfrac{\pi}{2}, & x = 0, \, y < 0, \\[1ex]
\dfrac{\pi}{2}, & x = 0, \, y > 0, \\[1ex]
\arctan\!\left(\dfrac{y}{x}\right), & x > 0, \, y < 0.
\end{cases}
\]

\medskip
\noindent Nótese que se verifica que
\[
\arg(z) = \arg(z) + 2k\pi, \qquad k \in \mathbb{Z}.
\]

Así, geométricamente, un número complejo $z$ tendría esta información:
\[
z = |z| e^{i\arg(z)}.
\]

\begin{figure}[h]
\centering
\begin{tikzpicture}[scale=1.2]

  % ======= parámetros editables =======
  \def\angZ{30}    % ángulo de z   (grados)
  \def\rZ{2.2}     % módulo de z   (longitud del vector)

  % ======= ejes =======
  \draw[->] (-0.6,0) -- (3.2,0) node[below right] {\(\Re\)};
  \draw[->] (0,-0.6) -- (0,2.6) node[left] {\(\Im\)};

  % ======= origen y punto z =======
  \coordinate (O) at (0,0);
  \coordinate (Z) at ({\rZ*cos(\angZ)},{\rZ*sin(\angZ)});

  % ======= vector OZ =======
  \draw[thick,->] (O) -- (Z) node[pos=0.9, above left] {$z$};

  % ======= arco del argumento =======
  \draw (0.55,0) arc (0:\angZ:0.55);
  \node at (0.7,-0.25) {\(\arg(z)\)}; % debajo del eje real

  % ======= línea guía punteada (opcional) =======
  \draw[dashed] (Z) -- ({\rZ*cos(\angZ)},0);

  % punto en la punta
  \fill (Z) circle (2pt);

\end{tikzpicture}

\medskip
{\small Argumento de \(z\).}
\end{figure}

Al tener el módulo de cualquier complejo podemos hablar de la noción de distancia entre complejos,
dada por:
\[
d(z,w) = |z-w|.
\]

\noindent\textbf{Definición.} El disco centrado en $z_0 \in \mathbb{C}$ y de radio $\varepsilon > 0$ es:
\[
D(z_0,\varepsilon) = \{ z \in \mathbb{C} : |z - z_0| < \varepsilon \}.
\]

\noindent\textbf{Ejemplo.} 
\[
D(1+i,1) = \{ z \in \mathbb{C} : |z-(1+i)| < 1 \}.
\]

El disco cerrado se denota por:
\[
\overline{D}(z_0,\varepsilon) = \{ z \in \mathbb{C} : |z - z_0| \leq \varepsilon \}.
\]

\begin{figure}[h]
\centering

% ---- Disco abierto D(1+i,1) ----
\begin{minipage}{0.4\textwidth}
\centering
\begin{tikzpicture}[scale=1.1]
  % ejes
  \draw[->] (-0.2,0) -- (3.0,0) node[below right] {\(\Re\)};
  \draw[->] (0,-0.2) -- (0,3.0) node[left] {\(\Im\)};

  % centro y radio
  \coordinate (C) at (1,1);
  \def\R{1}

  % disco ABIERTO: círculo discontinuo, sin relleno
  \draw[dashed, very thick] (C) circle (\R);

  % centro (solo punto, sin coordenada)
  \fill (C) circle (2pt);
\end{tikzpicture}

\medskip
{\small Disco abierto \(D(1+i,1)\)}
\end{minipage}
\hfill
% ---- Flecha entre figuras ----
\begin{minipage}{0.1\textwidth}
\centering
\begin{tikzpicture}
  \draw[->, thick] (0,0) -- (1.6,0);
\end{tikzpicture}
\end{minipage}
\hfill
% ---- Disco cerrado \overline{D}(1+i,1) ----
\begin{minipage}{0.4\textwidth}
\centering
\begin{tikzpicture}[scale=1.1]
  % ejes
  \draw[->] (-0.2,0) -- (3.0,0) node[below right] {\(\Re\)};
  \draw[->] (0,-0.2) -- (0,3.0) node[left] {\(\Im\)};

  % centro y radio
  \coordinate (C) at (1,1);
  \def\R{1}

  % disco CERRADO: relleno suave + borde continuo
  \filldraw[fill=orange!25, draw=black, very thick] (C) circle (\R);

  % centro (solo punto, sin coordenada)
  \fill (C) circle (2pt);
\end{tikzpicture}

\medskip
{\small Disco cerrado \(\overline{D}(1+i,1)\)}
\end{minipage}

\end{figure}

\subsection{Forma exponencial y multiplicación de números complejos}

Por conveniencia, escribimos
\[
\cos\theta+i\sin\theta \;=\; e^{i\theta}.
\]
Así, la forma polar \(z=|z|(\cos\theta+i\sin\theta)\) se expresa como
\[
z=|z|\,e^{i\theta}.
\]

Esta representación hace muy sencilla la multiplicación de complejos. En efecto, si:
\[
z = r\,e^{i\theta}, \qquad w=\rho\,e^{i\varphi} \quad (r,\rho\ge 0),
\]
entonces:
\[
z\,w = (r\rho)\,e^{i(\theta+\varphi)}.
\]

\begin{figure}[h]
\centering

% --- Panel 1: vectores z y w (sin argumentos) ---
\begin{minipage}{0.42\textwidth}
\centering
\begin{tikzpicture}[scale=1.2]
  \def\angZ{25} \def\angW{50}
  \def\rZ{1.6}  \def\rW{1.2}

  \draw[->] (-0.6,0) -- (3.0,0) node[below right] {\(\Re\)};
  \draw[->] (0,-0.6) -- (0,2.4) node[left] {\(\Im\)};

  \coordinate (O) at (0,0);
  \coordinate (Z) at ({\rZ*cos(\angZ)},{\rZ*sin(\angZ)});
  \coordinate (W) at ({\rW*cos(\angW)},{\rW*sin(\angW)});

  \draw[thick,->] (O) -- (Z) node[pos=0.92, above left] {$z$};
  \draw[thick,->] (O) -- (W) node[pos=0.92, above right] {$w$};

  % Guías (opcional): comenta si no las quieres
  \draw[dashed] (Z) -- ({\rZ*cos(\angZ)},0);
  \draw[dashed] (W) -- ({\rW*cos(\angW)},0);

  \fill (Z) circle (2pt);
  \fill (W) circle (2pt);
\end{tikzpicture}
\end{minipage}
\hfill
% --- Flecha entre paneles ---
\begin{minipage}{0.08\textwidth}
\centering
\begin{tikzpicture}
  \draw[->, thick] (0,0) -- (1.6,0);
\end{tikzpicture}
\end{minipage}
\hfill
% --- Panel 2: producto zw (sin argumentos) ---
\begin{minipage}{0.42\textwidth}
\centering
\begin{tikzpicture}[scale=1.2]
  \def\angZ{25} \def\angW{50}
  \def\rZ{1.6}  \def\rW{1.2}
  \pgfmathsetmacro{\angP}{\angZ+\angW}
  \pgfmathsetmacro{\rP}{\rZ*\rW}

  \draw[->] (-0.6,0) -- (3.2,0) node[below right] {\(\Re\)};
  \draw[->] (0,-0.6) -- (0,2.6) node[left] {\(\Im\)};

  \coordinate (O) at (0,0);
  \coordinate (P) at ({\rP*cos(\angP)},{\rP*sin(\angP)});

  \draw[thick,->] (O) -- (P) node[pos=0.92, above right] {$zw$};

  % Guía (opcional)
  \draw[dashed] (P) -- ({\rP*cos(\angP)},0);

  \fill (P) circle (2pt);
\end{tikzpicture}
\end{minipage}

\medskip
{\small Multiplicación en forma polar (solo vectores).}
\end{figure}

En particular,
\[
|zw|=|z|\,|w|, \qquad \arg(zw)=\arg(z)+\arg(w)\pmod{2\pi}.
\]
\textbf{Nota.} En general es \emph{falso} que
\[
\arg(zw)=\arg(z)+\arg(w),
\]
pues \(\arg\) es el argumento principal (restringido a \((-\pi,\pi]\)) y puede requerir ajustar por múltiplos de \(2\pi\).

\paragraph{Ejemplo.}
\[
\arg(i)=\frac{\pi}{2}, \qquad \arg(-i)=-\frac{\pi}{2},
\]
mientras que
\[
\arg(i)=\frac{\pi}{2}+2k\pi, \qquad \arg(-i)=-\frac{\pi}{2}+2k\pi, \quad k\in\mathbb{Z}.
\]

\paragraph{Potencias (fórmula de De Moivre).}
Si \(z=r\,e^{i\theta}\) y \(n\in\mathbb{N}\), entonces:
\[
z^n=r^n\,e^{in\theta}.
\]

\subsection{Ejemplo: cálculo en forma exponencial}
\noindent\textbf{Ejemplo.} Calcular \((1+i)^4\).
\[
1+i=\sqrt{2}\,e^{i\pi/4}\;\;\Rightarrow\;\;(1+i)^4=(\sqrt{2})^4e^{i\pi}=4\cdot(-1)=-4,
\]
llegamos a:
\[
e^{i\pi}+1=0
\]

\medskip
{\small (Fórmula de Euler, considerada la más bonita de las matemáticas).}

\subsection{Raíces de números complejos}\label{subsec:raices}
Queremos estudiar qué números complejos cumplen la ecuación \(z^n=w\) para un \(n\ge 2\), \(w\in\mathbb{C}\).

\paragraph{Proposición.}
Dado un número complejo no nulo \(w\in\mathbb{C}\setminus\{0\}\) y \(n\ge 2\), existen exactamente \(n\) números complejos que cumplen \(z^n=w\). Si
\[
w=r\,e^{i\theta}\quad (r>0,\;\theta\in\mathbb{R}),
\]
entonces las soluciones son
\[
z_k=r^{1/n}e^{\,i(\theta+2k\pi)/n},\qquad k=0,1,\dots,n-1,
\]
que son los vértices de un \(n\)-gono regular centrado en el origen.

\paragraph{Demostración.}
Si \(z=\rho e^{i\varphi}\), entonces \(z^n=\rho^n e^{in\varphi}\).
Imponiendo \(z^n=w=r e^{i\theta}\) se obtiene \(\rho^n=r\Rightarrow \rho=r^{1/n}\) y
\(n\varphi=\theta+2k\pi\Rightarrow \varphi=(\theta+2k\pi)/n\).

\paragraph{Definición.}
El \emph{conjunto de las raíces \(n\)-ésimas de \(w\)} es
\[
\sqrt[n]{w}:=\bigl\{\,r^{1/n}e^{\,i(\theta+2k\pi)/n}\;:\;k=0,\dots,n-1\,\bigr\}.
\]
Por abuso, se llama \emph{raíz \(n\)-ésima principal} de \(w\) a
\[
\sqrt[n]{w}_{\mathrm{pr}}:=r^{1/n}e^{\,i\,\arg(w)/n},
\]
donde \(\arg(w)\in(-\pi,\pi]\) es el argumento principal.

\paragraph{Ejemplo.} Resolver \(z^3=-8i\).
\[
-8i=8\,e^{-i\pi/2}\quad\Rightarrow\quad
z_k=2\,e^{\,i(-\pi/2+2k\pi)/3},\;k=0,1,2.
\]
Explícitamente:
\[
z_0=2e^{-i\pi/6}=\sqrt{3}-i,\quad
z_1=2e^{\,i\pi/2}=2i,\quad
z_2=2e^{\,i7\pi/6}=-\sqrt{3}-i.
\]
La raíz cúbica \emph{principal} es \(2e^{-i\pi/6}\).

\paragraph{Notas.}
\begin{itemize}[leftmargin=1.2em]
  \item \((z_k)^n=w\) para todo \(k\), y \(z_k=z_0\,e^{\,i2k\pi/n}\).
  \item \(\sqrt[n]{w}\) denota un \emph{conjunto}; la notación de raíz \emph{principal} usa \(\arg(w)\).
\end{itemize}

\newpage

\section{Teoría de las funciones de variable compleja}
Estudiaremos funciones de la forma \(f:A\subset\mathbb{C}\to\mathbb{C}\).

\paragraph{Parte real e imaginaria.}
Si \(z=x+iy\), toda función \(f\) puede escribirse como
\[
f(z)=u(x,y)+i\,v(x,y),
\]
donde \(u=\Re f\) y \(v=\Im f\).

\paragraph{Ejemplos.}
\begin{align*}
f(z)=z^2=(x+iy)^2 &= (x^2-y^2)+i(2xy)
&&\Rightarrow&& u(x,y)=x^2-y^2,\;\; v(x,y)=2xy,\\
f(z)=\overline{z}=x-iy &&&\Rightarrow&& u(x,y)=x,\;\; v(x,y)=-y.
\end{align*}
\subsection{Límites y continuidad de funciones complejas}

\textbf{Definición.} Sea $f$ definida en un conjunto $A\subset\mathbb{C}$, $f:A\subset\mathbb{C}\to\mathbb{C}$. Diremos que el límite de $f$ cuando $z$ tiende a $z_0$ es $w\in\mathbb{C}$ si
\[
\forall\,\varepsilon>0\ \exists\,\delta>0\ \text{tal que}\ \ 0<|z-z_0|<\delta \ \Rightarrow\ |f(z)-w|<\varepsilon.
\]
Se denota por \(\displaystyle \lim_{z\to z_0} f(z)=w\).

\medskip

\textbf{Proposición.} Sean $f,g$ dos funciones tales que $f,g:A\subset\mathbb{C}\to\mathbb{C}$ y existen
\(\displaystyle \lim_{z\to z_0} f(z)=w\) y \(\displaystyle \lim_{z\to z_0} g(z)=\ell\). Entonces:
\begin{enumerate}
  \item \(\displaystyle \lim_{z\to z_0} \big(f(z)+g(z)\big)
  = \lim_{z\to z_0} f(z) + \lim_{z\to z_0} g(z) = w+\ell.\)
  \item \(\displaystyle \lim_{z\to z_0} \big(f(z)\,g(z)\big)
  = \lim_{z\to z_0} f(z)\ \cdot\ \lim_{z\to z_0} g(z) = w\,\ell.\)
  \item Si \(\ell\neq 0\), entonces \(\displaystyle
  \lim_{z\to z_0} \frac{f(z)}{g(z)} = \frac{w}{\ell}.\)
\end{enumerate}
\newpage
\paragraph{Proposición.}
\[
\exists \lim_{z\to z_0} f(z)=w \quad\Longleftrightarrow\quad
\exists \lim_{(x,y)\to(x_0,y_0)} \bigl(u(x,y)+i\,v(x,y)\bigr)=a+ib
\]
equivalentemente
\[
\begin{cases}
\displaystyle \exists \lim_{(x,y)\to(x_0,y_0)} u(x,y)=a,\\[4pt]
\displaystyle \exists \lim_{(x,y)\to(x_0,y_0)} v(x,y)=b.
\end{cases}
\]

\paragraph{Demostración.}
\[
\exists \lim_{z\to z_0} f(z)=a+bi=w
\ \Longleftrightarrow\
\forall \varepsilon>0\ \exists \delta>0:\ 0<|z-z_0|<\delta \ \Rightarrow\ |f(z)-(a+bi)|<\varepsilon .
\]
Analicemos lo siguiente:
\[
\lim_{(x,y)\to(x_0,y_0)} u(x,y)=a
\ \Longleftrightarrow\
\forall \varepsilon>0\ \exists \delta>0:\ 0<\| (x,y)-(x_0,y_0)\|<\delta \Rightarrow |u(x,y)-a|<\varepsilon .
\]
En efecto,
\[
|u(x,y)-a|
=\bigl|\,u(x,y)-a+i(v(x,y)-b)\,-\,i(v(x,y)-b)\,\bigr|
\le |f(z)-(a+bi)|
<\varepsilon,
\]
y análogamente para la parte imaginaria \(v(x,y)\).
Además,
\[
0<|z-z_0|<\delta
\ \Longleftrightarrow\
0<|x+iy-(x_0+iy_0)|<\delta
\ \Longleftrightarrow\
0<\sqrt{(x-x_0)^2+(y-y_0)^2}<\delta .
\]
La parte imaginaria es análoga. \(\square\)

\paragraph{Ejemplo.} Analicemos \(\displaystyle \lim_{z\to 2i} z^2\).
Como
\[
(x+iy)^2=(x^2-y^2)+i(2xy),
\]
se tiene
\[
\lim_{(x,y)\to(0,2)}(x^2-y^2)=-4,
\qquad
\lim_{(x,y)\to(0,2)}2xy=0,
\]
luego \(\displaystyle \lim_{z\to 2i} z^2=-4\).

\paragraph{Ejemplo (analítico por definición).}
Probemos por \(\varepsilon\)-\(\delta\) que \(\displaystyle \lim_{z\to 2i} z^2=-4\).
Sea \(\varepsilon>0\). Observamos
\[
|z^2-(2i)^2|=|z-2i|\,|z+2i|
\le |z-2i|\bigl(|z-2i|+|4i|\bigr)
=|z-2i|\,(|z-2i|+4).
\]
Si imponemos \(0<|z-2i|<\delta\) y además \(\delta\le 1\), entonces
\[
|z^2-(2i)^2|
< \delta(\delta+4)\le 5\delta.
\]
Eligiendo
\[
\delta=\min\!\left\{1,\ \frac{\varepsilon}{5}\right\}
\]
se obtiene \(0<|z-2i|<\delta \Rightarrow |z^2+4|<\varepsilon\). Por lo tanto,
\[
\lim_{z\to 2i} z^2=-4.
\]
\subsection{Continuidad}

\textbf{Definición.}
Sea \(f:A\subset\mathbb{C}\to\mathbb{C}\).
Decimos que \(f\) es continua en \(z_0\in A\) si
\[
\lim_{z\to z_0} f(z)=f(z_0).
\]

El ejercicio anterior muestra que:

\textbf{Proposición.}
Sean \(f,g:A\subset\mathbb{C}\to\mathbb{C}\) continuas en \(z_0\).
Entonces
\begin{enumerate}[label=(\arabic*)]
  \item \(f+g\) es continua en \(z_0\).
  \item \(f\cdot g\) es continua en \(z_0\).
  \item \(\displaystyle \frac{f}{g}\) es continua en \(z_0\) si \(g(z_0)\neq 0\).
\end{enumerate}

\textbf{Proposición.}
Si \(f:A\subset\mathbb{C}\to\mathbb{C}\) es continua en \(z_0\) y
\(g:\mathbb{C}\to\mathbb{C}\) es continua en \(f(z_0)\), entonces \(g\circ f\)
es continua en \(z_0\).

\textbf{Ejercicio.}
Veamos que \(f(z)\equiv k\in\mathbb{C}\) es continua en todo \(z\in\mathbb{C}\).
En efecto,
\[
|f(z)-k|=|k-k|=0<\varepsilon \qquad (\forall\,\varepsilon>0).
\]

(2) \textit{Estudiemos} \(f(z)=|z|\):
dado \(\varepsilon>0\), si \(0<|z-z_0|<\delta\) con \(\delta=\varepsilon\), entonces
\[
\big||z|-|z_0|\big|\le |z-z_0|<\varepsilon.
\]

(3) Todo polinomio \(p(z)=a_0+a_1z+\cdots+a_nz^n\) es continuo (suma y
producto de continuas).

(4) Para \(f(z)=|z|\): usando \(\big||z|-|w|\big|\le |z-w|\), se obtiene continuidad.

(5) Para \(f(z)=|z|^2\): \(|\,|z|^2-|w|^2|=|z\overline{z}-w\overline{w}|
\le (|z|+|w|)\,|z-w|\), y eligiendo \(\delta\) conveniente resulta continua.

\subsection{Derivación de funciones complejas}

\textbf{Definición.}
Sea \(f:A\subset\mathbb{C}\to\mathbb{C}\). Se dice que \(f\) es
\emph{holomorfa} en \(z_0\in A\) si existe el límite
\[
f'(z_0)=\lim_{h\to 0}\frac{f(z_0+h)-f(z_0)}{h}.
\]
Se dirá que \(f\) es holomorfa en \(A\) si lo es en todo punto de \(A\); se
denota \(f\in H(A)\).
Si \(f\) es holomorfa en \(\mathbb{C}\) se dice \emph{entera}, esto es,
\(f\in H(\mathbb{C})\).

\textbf{Ejemplo.}
Veamos que \(f(z)=z\) es entera:
\[
\lim_{h\to 0}\frac{f(z_0+h)-f(z_0)}{h}
=\lim_{h\to 0}\frac{(z_0+h)-z_0}{h}=1=f'(z_0).
\]

\textbf{Proposición.}
Si \(f:A\to\mathbb{C}\) es holomorfa en \(z_0\), entonces \(f\) es continua en \(z_0\).

\textit{Demostración.}
\[
\lim_{h\to 0}\bigl(f(z_0+h)-f(z_0)\bigr)
=\lim_{h\to 0}h\,\frac{f(z_0+h)-f(z_0)}{h}=0.
\qquad\square
\]

Observamos que se cumplen todas las reglas de derivación en \(\mathbb{C}\)
de la misma forma que en \(\mathbb{R}\).
\section*{Ejercicios}

\subsection*{1.1} Dado \(z\neq 0\in\mathbb{C}\ \Rightarrow\ z^{-1}\in\mathbb{C}\).

Sabemos que \(z\,\overline z=|z|^{2}\ \Rightarrow\ z^{-1}=\dfrac{\overline z}{|z|^{2}}\).

Además, si \(z=x+iy\), entonces
\[
z^{-1}=\frac{1}{x^{2}+y^{2}}(x-iy)
= \frac{x}{x^{2}+y^{2}}-\frac{y}{x^{2}+y^{2}}\,i .
\]

Así, por ejemplo,
\[
(1+i)^{-1}=\frac{1-i}{1^{2}+1^{2}}=\frac12-\frac12\,i .
\]

Por lo tanto,
\[
1=(1+i)\Bigl(\tfrac12-\tfrac12\,i\Bigr).
\]

\subsection*{1.2} Calcula el módulo y argumento principal de los siguientes números.\\[2pt]
\textbf{(b)} \(\displaystyle \frac{i}{2-2i}\).

\[
\frac{i}{2-2i}
=\frac{i}{2-2i}\cdot\frac{2+2i}{2+2i}
=\frac{i(2+2i)}{(2)^{2}+(2)^{2}}
=\frac{2i-2}{8}
=-\frac{1}{4}+\frac{1}{4}\,i .
\]

Una vez hecho esto, calculamos su módulo y argumento:
\[
\left|-\frac14+\frac14\,i\right|
=\sqrt{\left(\frac14\right)^{2}+\left(\frac14\right)^{2}}
=\frac{\sqrt2}{4}.
\]
\[
\arg\!\left(-\tfrac14+\tfrac14\,i\right)
=\arctan\!\left(\frac{\,\tfrac14\,}{-\tfrac14}\right)+\pi
=\arctan(-1)+\pi=\frac{3\pi}{4}.
\]
\emph{*Le sumamos \(\pi\) porque estamos en el segundo cuadrante (según la convención del \(\arctan\)).*}

\subsection*{1.3} \(\displaystyle (\sqrt{3}-i)^{6}\).

Pasamos a forma exponencial:
\[
\sqrt{3}-i=2\,e^{-i\pi/6}
\quad\Longrightarrow\quad
(\sqrt{3}-i)^{6}=(2e^{-i\pi/6})^{6}=2^{6}e^{-i\pi}=64\,e^{-i\pi}.
\]

\subsection*{1.4} Representa los siguientes subconjuntos del plano complejo \(\mathbb{C}\).

\textbf{(a)} \(\displaystyle \Omega=\{\,z\in\mathbb{C}:\ |z-2+i|\le 1\,\}\).

\[
|z-2+i|\le 1\ \Longleftrightarrow\ |z-(2-i)|\le 1
\ \Longrightarrow\ 
\Omega=\overline{D}(2-i,1),
\]
lo que quedaría como un disco centrado en \(2-i\) y de radio \(1\).
c) $\Omega = \{ z \in \mathbb{C} : |2 - 4| \geq |2i| \}$  
Para analizarlo, nos fijamos en que $z - 2i$ tiene la misma distancia a $0$ que a $4$.

Así, $\Omega = \{ z \in \mathbb{C} : \Re(z) \leq 2 \}$

\subsection*{1.8 - Calcula:}

c) 
\[
\sqrt[5]{-1 - i} = \sqrt[5]{w} \left[ \cos\left( \frac{\arg w + 2k\pi}{5} \right) + i \sin\left( \frac{\arg w + 2k\pi}{5} \right) \right], \quad k = 0, 1, 2, 3, 4
\]

Aplicamos la fórmula a nuestro caso, para ello calculamos el módulo y argumento:
\[
|-1 - i| = \sqrt{2}
\]
\[
\arg(-1 - i) = -\frac{3\pi}{4}
\]

\[
= \sqrt[5]{\sqrt{2}} \left[ \cos\left( \frac{-3\pi}{20} \right) + i \sin\left( \frac{-3\pi}{20} \right) \right] \ldots
\]

\textbf{Teorema:} Sea $f: \Omega \rightarrow \mathbb{C}$. Entonces, denotando $f(z) = u(x, y) + iv(x, y)$:  
Si $f$ es holomorfa en $z_0$, es decir, existe la derivada en $z_0$, entonces existen las derivadas parciales de $u$ y $v$ verificando las llamadas \textit{ecuaciones de Cauchy-Riemann (C-R)} dadas por:

\[
\begin{cases}
u_x(x_0, y_0) = v_y(x_0, y_0) \\
u_y(x_0, y_0) = -v_x(x_0, y_0)
\end{cases}
\]

donde $u_x(x_0, y_0) = \dfrac{\partial u}{\partial x}(x_0, y_0)$, $v_x(x_0, y_0) = \dfrac{\partial v}{\partial x}(x_0, y_0)$

Además
\[
f'(z_0) = u_x(x_0, y_0) + i v_x(x_0, y_0)
\]

\textbf{Demostración:}

\[
\exists f'(z_0) = \lim_{h \to 0} \frac{f(z_0 + h) - f(z_0)}{h}
\]

\[
= \lim_{t \to 0} \frac{u(x_0 + t, y_0) + i v(x_0 + t, y_0) - u(x_0, y_0) - i v(x_0, y_0)}{t}
\]

\[
= u_x(x_0, y_0) + i v_x(x_0, y_0)
\]

\[
\lim_{t \to 0} \frac{i \big( v(x_0, y_0 + t) - v(x_0, y_0) \big) - \big( u(x_0, y_0 + t) - u(x_0, y_0) \big)}{it} = -i v_y(x_0, y_0) + u_y(x_0, y_0)
\]
\newpage
\textbf{Ejemplo.} Analicemos que $f(z) = z$ cumple las ecuaciones de Cauchy-Riemann

\[
f(x + iy) = x + iy \Rightarrow
\begin{cases}
u(x, y) = x \\
v(x, y) = y
\end{cases}
\]

Recordando las ecuaciones de Cauchy-Riemann

\[
\text{(CR)} \quad
\begin{cases}
u_x(x_0, y_0) = v_y(x_0, y_0) \\
u_y(x_0, y_0) = -v_x(x_0, y_0)
\end{cases}
\Rightarrow
\begin{cases}
1 = 1 \\
0 = 0
\end{cases}
\]

Además:
\[
f'(z) = u_x(x + iy) + i v_x(x, y) = v_y(x, y) + i u_x(x, y) = 1
\]

\textbf{Observación:} Ya sabíamos que $f(z) = z$ es holomorfa por definición.

\textbf{Ejemplo.} Analiza si $f(z) = z^2$ cumple las ecuaciones de Cauchy-Riemann

\[
f(x + iy) = (x + iy)(x + iy) = (x^2 - y^2) + i (2xy) \Rightarrow
\begin{cases}
u(x, y) = x^2 - y^2 \\
v(x, y) = 2xy
\end{cases}
\]

\[
\text{(C-R)} \quad
\begin{cases}
u_x(x, y) = 2x = v_y(x, y) \\
u_y(x, y) = -2y = -v_x(x, y)
\end{cases}
\]

Además:
\[
f'(z) = u_x(x, y) + i v_x(x, y) = 2x + i 2y = 2(x + iy) = 2z
\]

\textbf{Ejemplo.} Analicemos si $f(z) = \bar{z}$ es holomorfa usando las ecuaciones de Cauchy-Riemann

\[
f(x + iy) = x - iy \Rightarrow
\begin{cases}
u(x, y) = x \\
v(x, y) = -y
\end{cases}
\]

\[
f(z) = \bar{z} \quad \text{no es holomorfa.}
\]
\newpage
\textbf{Ejemplo:} Analicemos si $f(z) = |z|^2$ es holomorfa usando las ecuaciones de Cauchy-Riemann

\[
f(x + iy) = x^2 + y^2 \Rightarrow 
\begin{cases}
u(x, y) = x^2 + y^2 \\
v(x, y) = 0
\end{cases}
\]

Aplicamos las ecuaciones de Cauchy-Riemann

\[
\text{(C-R)} \quad 
\begin{cases}
2x = 0 \\
2y = 0
\end{cases}
\Rightarrow \text{Si } z \neq 0 \Rightarrow f(z) = |z|^2 \text{ no es holomorfa}
\]

No obstante, en $z = 0$ se cumplen las ecuaciones de Cauchy-Riemann $\nRightarrow f$ sea holomorfa en $z = 0$

Analicemos por definición si $f(z) = |z|^2$ es holomorfa en $z = 0$

\[
\lim_{h \to 0} \frac{f(0 + h) - f(0)}{h} = \lim_{h \to 0} \frac{|h|^2 - 0}{h} = \lim_{h \to 0} \frac{|h|^2}{h} = \lim_{h \to 0} \bar{h} = 0
\]

Por lo tanto $f(z) = |z|^2$ es holomorfa en $z = 0$

\textbf{Proposición:} Sea $f: \Omega \subset \mathbb{C} \rightarrow \mathbb{C}$, $f(x + iy) = u(x, y) + i v(x, y)$. Entonces

\[
\begin{aligned}
& f \text{ es holomorfa en } z_0 \Rightarrow u, v : \Omega \subset \mathbb{R}^2 \to \mathbb{R} \text{ son diferenciables en } (x_0, y_0) \\
& \text{y se cumplen las ecuaciones de Cauchy-Riemann}
\end{aligned}
\]
\[
\begin{cases}
u_x(x_0, y_0) = v_y(x_0, y_0) \\
u_y(x_0, y_0) = -v_x(x_0, y_0)
\end{cases}
\]

\textbf{Ejemplo:} Analicemos por definición si $f(z) = |z|$ es holomorfa en $z_0 \in \mathbb{C}$

\[
\lim_{h \to 0} \frac{f(z_0 + h) - f(z_0)}{h} = \lim_{h \to 0} \frac{|z_0 + h| - |z_0|}{h} = \lim_{h \to 0} \frac{|h|}{h}
\]

Vamos a acercarnos a $0$ de varias formas diferentes:

\[
\bullet \ \lim_{h \to 0^+} \frac{h}{h} = 1
\]
\[
\bullet \ \lim_{h \to 0^-} \frac{-h}{h} = -1
\]

Como 
\[
\lim_{h \to 0^+} \neq \lim_{h \to 0^-}
\]
entonces concluimos que $f$ no es holomorfa en $z_0$.\\

Acercarme de formas distintas y que dé diferente garantiza que no es holomorfa.  
Pero si da lo mismo no me garantiza que sea holomorfa.

\textbf{Proposición:} Sean $f, g : \Omega \rightarrow \mathbb{C}$ y $\alpha \in \mathbb{C}$. Entonces:

\begin{itemize}
    \item $f + g$ es holomorfa y cumple $(f + g)'(z) = f'(z) + g'(z)$
    \item $\alpha f$ es holomorfa y cumple $(\alpha f)'(z) = \alpha f'(z)$
    \item $f \cdot g$ es holomorfa y cumple $(f \cdot g)' = f' g + f g'$
    \item $\dfrac{f}{g}$ es holomorfa y cumple $\left(\dfrac{f}{g}\right)' = \dfrac{f' g - f g'}{g^2}$
    \item $f(g(z))$ es holomorfa y cumple $(f \circ g)'(z) = f'(g(z)) \cdot g'(z)$
\end{itemize}
\end{document}