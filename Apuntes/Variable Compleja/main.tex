\documentclass[a4paper,12pt]{article}

% --- Idioma y codificación ---
\usepackage[utf8]{inputenc}   % Acentos directos
\usepackage[T1]{fontenc}      % Codificación de salida
\usepackage[spanish]{babel}   % Español
\usepackage{lmodern}          % Fuente moderna

% --- Matemáticas ---
\usepackage{amsmath}          % Entornos matemáticos
\usepackage{amssymb}          % Símbolos
\usepackage{amsthm}           % Teoremas
\usepackage{mathtools}        % Extras de amsmath
\usepackage{bm}               % Negrita en símbolos
\usepackage{esint}            % Integrales de contorno (∮)

% --- Utilidades ---
\usepackage{graphicx}         % Imágenes
\usepackage{xcolor}           % Colores
\usepackage{enumitem}         % Listas
\usepackage{titlesec}         % Títulos
\usepackage[hidelinks]{hyperref} % Enlaces (cargar al final)

% --- Esquemas y Dibujos ---
\usepackage{tikz}
\usetikzlibrary{arrows.meta}

\titleformat{\subsection}
  {\normalfont\small\bfseries}   % formato más pequeño y en negrita
  {\thesubsection}{1em}{} 

\title{Apuntes de Variable Compleja}
\author{Luis López}
\date{Septiembre 2025}

\begin{document}

\maketitle
\newpage
\tableofcontents
\newpage

\section*{Introducción}

Los \textbf{números complejos}, denotados por $\mathbb{C}$, constituyen una extensión de los números reales $\mathbb{R}$, cumpliéndose que $\mathbb{R} \subset \mathbb{C}$. 
A diferencia de los reales, los complejos forman un \textit{cuerpo algebraicamente cerrado}, lo que significa que todo polinomio con coeficientes complejos admite todas sus raíces en $\mathbb{C}$.

Todo número complejo puede escribirse como 
\[
z = x + iy,
\]
donde $x, y \in \mathbb{R}$ e $i$ es la unidad imaginaria ($i^2=-1$). 
También pueden representarse en \textit{forma polar}, mediante su módulo y argumento.

El conjunto $\mathbb{C}$ no solo es fundamental en álgebra y análisis, sino que resulta indispensable en múltiples áreas de las matemáticas aplicadas y la física. 
Asimismo, los números complejos son herramientas habituales en ingeniería.

\newpage
\section{Números complejos}

\subsection{Teoría y estructura elemental}

La imposibilidad de resolver ciertas ecuaciones con números reales nos obliga a introducir los 
\textbf{números imaginarios}, definidos a partir de la unidad $i$ tal que
\[
    i^2 = -1.
\]
\subsection{Introducción elemental}

Denotamos los números complejos como
\[
    \mathbb{C} = \{ z = a + bi \;|\; a,b \in \mathbb{R} \}.
\]

Dado $z = a+bi \in \mathbb{C}$, se definen:
\begin{itemize}
    \item Parte real: $\Re(z) = a \in \mathbb{R}$.
    \item Parte imaginaria: $\Im(z) = b \in \mathbb{R}$.
    \item Módulo: $|z| = \sqrt{a^2+b^2}$.
    \item Conjugado: $\overline{z} = a - bi$.
\end{itemize}

\noindent\textbf{Ejemplo.} Sea \( z = 1 - 2i \). Entonces:
\[
\Re(1-2i) = 1, \; \Im(1-2i) = -2, \; \overline{1-2i} = 1 + 2i, \; |1-2i| = \sqrt{1^2 + (-2)^2} = \sqrt{5}.
\]
\subsection{Propiedades elementales}

\begin{enumerate}
    \item $\overline{\overline{z}} = z$. \; Demostración: si $z = a + bi \;\Rightarrow\; \overline{z} = a - bi \;\Rightarrow\; \overline{\overline{z}} = a + bi = z$.
    \item $z + \overline{z} = 2 \Re(z)$.
    \item $z - \overline{z} = 2i \Im(z)$.
    \item $|\overline{z}| = |z|$.
    \item $\overline{z + z'} = \overline{z} + \overline{z'}$.
    \item $\overline{z \cdot z'} = \overline{z} \cdot \overline{z'}$.
\end{enumerate}

Además, tenemos las siguientes propiedades asociadas al \textbf{módulo}:

\begin{enumerate}
    \item $|\Re(z)| \leq |z|$.
    \item $|\Im(z)| \leq |z|$. \; En efecto, $|z| = \sqrt{a^2 + b^2} \geq |b|$.
    \item \textbf{Desigualdad triangular:} \; $|z+w| \leq |z| + |w| \quad \forall z,w \in \mathbb{C}$.
    \item $z \cdot \overline{z} = |z|^2$.
    \item \textbf{Desigualdad triangular inversa:} \; $\big||z| - |z'|\big| \leq |z-z'|$.
\end{enumerate}

Todas estas propiedades, junto con la suma y producto de números complejos, generalizan las propiedades de los números reales:

\[
z+z' = (x+iy) + (x' + i y') = (x+x') + (y+y')i \in \mathbb{C},
\]
\[
z \cdot z' = (x+iy)(x'+iy') = (xx'-yy') + (xy'+x'y)i \in \mathbb{C}.
\]

En particular, si $z = a + bi$, se cumple que
\[
|z| = \sqrt{a^2 + b^2},
\]
es decir, el módulo de $z$ coincide con el valor absoluto en los reales.

\subsection{Forma polar y geometría de los números complejos}

El conjunto $\mathbb{C}$ se puede representar como $\mathbb{R}^2$ mediante la asignación
\[
z = a + bi \;\longmapsto\; (a, b) \in \mathbb{R}^2.
\]
De esta forma obtenemos el denominado \textbf{plano complejo}, por lo tanto la interpretación geométrica de todo los visto sería:



\end{document}