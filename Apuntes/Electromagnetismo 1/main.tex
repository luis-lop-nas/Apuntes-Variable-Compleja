\documentclass[a4paper,12pt]{article}

% --- Idioma y codificación ---
\usepackage[utf8]{inputenc}   % Acentos directos
\usepackage[T1]{fontenc}      % Codificación de salida
\usepackage[spanish]{babel}   % Español
\usepackage{lmodern}          % Fuente moderna

% --- Matemáticas ---
\usepackage{amsmath}          % Entornos matemáticos
\usepackage{amssymb}          % Símbolos
\usepackage{amsthm}           % Teoremas
\usepackage{mathtools}        % Extras de amsmath
\usepackage{bm}               % Negrita en símbolos
\usepackage{esint}            % Integrales de contorno (∮)

% --- Utilidades ---
\usepackage{graphicx}         % Imágenes
\usepackage{xcolor}           % Colores
\usepackage{enumitem}         % Listas
\usepackage{titlesec}         % Títulos
\usepackage[hidelinks]{hyperref} % Enlaces (cargar al final)

% --- Esquemas y Dibujos ---
\usepackage{tikz}
\usetikzlibrary{arrows.meta}

% subsecciones más pequeñas
\titleformat{\subsection}
  {\normalfont\small\bfseries}   % formato más pequeño y en negrita
  {\thesubsection}{1em}{} 

\title{Apuntes Electromagnetismo I}
\author{Luis López}
\date{September 2025}

\begin{document}
\maketitle % Genera la portada

\newpage 

\tableofcontents % Índice automático
\newpage 

\section*{Introducción.}

La asignatura \textbf{Electromagnetismo I}, perteneciente al \textit{Grado en Física}, aborda el estudio sistemático de los fenómenos eléctricos y magnéticos en el vacío y en medios materiales.  

Se estructura en los siguientes bloques temáticos principales:

\begin{itemize}
    \item \textbf{Tema 1. Campo electrostático en el vacío:} Fuerza eléctrica, Ley de Coulomb, Ley de Gauss y potencial eléctrico. 
    \item \textbf{Tema 2. Campo electrostático en medios materiales:} Conductores y dieléctricos, polarización y vector desplazamiento, y condiciones de continuidad de los campos.
    \item \textbf{Tema 3. Campo magnetostático en el vacío:} Movimiento de cargas y corrientes, efecto Hall, fuentes de campo magnético (leyes de Biot-Savart y Ampère), potencial vectorial y ley de Gauss magnética.
    \item \textbf{Tema 4. Campo magnetostático en medios materiales:} Propiedades magnéticas de los materiales, campo generado por un material magnetizado y condiciones de contorno.
    \item \textbf{Tema 5. Campos electromagnéticos:} Ley de Faraday y transformadores electromagnéticos.
    \item \textbf{Tema 6. Leyes de Maxwell:} formulación y aplicación de las cuatro ecuaciones fundamentales en el vacío.
\end{itemize}

\medskip

El \textbf{electromagnetismo} constituye una rama esencial de la física, ya que explica las interacciones entre las cargas eléctricas y los campos magnéticos. Dichas interacciones son responsables de una gran variedad de fenómenos naturales y tecnológicos, desde la luz visible hasta las ondas de radio, y desde el funcionamiento de los motores eléctricos hasta la transmisión y transformación de la energía eléctrica.  

La base teórica de toda la asignatura se encuentra en las \textbf{Ecuaciones de Maxwell}, que veremos en el tramo final del curso y que condensan de forma unificada cómo los campos eléctricos y magnéticos se generan, interactúan y se propagan.

\newpage

\section{Ley de Coulomb}

\noindent
La \textbf{ley de Coulomb} describe la interacción eléctrica entre dos cargas puntuales en reposo. 
Establece que la fuerza es proporcional al producto de las cargas e inversamente proporcional al cuadrado de la distancia que las separa, 
actuando a lo largo de la línea que une ambas cargas.

\[
\vec{F}_e = \frac{1}{4\pi\varepsilon_0} \frac{q_1 q_2}{r^2} \, \hat{r}_{12}
\]
Siendo $\varepsilon_0$ la \textit{permeabilidad eléctrica en el vacío}.

\[
k = \frac{1}{4\pi\varepsilon_0}
\]
Hay que poner claramente el sistema de referencia que usamos en cada caso de forma explícita.  

\subsection*{Ejemplo:}

\begin{center}
\begin{tikzpicture}[scale=1.2]
  % Ejes
  \draw[-{Latex}] (0,0) -- (3,0) node[right] {$x$};
  \draw[-{Latex}] (0,0) -- (0,3) node[above] {$y$};

  % Cargas
  \filldraw[purple] (0,0) circle (2pt) node[below left] {$q_1$};
  \filldraw[purple] (0,2) circle (2pt) node[left] {$q_2$};
  \filldraw[purple] (2,2) circle (2pt) node[right] {$q_3$};

  % Distancias
  \draw[purple] (0,0) -- (0,2) node[midway,left] {$a$};
  \draw[purple] (0,2) -- (2,2) node[midway,above] {$a$};
\end{tikzpicture}
\end{center}
Determinar la $\vec{F}_e$ sobre $q_3$. \\
Atendiendo al principio de superposición tenemos:
\[
\vec{F}_e = \vec{F}_{q_2 q_3} + \vec{F}_{q_1 q_3}
\]
Aplicamos la ley de Coulomb:
\[
\vec{F}_{q_1 q_3} = \frac{1}{4\pi \varepsilon_0} \frac{q_1 q_3}{r_{13}^2} \, \hat{r}_{13}, 
\]
calculamos el vector unitario $\hat{r}_{13}$:
\[
\vec{r}_{13} = (2a,2a), 
\qquad r_{13} = \sqrt{(2a)^2+(2a)^2} = \sqrt{8a^2} = 2\sqrt{2}\,a
\]
\[
\hat{r}_{13} = \frac{\vec{r}_{13}}{|\vec{r}_{13}|} = \frac{2a\,\hat{\imath} + 2a\,\hat{\jmath}}{2\sqrt{2}\,a} 
= \frac{\sqrt{2}}{2} (\hat{\imath} + \hat{\jmath})
\]
Entonces:\\
\[
\vec{F}_{q_1 q_3} = \frac{1}{4\pi \varepsilon_0} \frac{q_1 q_3}{(2\sqrt{2}\,a)^2} \, \hat{r}_{13}
= \frac{1}{4\pi \varepsilon_0} \frac{q_1 q_3}{8a^2} \, \frac{\sqrt{2}}{2} (\hat{\imath} + \hat{\jmath})
\]
Vemos ahora $q_2 \to q_3$:\\
\[
\vec{F}_{q_2 q_3} = \frac{1}{4\pi \varepsilon_0} \frac{q_2 q_3}{r_{23}^2} \, \hat{r}_{23}, \\
\]
Calculamos el vector unitario $\hat{r}_{23}$:
\[
\vec{r}_{23} = (a,0), \qquad r_{23}=a, \qquad \hat{r}_{23} = \frac{\vec{r}_{23}}{|\vec{r}_{23}|} = \hat{\imath}
\]
Entonces:
\[
\vec{F}_{q_2 q_3} = \frac{1}{4\pi \varepsilon_0} \frac{q_2 q_3}{a^2} \, \hat{\imath}\\
\]
Por último calculamos la fuerza total en $q_3$ con el principio de superposición:
\[
\vec{F}_e = \vec{F}_{13} + \vec{F}_{23}
\]
\[
\vec{F}_e =
\frac{1}{4\pi\varepsilon_0}\,\frac{q_1 q_3}{2a^2}\,\frac{\sqrt{2}}{2}\,(\hat{\imath}+\hat{\jmath})
\;+\;
\frac{1}{4\pi\varepsilon_0}\,\frac{q_2 q_3}{a^2}\,\hat{\imath}
\;=\;
\frac{q_3}{4\pi\varepsilon_0}\!\left[
\left(\frac{q_2}{a^2}-\frac{q_1}{2a^2}\right)\hat{\imath}
+\frac{q_1}{2a^2}\hat{\jmath}
\right]
\]

\newpage

\section{Campos eléctricos}
\noindent
El \textbf{campo eléctrico} en un punto se define como la fuerza eléctrica por unidad de carga de prueba positiva colocada en ese punto:
\[
\vec E(\vec r) \;=\; \lim_{q_0 \to 0}\,\frac{\vec F_e(\vec r)}{q_0}.
\]

\subsection*{Derivación desde la ley de Coulomb}
\noindent
Para una carga puntual $q$ situada en $\vec r'$, la fuerza sobre una carga de prueba $q_0$ en $\vec r$ es
\[
\vec F_e \;=\; \frac{1}{4\pi\varepsilon_0}\,\frac{q\,q_0}{|\vec r-\vec r'|^2}\,\hat{\mathbf R},
\qquad \hat{\mathbf R}=\frac{\vec r-\vec r'}{|\vec r-\vec r'|}.
\]
Dividiendo por $q_0$, queda:
\[
\boxed{\;\vec E(\vec r)\;=\;\frac{1}{4\pi\varepsilon_0}\,\frac{q}{|\vec r-\vec r'|^2}\,\hat{\mathbf R}\;}
\]
y, en el caso de tomar el origen en la carga y $r=|\vec r|$,
\[
\vec E_q \;=\; \frac{1}{4\pi\varepsilon_0}\,\frac{q}{r^2}\,\hat r .
\]

\begin{center}
\begin{tikzpicture}[scale=1]
% Carga positiva (sale)
\begin{scope}
  \filldraw (0,0) circle (2pt);
  \foreach \a in {0,30,...,330}{
    \draw[-{Latex}] (0,0) -- ({1.6*cos(\a)},{1.6*sin(\a)});
  }
  \node[below=6pt] at (0,0) {$+$};
  \node[below] at (0,-1.9) {\small \textit{Diverge}};
\end{scope}
% Carga negativa (entra)
\begin{scope}[xshift=4.5cm]
  \filldraw (0,0) circle (2pt);
  \foreach \a in {0,30,...,330}{
    \draw[-{Latex}] ({1.6*cos(\a)},{1.6*sin(\a)}) -- (0,0);
  }
  \node[below=6pt] at (0,0) {$-$};
  \node[below] at (0,-1.9) {\small \textit{Lo succiona}};
\end{scope}
\end{tikzpicture}
\end{center}

\[
\vec E_q \;=\; \frac{1}{4\pi\varepsilon_0}\,\frac{q}{r^2}\,\hat r_{qP}
\]

\newpage

\subsection*{Distribuciones de carga}
Pueden ser:
\[
\text{Lineales} \;\Rightarrow\; \lambda \;=\; \frac{dq}{dl},
\qquad
\text{Superficiales} \;\Rightarrow\; \sigma \;=\; \frac{dq}{dA},
\qquad
\text{Volumétrica} \;\Rightarrow\; \rho \;=\; \frac{dq}{dV}.
\]

\begin{center}
\begin{tikzpicture}[scale=1]
% Lineal
\begin{scope}
  \draw (0,0) -- (2.6,0);
  \draw (0,-0.25) -- (2.6,-0.25);
  \node at (1.3,0.2) {$\lambda$};
  \draw (1.3,-0.25) rectangle +(0.5,0.5);
  \node at (1.55,0.6) {\small $dq$};
\end{scope}
% Superficial
\begin{scope}[xshift=4cm]
  \draw (0,0) rectangle (1.6,1.0);
  \node at (0.8,1.25) {$\sigma$};
  \draw (0.55,0.35) rectangle +(0.5,0.3);
  \node at (0.8,0.9) {\small $dq$};
\end{scope}
% Volumétrica
\begin{scope}[xshift=8cm]
  \draw (0,0,0) -- (1.4,0,0) -- (1.8,0.4,0) -- (0.4,0.4,0) -- cycle;
  \draw (0,0,1) -- (1.4,0,1) -- (1.8,0.4,1) -- (0.4,0.4,1) -- cycle;
  \draw (0,0,0) -- (0,0,1);
  \draw (1.4,0,0) -- (1.4,0,1);
  \draw (1.8,0.4,0) -- (1.8,0.4,1);
  \draw (0.4,0.4,0) -- (0.4,0.4,1);
  \node at (0.9,1.3) {$\rho$};
  \draw (0.9,0.2) rectangle +(0.35,0.35);
  \node at (1.07,0.75) {\small $dq$};
\end{scope}
\end{tikzpicture}
\end{center}

\subsection*{Ejemplo:}

Tenemos una barra de longitud $l$ con una distribución de carga lineal homogénea $\lambda$.
Determine el campo eléctrico generado por la barra a una distancia $d$ de uno de sus extremos.
¿Qué ocurre cuando $d \gg l$?

\begin{center}
\begin{tikzpicture}[scale=1]
  % Eje x
  \draw[-{Latex}] (-0.5,0) -- (6.2,0) node[right] {$x$};
  % Barra de longitud l desde x=0 hasta x=l
  \draw[thick] (0,0.3) -- (4,0.3);
  \draw (0,0.15) -- (4,0.15);
  \node[above] at (2,0.35) {$\lambda$};
  \draw[-{Latex}] (0,0.55) -- node[above] {$l$} (4,0.55);
  % Punto P a distancia d del extremo derecho
  \draw[dashed] (4,0.3) -- (5.6,0.3);
  \fill (5.6,0.3) circle (1.6pt) node[above] {$P$};
  \draw[-{Latex}] (4,-0.25) -- node[below] {$d$} (5.6,-0.25);
  % Origen
  \fill (0,0) circle (1.3pt);
  \node[below] at (0,0) {\small $0$};
  % x genérico en la barra
  \draw[gray] (2,0.3) -- (2,0);
  \node[below] at (2,0) {\small $x$};
\end{tikzpicture}
\end{center}
Planteamos la solución del campo eléctrico
\[
\vec E \;=\; \frac{1}{4\pi\varepsilon_0}\int \frac{dq}{r^2}\,\hat r_{qP},
\qquad
\]
Siendo el vector unitario $\hat{r}_{qP}$:
\[
\hat r_{qP} \;=\; \frac{\vec r_P-\vec r_q}{\lvert \vec r_P-\vec r_q\rvert}
\;=\; \frac{(l+d)\,\hat{\imath}-x\,\hat{\imath}}{\lvert (l+d)-x\rvert}
\; \hat{\imath}
\]
Donde:
\[
r^2 = \bigl((l+d)-x\bigr)^2,
\qquad
dq = \lambda\,dl = \lambda\,dx.
\]
Entonces:
\[
\vec E \;=\; \frac{1}{4\pi\varepsilon_0}\int_{0}^{l}
\frac{\lambda\,dx}{\bigl((l+d)-x\bigr)^2}\,\hat{\imath}
\;=\;
\frac{\lambda}{4\pi\varepsilon_0}\left[ -\,\frac{1}{(l+d)-x} \right]_{0}^{l}\hat{\imath}
\;=\;
\frac{\lambda}{4\pi\varepsilon_0}\!\left(\frac{1}{d}-\frac{1}{l+d}\right)\hat{\imath}.
\]

\paragraph*{b) $d\gg l$}
\[
\vec E \;\approx\; \frac{\lambda}{4\pi\varepsilon_0}\left(\frac{l}{d(l+d)}\right)\hat{\imath}
\;\approx\; \frac{\lambda\,l}{4\pi\varepsilon_0\,d^{2}}\,\hat{\imath}
\;\approx\; \frac{Q}{4\pi\varepsilon_0\,d^{2}}\,\hat{\imath}
\quad\text{con } Q=\lambda\,l.
\]
Llegamos a la expresión del campo eléctrico de una carga puntual $Q$. ya que al estar tan alejados la barra se comporta como una carga puntual.

\newpage

\subsection*{Ejemplo:}

Calcular $\vec E$ en el punto $P$.  
¿Qué ocurre si $P \gg L$?

\begin{center}
\begin{tikzpicture}[scale=1]
  % Ejes
  \draw[-{Latex}] (0,0) -- (6,0) node[right] {$x$};
  \draw[-{Latex}] (0,0) -- (0,3) node[above] {$y$};
  
  % Barra
  \draw[thick] (1,0) -- (5,0);
  \node[below] at (3,0) {$\lambda$};
  \draw (1,-0.15) -- (1,0.15);
  \draw (5,-0.15) -- (5,0.15);
  \draw[-{Latex}] (1,-0.5) -- node[below] {$L$} (5,-0.5);
  
  % Punto P
  \draw[dashed] (3,0) -- (3,2);
  \fill (3,2) circle (2pt) node[above] {$P$};
  \node[right] at (3,1) {$y$};
\end{tikzpicture}
\end{center}

\noindent Pista: 

\[
\int \frac{y}{(x^2+y^2)^{3/2}}\,dx \;=\; \frac{x}{y\sqrt{x^2+y^2}}
\]
La solución del campo será:

\[
\vec E = \frac{1}{4\pi\varepsilon_0} \int \frac{dq}{r^2}\,\hat r_{qP},
\qquad
\hat r_{qP} = \frac{\vec r_P-\vec r_q}{|\vec r_P-\vec r_q|}
= \frac{-x}{\sqrt{x^2+y^2}}\,\hat{\imath} + \frac{y}{\sqrt{x^2+y^2}}\,\hat{\jmath}.
\]
Sustituyendo:
\[
\vec E = \frac{1}{4\pi\varepsilon_0} \int_{-L/2}^{L/2}
\frac{\lambda\,dx}{(x^2+y^2)}\,
\left( -\frac{x}{\sqrt{x^2+y^2}}\,\hat{\imath}
+ \frac{y}{\sqrt{x^2+y^2}}\,\hat{\jmath} \right)
\]
Al tratarse de una integral vectorial, se integra cada componente por separado:

\[
\vec E =
\frac{\lambda}{4\pi\varepsilon_0} \left[
\int_{-L/2}^{L/2} \frac{-x}{(x^2+y^2)^{3/2}}\,dx \;\hat{\imath}
+ \int_{-L/2}^{L/2} \frac{y}{(x^2+y^2)^{3/2}}\,dx \;\hat{\jmath}
\right]
\]

\[
= \frac{\lambda}{4\pi\varepsilon_0} \left[
\left. -\frac{1}{\sqrt{x^2+y^2}} \right|_{-L/2}^{L/2} \hat{\imath}
+ \left. \frac{x}{y\sqrt{x^2+y^2}} \right|_{-L/2}^{L/2} \hat{\jmath}
\right]
\]


\newpage
\subsection{Coordenadas cilíndricas}

\noindent Las \textbf{coordenadas cilíndricas} representan un punto en el espacio mediante la tripleta
 $(r,\theta,z)$, donde $r$ es la distancia desde el punto al eje $z$ (coordenada radial),
  $\theta$ es el ángulo que forma el radio con el eje $x$ (coordenada acimutal) y $z$ es la
   altura del punto sobre el plano $xy$ (coordenada vertical). Este sistema es una generalización
    de las coordenadas polares al espacio tridimensional y es útil en problemas con simetría cilíndrica.\\

\noindent Siendo las nuevas coordenadas:
\[
x = r\cos\theta, \quad y = r\sin\theta, \quad z = z
\]

% --- Medida de \theta y croquis ---
\begin{center}
\begin{tikzpicture}[scale=1]
  % Ejes "3D" esquemáticos
  \draw[-{Latex}] (0,0) -- (2.2,0) node[right] {$x$};
  \draw[-{Latex}] (0,0) -- (-1.2,-0.7);
  \node at (-1.4,-0.85) {$y$};
  \draw[-{Latex}] (0,0) -- (0,2.2) node[above] {$z$};

  % Punto y proyección al plano xy
  \fill (1.1,1.6) circle (1.2pt);
  \draw[dashed] (1.1,1.6) -- (1.1,0);
  \draw (0,0) -- (1.1,0);

  % Arco y ángulo theta en el plano xy
  \draw (0.6,0) arc[start angle=0,end angle=330,radius=0.6];
  \node at (0.75,-0.28) {$\theta$};
\end{tikzpicture}
\hspace{2cm}
\begin{tikzpicture}[scale=1]
  \draw[-{Latex}] (0,0) -- (2,0) node[right] {$x$};
  \draw[-{Latex}] (0,0) -- (0,2) node[above] {$y$};
  \draw[-{Latex}] (0,0) -- (1.5,1.0);
  \draw (0.8,0) arc[start angle=0,end angle=33.7,radius=0.8];
  \node at (1.05,0.25) {$\theta$};
\end{tikzpicture}
\end{center}
Donde $\theta$ va siempre desde el eje positivo de $x$ al eje positivo de $y$.
Al igual que en coordenadas cartesianas, las coordenadas cilíndricas también tienen \textit{vectores unitarios}:
\[
\hat{\mathbf e}_r = \cos\theta\,\hat{\imath} + \sin\theta\,\hat{\jmath},\qquad
\hat{\mathbf e}_\theta = -\sin\theta\,\hat{\imath} + \cos\theta\,\hat{\jmath},\qquad
\hat{\mathbf e}_z = \hat{\mathbf k}.
\]

\begin{center}
\begin{tikzpicture}[scale=1]
  % Ejes "3D" esquemáticos
  \draw[-{Latex}] (0,0) -- (2.2,0) node[right] {$x$};
  \draw[-{Latex}] (0,0) -- (-1.2,-0.7);
  \node at (-1.4,-0.85) {$y$};
  \draw[-{Latex}] (0,0) -- (0,2.2) node[above] {$z$};

  % Arco y versores en el plano xy
  \draw (0.9,0) arc[start angle=0,end angle=330,radius=0.9];
  \node at (0.7,-0.2) {$\theta$};

  % e_r y e_theta en el plano xy
  \draw[-{Latex}] (0,0) -- (1.4,0.5) node[right] {$\hat{\mathbf e}_r$};
  \draw[-{Latex}] (0.9,0.32) -- (0.6,0.95) node[above] {$\hat{\mathbf e}_\theta$};
\end{tikzpicture}
\end{center}

\newpage

\subsection{Diferenciales de longitud, área y volumen (cartesianas)}

De longitud:
\[
d\vec{\ell} = dx\,\hat{\imath} + dy\,\hat{\jmath} + dz\,\hat{\mathbf k}
\]

De área:
\[
\begin{aligned}
d\vec{A}_x &= dy\,dz\,\hat{\imath},\\
d\vec{A}_y &= dx\,dz\,\hat{\jmath},\\
d\vec{A}_z &= dx\,dy\,\hat{\mathbf k}.
\end{aligned}
\]

De volumen:
\[
dV = dx\,dy\,dz
\]

\textit{Hacer esquemas visuales.}


\subsection{Diferenciales de longitud, área y volumen (cilíndricas)}

De longitud:
\[
d\vec{\ell} = dr\,\hat{\mathbf e}_r + r\,d\theta\,\hat{\mathbf e}_\theta + dz\,\hat{\mathbf e}_z
\]

De área:
\[
\begin{aligned}
d\vec{S}_r      &= r\,d\theta\,dz\,\hat{\mathbf e}_r,\\
d\vec{S}_\theta &= dr\,dz\,\hat{\mathbf e}_\theta,\\
d\vec{S}_z      &= r\,dr\,d\theta\,\hat{\mathbf e}_z.
\end{aligned}
\]

De volumen:
\[
dV = r\,dr\,d\theta\,dz
\]

\begin{center}
\begin{tikzpicture}[scale=1]
  % Bloque cilíndrico elemental (croquis)
  \draw (0,0) -- (2,0) -- (2,2) -- (0,2) -- cycle;
  \draw (2,0) -- (3.2,0.6) -- (3.2,2.6) -- (2,2);
  \draw (0,2) -- (1.2,2.6) -- (3.2,2.6);
  % Arco y ángulo d\theta
  \draw (0,0) -- (2,0);
  \draw (0,0) -- (1.8,0.6);
  \draw (1,0) arc[start angle=0,end angle=17,radius=1];
  \node at (1.25,0.22) {\small $d\theta$};
  % Radios r y dr (indicativos)
  \draw[-{Latex}] (0,0) -- (1.2,0) node[below] {\small $r$};
\draw[-{Latex}] (1.2,0) -- (1.8,0) node[below] {\small $dr$};
\end{tikzpicture}
\end{center}

\subsection*{Ejemplo:}
\noindent Un anillo cargado con densidad de carga lineal variable $\lambda=\cos^{2}\theta$ y radio $a$.
Determinar la carga total.

La densidad lineal, por tanto la carga vendrá dada como:
\[
q_T \;=\; \oint \lambda\, dl
 \;=\; \int_{0}^{2\pi} \lambda\, a\, d\theta
 \;=\; a \int_{0}^{2\pi} \cos^{2}\theta\, d\theta .
\]

\textit{* Los límites de integración varían dependiendo de nuestro sistema de referencia.  
En nuestro caso $r=a$ y $\theta\in[0,2\pi]$.}

\bigskip

\subsection*{Ejemplo:}

\noindent
Un disco cargado con densidad de carga superficial $\sigma=\rho\,\sin^{2}\theta$ y radio $R$.  
Determina la \textbf{carga total} del disco.

\[
q_T \;=\; \iint_{S} \sigma\, dA
   \;=\; \int_{0}^{2\pi}\!\left(\int_{0}^{R} \rho\,\sin^{2}\theta \; r\,dr\right)\! d\theta .
\]

\subsection*{Ejemplo:}
\noindent
Un anillo cargado con densidad de carga lineal constante $\lambda$ y radio $a$.  
Determine el valor del \textbf{campo eléctrico} en el punto $P$ (sobre el eje del anillo, a altura $z$).
\[
\vec E
= \frac{1}{4\pi\varepsilon_0}\oint \frac{dq}{r^2}\,\hat r
= \frac{1}{4\pi\varepsilon_0}\int_{0}^{2\pi}
   \frac{\lambda a\, d\theta}{a^{2}+z^{2}}\;
   \frac{-a\cos\theta\,\hat{\imath}\;-\;a\sin\theta\,\hat{\jmath}\;+\;z\,\hat{\mathbf k}}
        {\sqrt{a^{2}+z^{2}}}.
\]
Como integral vectorial, se integran las componentes por separado:
\[
\int_{0}^{2\pi} \frac{-a\cos\theta}{(a^{2}+z^{2})^{3/2}}\, d\theta = 0, \qquad
\int_{0}^{2\pi} \frac{-a\sin\theta}{(a^{2}+z^{2})^{3/2}}\, d\theta = 0,
\]
\[
\int_{0}^{2\pi} \frac{z}{(a^{2}+z^{2})^{3/2}}\, d\theta
= \frac{2\pi z}{(a^{2}+z^{2})^{3/2}}.
\]
Por tanto,
\[
\boxed{\;
\vec E(P) \;=\; \frac{\lambda a z}{2\,\varepsilon_0\,(a^{2}+z^{2})^{3/2}}\;\hat{\mathbf k}
\;}
\]
\subsection{Coordenadas esféricas}
\noindent Las \textbf{coordenadas cilíndricas} representan un punto en el espacio mediante la tripleta
 $(r,\theta,\phi)$, donde $r$ es la distancia desde el punto al origen (coordenada radial),
  $\theta$ es el ángulo que forma el radio con el eje $x$ (coordenada acimutal) y $\phi$ es el
   ángulo que forma el radio con el eje $z$ (coordenada polar). Este sistema es una generalización
    de las coordenadas polares al espacio tridimensional y es útil en problemas con simetría esférica.\\

\noindent Siendo las nuevas coordenadas:
\[
x = r\sin\theta\cos\phi, \quad y = r\sin\theta\sin\phi, \quad z = r\cos\theta
\]

\end{document}