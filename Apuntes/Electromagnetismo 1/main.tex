\documentclass[a4paper,12pt]{article}

% --- Idioma y codificación ---
\usepackage[utf8]{inputenc}   % Acentos directos
\usepackage[T1]{fontenc}      % Codificación de salida
\usepackage[spanish]{babel}   % Español
\usepackage{lmodern}          % Fuente moderna

% --- Matemáticas ---
\usepackage{amsmath}          % Entornos matemáticos
\usepackage{amssymb}          % Símbolos
\usepackage{amsthm}           % Teoremas
\usepackage{mathtools}        % Extras de amsmath
\usepackage{bm}               % Negrita en símbolos
\usepackage{esint}            % Integrales de contorno (∮)

\usepackage{graphicx}         % Imágenes
% Rutas de búsqueda para imágenes
\graphicspath{{./}{imagenes/}}
\DeclareGraphicsExtensions{.pdf,.png,.jpg,.jpeg}
\usepackage{xcolor}           % Colores
\usepackage{enumitem}         % Listas
\usepackage{titlesec}         % Títulos
\usepackage[hidelinks]{hyperref} % Enlaces (cargar al final)

% --- Esquemas y Dibujos ---
\usepackage{tikz}
\usetikzlibrary{arrows.meta}

% subsecciones más pequeñas
\titleformat{\subsection}
  {\normalfont\small\bfseries}   % formato más pequeño y en negrita
  {\thesubsection}{1em}{} 

\title{Apuntes Electromagnetismo I}
\author{Luis López}
\date{September 2025}

\begin{document}
\maketitle % Genera la portada

\newpage 

\tableofcontents % Índice automático
\newpage 

\section*{Introducción.}

La asignatura \textbf{Electromagnetismo I}, perteneciente al \textit{Grado en Física}, aborda el estudio sistemático de los fenómenos eléctricos y magnéticos en el vacío y en medios materiales.  

Se estructura en los siguientes bloques temáticos principales:

\begin{itemize}
    \item \textbf{Tema 1. Campo electrostático en el vacío:} Fuerza eléctrica, Ley de Coulomb, Ley de Gauss y potencial eléctrico. 
    \item \textbf{Tema 2. Campo electrostático en medios materiales:} Conductores y dieléctricos, polarización y vector desplazamiento, y condiciones de continuidad de los campos.
    \item \textbf{Tema 3. Campo magnetostático en el vacío:} Movimiento de cargas y corrientes, efecto Hall, fuentes de campo magnético (leyes de Biot-Savart y Ampère), potencial vectorial y ley de Gauss magnética.
    \item \textbf{Tema 4. Campo magnetostático en medios materiales:} Propiedades magnéticas de los materiales, campo generado por un material magnetizado y condiciones de contorno.
    \item \textbf{Tema 5. Campos electromagnéticos:} Ley de Faraday y transformadores electromagnéticos.
    \item \textbf{Tema 6. Leyes de Maxwell:} formulación y aplicación de las cuatro ecuaciones fundamentales en el vacío.
\end{itemize}

\medskip
\noindent
El \textbf{electromagnetismo} constituye una rama esencial de la física, ya que explica las interacciones entre las cargas eléctricas y los campos magnéticos. Dichas interacciones son responsables de una gran variedad de fenómenos naturales y tecnológicos, desde la luz visible hasta las ondas de radio, y desde el funcionamiento de los motores eléctricos hasta la transmisión y transformación de la energía eléctrica.\\ 

\noindent
La base teórica de toda la asignatura se encuentra en las \textbf{Ecuaciones de Maxwell}, que veremos en el tramo final del curso y que condensan de forma unificada cómo los campos eléctricos y magnéticos se generan, interactúan y se propagan.

\newpage

\section{Ley de Coulomb}

\noindent
La \textbf{ley de Coulomb} describe la interacción eléctrica entre dos cargas puntuales en reposo. 
Establece que la fuerza es proporcional al producto de las cargas e inversamente proporcional al cuadrado de la distancia que las separa, 
actuando a lo largo de la línea que une ambas cargas.

\[
\vec{F}_e = \frac{1}{4\pi\varepsilon_0} \frac{q_1 q_2}{r^2} \, \hat{r}_{12}
\]
Siendo $\varepsilon_0$ la \textit{permeabilidad eléctrica en el vacío}.

\[
k = \frac{1}{4\pi\varepsilon_0}
\]
Hay que poner claramente el sistema de referencia que usamos en cada caso de forma explícita.  

\subsection*{Ejemplo:}
%%% Figura 1 %%%
\begin{figure}[h]
  \centering
  \includegraphics[width=0.35\textwidth]{imagen_1.jpeg}
  \caption{Descripción breve de la imagen.}
  \label{fig:imagen1}
\end{figure}
Determinar la $\vec{F}_e$ sobre $q_3$. \\
Atendiendo al principio de superposición tenemos:
\[
\vec{F}_e = \vec{F}_{q_2 q_3} + \vec{F}_{q_1 q_3}
\]
Aplicamos la ley de Coulomb:
\[
\vec{F}_{q_1 q_3} = \frac{1}{4\pi \varepsilon_0} \frac{q_1 q_3}{r_{13}^2} \, \hat{r}_{13}, 
\]
calculamos el vector unitario $\hat{r}_{13}$:
\[
\vec{r}_{13} = (2a,2a), 
\qquad r_{13} = \sqrt{(2a)^2+(2a)^2} = \sqrt{8a^2} = 2\sqrt{2}\,a
\]
\[
\hat{r}_{13} = \frac{\vec{r}_{13}}{|\vec{r}_{13}|} = \frac{2a\,\hat{\imath} + 2a\,\hat{\jmath}}{2\sqrt{2}\,a} 
= \frac{\sqrt{2}}{2} (\hat{\imath} + \hat{\jmath})
\]
Entonces:\\
\[
\vec{F}_{q_1 q_3} = \frac{1}{4\pi \varepsilon_0} \frac{q_1 q_3}{(2\sqrt{2}\,a)^2} \, \hat{r}_{13}
= \frac{1}{4\pi \varepsilon_0} \frac{q_1 q_3}{8a^2} \, \frac{\sqrt{2}}{2} (\hat{\imath} + \hat{\jmath})
\]
Vemos ahora $q_2 \to q_3$:\\
\[
\vec{F}_{q_2 q_3} = \frac{1}{4\pi \varepsilon_0} \frac{q_2 q_3}{r_{23}^2} \, \hat{r}_{23}, \\
\]
Calculamos el vector unitario $\hat{r}_{23}$:
\[
\vec{r}_{23} = (a,0), \qquad r_{23}=a, \qquad \hat{r}_{23} = \frac{\vec{r}_{23}}{|\vec{r}_{23}|} = \hat{\imath}
\]
Entonces:
\[
\vec{F}_{q_2 q_3} = \frac{1}{4\pi \varepsilon_0} \frac{q_2 q_3}{a^2} \, \hat{\imath}\\
\]
Por último calculamos la fuerza total en $q_3$ con el principio de superposición:
\[
\vec{F}_e = \vec{F}_{13} + \vec{F}_{23}
\]
\[
\vec{F}_e =
\frac{1}{4\pi\varepsilon_0}\,\frac{q_1 q_3}{2a^2}\,\frac{\sqrt{2}}{2}\,(\hat{\imath}+\hat{\jmath})
\;+\;
\frac{1}{4\pi\varepsilon_0}\,\frac{q_2 q_3}{a^2}\,\hat{\imath}
\;=\;
\frac{q_3}{4\pi\varepsilon_0}\!\left[
\left(\frac{q_2}{a^2}-\frac{q_1}{2a^2}\right)\hat{\imath}
+\frac{q_1}{2a^2}\hat{\jmath}
\right]
\]

\newpage

\section{Campos eléctricos}
\noindent
El \textbf{campo eléctrico} en un punto se define como la fuerza eléctrica por unidad de carga de prueba positiva colocada en ese punto:
\[
\vec E(\vec r) \;=\; \lim_{q_0 \to 0}\,\frac{\vec F_e(\vec r)}{q_0}.
\]

\subsection*{Derivación desde la ley de Coulomb}
\noindent
Para una carga puntual $q$ situada en $\vec r'$, la fuerza sobre una carga de prueba $q_0$ en $\vec r$ es
\[
\vec F_e \;=\; \frac{1}{4\pi\varepsilon_0}\,\frac{q\,q_0}{|\vec r-\vec r'|^2}\,\hat{\mathbf R},
\qquad \hat{\mathbf R}=\frac{\vec r-\vec r'}{|\vec r-\vec r'|}.
\]
Dividiendo por $q_0$, queda:
\[
\boxed{\;\vec E(\vec r)\;=\;\frac{1}{4\pi\varepsilon_0}\,\frac{q}{|\vec r-\vec r'|^2}\,\hat{\mathbf R}\;}
\]
y, en el caso de tomar el origen en la carga y $r=|\vec r|$,
\[
\vec E_q \;=\; \frac{1}{4\pi\varepsilon_0}\,\frac{q}{r^2}\,\hat r .
\]
%%% Figura 2 %%%
\begin{center}
\begin{tikzpicture}[scale=1]
% Carga positiva (sale)
\begin{scope}
  \filldraw (0,0) circle (2pt);
  \foreach \a in {0,30,...,330}{
    \draw[-{Latex}] (0,0) -- ({1.6*cos(\a)},{1.6*sin(\a)});
  }
  \node[below] at (0,-1.9) {\small \textit{Diverge}};
\end{scope}
% Carga negativa (entra)
\begin{scope}[xshift=4.5cm]
  \filldraw (0,0) circle (2pt);
  \foreach \a in {0,30,...,330}{
    \draw[-{Latex}] ({1.6*cos(\a)},{1.6*sin(\a)}) -- (0,0);
  }
  \node[below] at (0,-1.9) {\small \textit{Lo succiona}};
\end{scope}
\end{tikzpicture}
\end{center}

\[
\vec E_q \;=\; \frac{1}{4\pi\varepsilon_0}\,\frac{q}{r^2}\,\hat r_{qP}
\]

\newpage

\subsection*{Distribuciones de carga}
Pueden ser:
\[
\text{Lineales} \;\Rightarrow\; \lambda \;=\; \frac{dq}{dl},
\qquad
\text{Superficiales} \;\Rightarrow\; \sigma \;=\; \frac{dq}{dA},
\qquad
\text{Volumétrica} \;\Rightarrow\; \rho \;=\; \frac{dq}{dV}.
\]
%%% Figura 3 %%%
\vspace{-2.0em}
\begin{figure}[h]
  \centering
  \includegraphics[width=1.0\textwidth]{imagen_2.jpeg}
  \caption{Descripción breve de la imagen.}
  \label{fig:imagen2}
\end{figure}

\subsection*{Ejemplo:}
\noindent
Tenemos una barra de longitud $l$ con una distribución de carga lineal homogénea $\lambda$.
Determine el campo eléctrico generado por la barra a una distancia $d$ de uno de sus extremos.
¿Qué ocurre cuando $d \gg l$?

%%% Figura 4 %%%
\begin{figure}[h]
  \centering
  \includegraphics[width=0.8\textwidth]{imagen_3.jpeg}
  \caption{Descripción breve de la imagen.}
  \label{fig:imagen3}
\end{figure}
Planteamos la solución del campo eléctrico
\[
\vec E \;=\; \frac{1}{4\pi\varepsilon_0}\int \frac{dq}{r^2}\,\hat r_{qP},
\qquad
\]
Siendo el vector unitario $\hat{r}_{qP}$:
\[
\hat r_{qP} \;=\; \frac{\vec r_P-\vec r_q}{\lvert \vec r_P-\vec r_q\rvert}
\;=\; \frac{(l+d)\,\hat{\imath}-x\,\hat{\imath}}{\lvert (l+d)-x\rvert}
\; \hat{\imath}
\]
Donde:
\[
r^2 = \bigl((l+d)-x\bigr)^2,
\qquad
dq = \lambda\,dl = \lambda\,dx.
\]
Entonces:
\[
\vec E \;=\; \frac{1}{4\pi\varepsilon_0}\int_{0}^{l}
\frac{\lambda\,dx}{\bigl((l+d)-x\bigr)^2}\,\hat{\imath}
\;=\;
\frac{\lambda}{4\pi\varepsilon_0}\left[ -\,\frac{1}{(l+d)-x} \right]_{0}^{l}\hat{\imath}
\;=\;
\frac{\lambda}{4\pi\varepsilon_0}\!\left(\frac{1}{d}-\frac{1}{l+d}\right)\hat{\imath}.
\]

\paragraph*{(b) $d\gg l$}
\[
\vec E \;\approx\; \frac{\lambda}{4\pi\varepsilon_0}\left(\frac{l}{d(l+d)}\right)\hat{\imath}
\;\approx\; \frac{\lambda\,l}{4\pi\varepsilon_0\,d^{2}}\,\hat{\imath}
\;\approx\; \frac{Q}{4\pi\varepsilon_0\,d^{2}}\,\hat{\imath}
\quad\text{con } Q=\lambda\,l.
\]
Llegamos a la expresión del campo eléctrico de una carga puntual $Q$. ya que al estar tan alejados la barra se comporta como una carga puntual.

\newpage

\subsection*{Ejemplo:}

Calcular $\vec E$ en el punto $P$.  
¿Qué ocurre si $P \gg L$?

%%% Figura 5 %%%
\begin{figure}[h]
  \centering
  \includegraphics[width=0.5\textwidth]{imagen_4.jpeg}
  \caption{Descripción breve de la imagen.}
  \label{fig:imagen4}
\end{figure}

\noindent Pista: 

\[
\int \frac{y}{(x^2+y^2)^{3/2}}\,dx \;=\; \frac{x}{y\sqrt{x^2+y^2}}
\]
La solución del campo será:

\[
\vec E = \frac{1}{4\pi\varepsilon_0} \int \frac{dq}{r^2}\,\hat r_{qP},
\qquad
\hat r_{qP} = \frac{\vec r_P-\vec r_q}{|\vec r_P-\vec r_q|}
= \frac{-x}{\sqrt{x^2+y^2}}\,\hat{\imath} + \frac{y}{\sqrt{x^2+y^2}}\,\hat{\jmath}.
\]
Sustituyendo:
\[
\vec E = \frac{1}{4\pi\varepsilon_0} \int_{-L/2}^{L/2}
\frac{\lambda\,dx}{(x^2+y^2)}\,
\left( -\frac{x}{\sqrt{x^2+y^2}}\,\hat{\imath}
+ \frac{y}{\sqrt{x^2+y^2}}\,\hat{\jmath} \right)
\]
Al tratarse de una integral vectorial, se integra cada componente por separado:

\[
\vec E =
\frac{\lambda}{4\pi\varepsilon_0} \left[
\int_{-L/2}^{L/2} \frac{-x}{(x^2+y^2)^{3/2}}\,dx \;\hat{\imath}
+ \int_{-L/2}^{L/2} \frac{y}{(x^2+y^2)^{3/2}}\,dx \;\hat{\jmath}
\right]
\]

\[
= \frac{\lambda}{4\pi\varepsilon_0} \left[
\left. -\frac{1}{\sqrt{x^2+y^2}} \right|_{-L/2}^{L/2} \hat{\imath}
+ \left. \frac{x}{y\sqrt{x^2+y^2}} \right|_{-L/2}^{L/2} \hat{\jmath}
\right]
\]


\newpage
\subsection{Coordenadas cilíndricas}

\noindent Las \textbf{coordenadas cilíndricas} representan un punto en el espacio mediante la tripleta
 $(r,\theta,z)$, donde $r$ es la distancia desde el punto al eje $z$ (coordenada radial),
  $\theta$ es el ángulo que forma el radio con el eje $x$ (coordenada acimutal) y $z$ es la
   altura del punto sobre el plano $xy$ (coordenada vertical). Este sistema es una generalización
    de las coordenadas polares al espacio tridimensional y es útil en problemas con simetría cilíndrica.\\

\noindent Siendo las nuevas coordenadas:
\[
x = r\cos\theta, \quad y = r\sin\theta, \quad z = z
\]
%%% Figura 6 %%%
\vspace{-2.0em}
\begin{figure}[h]
  \centering
  \includegraphics[width=0.65\textwidth]{imagen_5.jpeg}
  \caption{Descripción breve de la imagen.}
  \label{fig:imagen5}
\end{figure}

  % tus fórmulas...
Donde $\theta$ va siempre desde el eje positivo de $x$ al eje positivo de $y$.
Al igual que en coordenadas cartesianas, las coordenadas cilíndricas también tienen \textit{vectores unitarios}:
\[
\hat{\mathbf e}_r = \cos\theta\,\hat{\imath} + \sin\theta\,\hat{\jmath},\qquad
\hat{\mathbf e}_\theta = -\sin\theta\,\hat{\imath} + \cos\theta\,\hat{\jmath},\qquad
\hat{\mathbf e}_z = \hat{\mathbf k}.
\]
%%% figura 7 %%%
\vspace{-2.0em}
\begin{figure}[h]
  \centering
  \includegraphics[width=0.3\textwidth]{imagen_6.jpeg}
  \caption{Descripción breve de la imagen.}
  \label{fig:imagen6}
\end{figure}
\newpage

\subsection{Diferenciales de longitud, área y volumen (cartesianas)}

De longitud:
\[
d\vec{\ell} = dx\,\hat{\imath} + dy\,\hat{\jmath} + dz\,\hat{\mathbf k}
\]

De área:
\[
\begin{aligned}
d\vec{A}_x &= dy\,dz\,\hat{\imath},\\
d\vec{A}_y &= dx\,dz\,\hat{\jmath},\\
d\vec{A}_z &= dx\,dy\,\hat{\mathbf k}.
\end{aligned}
\]

De volumen:
\[
dV = dx\,dy\,dz
\]

\textit{Hacer esquemas visuales.}


\subsection{Diferenciales de longitud, área y volumen (cilíndricas)}

De longitud:
\[
d\vec{\ell} = dr\,\hat{\mathbf e}_r + r\,d\theta\,\hat{\mathbf e}_\theta + dz\,\hat{\mathbf e}_z
\]

De área:
\[
\begin{aligned}
d\vec{S}_r      &= r\,d\theta\,dz\,\hat{\mathbf e}_r,\\
d\vec{S}_\theta &= dr\,dz\,\hat{\mathbf e}_\theta,\\
d\vec{S}_z      &= r\,dr\,d\theta\,\hat{\mathbf e}_z.
\end{aligned}
\]

De volumen:
\[
dV = r\,dr\,d\theta\,dz
\]
%%% Figura 8 %%%
\vspace{-2.0em}
\begin{figure}[h]
  \centering
  \includegraphics[width=0.25\textwidth]{imagen_7.jpeg}
  \caption{Descripción breve de la imagen.}
  \label{fig:imagen7}
\end{figure}


\subsection*{Ejemplo:}
\noindent Un anillo cargado con densidad de carga lineal variable $\lambda=\cos^{2}\theta$ y radio $a$.
Determinar la carga total.

La densidad lineal, por tanto la carga vendrá dada como:
\[
q_T \;=\; \oint \lambda\, dl
 \;=\; \int_{0}^{2\pi} \lambda\, a\, d\theta
 \;=\; a \int_{0}^{2\pi} \cos^{2}\theta\, d\theta .
\]

\textit{* Los límites de integración varían dependiendo de nuestro sistema de referencia.  
En nuestro caso $r=a$ y $\theta\in[0,2\pi]$.}


\subsection*{Ejemplo:}

\noindent
Un disco cargado con densidad de carga superficial $\sigma=\rho\,\sin^{2}\theta$ y radio $R$.  
Determina la \textbf{carga total} del disco.

\[
q_T \;=\; \iint_{S} \sigma\, dA
   \;=\; \int_{0}^{2\pi}\!\left(\int_{0}^{R} \rho\,\sin^{2}\theta \; r\,dr\right)\! d\theta .
\]

\subsection*{Ejemplo:}
\noindent
Un anillo cargado con densidad de carga lineal constante $\lambda$ y radio $a$.  
Determine el valor del \textbf{campo eléctrico} en el punto $P$ (sobre el eje del anillo, a altura $z$).
\[
\vec E
= \frac{1}{4\pi\varepsilon_0}\oint \frac{dq}{r^2}\,\hat r
= \frac{1}{4\pi\varepsilon_0}\int_{0}^{2\pi}
   \frac{\lambda a\, d\theta}{a^{2}+z^{2}}\;
   \frac{-a\cos\theta\,\hat{\imath}\;-\;a\sin\theta\,\hat{\jmath}\;+\;z\,\hat{\mathbf k}}
        {\sqrt{a^{2}+z^{2}}}.
\]
Como integral vectorial, se integran las componentes por separado:
\[
\int_{0}^{2\pi} \frac{-a\cos\theta}{(a^{2}+z^{2})^{3/2}}\, d\theta = 0, \qquad
\int_{0}^{2\pi} \frac{-a\sin\theta}{(a^{2}+z^{2})^{3/2}}\, d\theta = 0,
\]
\[
\int_{0}^{2\pi} \frac{z}{(a^{2}+z^{2})^{3/2}}\, d\theta
= \frac{2\pi z}{(a^{2}+z^{2})^{3/2}}.
\]
Por tanto,
\[
\boxed{\;
\vec E(P) \;=\; \frac{\lambda a z}{2\,\varepsilon_0\,(a^{2}+z^{2})^{3/2}}\;\hat{\mathbf k}
\;}
\]
\newpage
\subsection{Coordenadas esféricas}
\noindent Las \textbf{coordenadas cilíndricas} representan un punto en el espacio mediante la tripleta
 $(r,\theta,\phi)$, donde $r$ es la distancia desde el punto al origen (coordenada radial),
  $\theta$ es el ángulo que forma el radio con el eje $x$ (coordenada acimutal) y $\phi$ es el
   ángulo que forma el radio con el eje $z$ (coordenada polar). Este sistema es una generalización
    de las coordenadas polares al espacio tridimensional y es útil en problemas con simetría esférica.\\

\noindent Siendo las nuevas coordenadas:
\[
x = r\sin\theta\cos\phi, \quad y = r\sin\theta\sin\phi, \quad z = r\cos\theta
\]
\vspace{-2.0em}
\begin{figure}[htbp]
  \centering
  % Si renombraste el archivo:
  \includegraphics[width=0.40\textwidth]{esfericas.jpg}
  % O si mantienes el nombre con espacios:
  % \includegraphics[width=0.82\textwidth]{Esquemas\ Y\ Dibujos.jpeg}
  \caption{Coordenadas esféricas: $r$, $\theta$, $\varphi$.}
  \label{fig:esfericas}
\end{figure}

\subsection*{Diferenciales de longitud, área y volumen (esféricas)}

De longitud:
\[
d\vec{\ell} = dr\,\hat{\mathbf e}_r + r\,d\theta\,\hat{\mathbf e}_\theta + r\sin\theta\,d\phi\,\hat{\mathbf e}_\phi
\]

De área:
\[
\begin{aligned}
d\vec{S}_r      &= r^{2}\sin\theta\,d\theta\,d\phi\,\hat{\mathbf e}_r, \\
d\vec{S}_\theta &= r\sin\theta\,dr\,d\phi\,\hat{\mathbf e}_\theta, \\
d\vec{S}_\phi   &= r\,dr\,d\theta\,\hat{\mathbf e}_\phi.
\end{aligned}
\]

De volumen:
\[
dV = r^{2}\sin\theta\,dr\,d\theta\,d\phi
\]

\subsection*{Ejemplo:}
\noindent
Una esfera de radio $r$ tiene una densidad superficial dada como:
\[
\sigma = x^{2} + y^{2} + z^{2}
\]
Determina su carga total:

\[
Q_T = \int_{A} \sigma\, dA
\]
donde:

\[
dA = r^{2}\sin\theta\, d\theta\, d\phi
\]

\[
Q_T = \int_{0}^{2\pi}\int_{0}^{\pi} r^{2}\sin\theta\, d\theta\, d\phi = 4\pi r^{4}
\]
\subsection*{Ejemplo:}
\[
V = \int_{V} dV = \int_{0}^{2\pi}\int_{0}^{\pi}\int_{0}^{r} r^{2}\sin\theta\, d\phi\, d\theta\, dr
\]\\
Una esfera de radio $R$ tiene una densidad de carga volumétrica de 
\[
\rho = e^{(x^{2}+y^{2}+z^{2})^{1/2}} = e^{r}
\]
Determina la carga total:

\[
Q_T = \int_{V} \rho\, dV = \int_{0}^{2\pi}\int_{0}^{\pi}\int_{0}^{R} e^{r} r^{2}\sin\theta\, dr\, d\theta\, d\phi
\]\\
\[
Q_T = 4\pi \int_{0}^{R} e^{r} r^{2}\, dr = \frac{4\pi}{3}\left(e^{R} - 1\right)
\]
\newpage
\noindent
\subsection*{Ejemplo:}
\noindent 
Determina el campo eléctrico que genera una carga puntual en cualquier punto del espacio.

\textbf{a) Con coordenadas cartesianas}
\[
\vec E \;=\; \frac{1}{4\pi\varepsilon_0}\,\frac{q}{r_q^{\,2}}\;\hat r_q
\]

Calculamos el vector unitario $\hat r_q$:
\[
\hat r_q \;=\; \frac{\vec r_P-\vec r_q}{\lvert \vec r_P-\vec r_q\rvert}
= \frac{x\,\hat{\imath}+y\,\hat{\jmath}+z\,\hat{\mathbf k}-0}{\sqrt{x^{2}+y^{2}+z^{2}}}
\]

La añadimos a la expresión del campo:
\[
\vec E \;=\; \frac{1}{4\pi\varepsilon_0}\;
\frac{q}{x^{2}+y^{2}+z^{2}}\;
\frac{x\,\hat{\imath}+y\,\hat{\jmath}+z\,\hat{\mathbf k}}{\sqrt{x^{2}+y^{2}+z^{2}}}
= \frac{1}{4\pi\varepsilon_0}\;
\frac{q\,(x\,\hat{\imath}+y\,\hat{\jmath}+z\,\hat{\mathbf k})}{(x^{2}+y^{2}+z^{2})^{3/2}}.
\]\\

\textbf{b) Con coordenadas cilíndricas}
\[
\vec E \;=\; \frac{1}{4\pi\varepsilon_0}\,\frac{q}{r_q^{\,2}}\;\hat r_q
\]

Calculamos el vector unitario $\hat r_q$:
\[
\hat r_q \;=\; \frac{\vec r_P-\vec r_q}{\lvert \vec r_P-\vec r_q\rvert}
= \frac{\rho\,\hat{\mathbf e}_\rho + z\,\hat{\mathbf k}-0}{\sqrt{\rho^{2}+z^{2}}}
\]

La añadimos a la expresión del campo:
\[
\vec E \;=\; \frac{1}{4\pi\varepsilon_0}\;
\frac{q}{\rho^{2}+z^{2}}\;
\frac{\rho\,\hat{\mathbf e}_\rho + z\,\hat{\mathbf k}}{\sqrt{\rho^{2}+z^{2}}}
= \frac{1}{4\pi\varepsilon_0}\;
\frac{q\left(\rho\,\hat{\mathbf e}_\rho + z\,\hat{\mathbf k}\right)}{(\rho^{2}+z^{2})^{3/2}}.
\]\\

\textbf{c) Con coordenadas esféricas}
\[
\vec E \;=\; \frac{1}{4\pi\varepsilon_0}\,\frac{q}{r_q^{\,2}}\;\hat r_q
\]

Calculamos el vector unitario $\hat r_q$:
\[
\hat r_q \;=\; \frac{\vec r_P-\vec r_q}{\lvert \vec r_P-\vec r_q\rvert}
= \frac{r\,\hat{\mathbf e}_r - 0}{r} \;=\; \hat{\mathbf e}_r
\]

La añadimos a la expresión del campo:
\[
\vec E \;=\; \frac{1}{4\pi\varepsilon_0}\,\frac{q}{r^{2}}\;\hat{\mathbf e}_r .
\]
\newpage
\noindent
\textbf{Ejercicio 2:} Un cilindro de radio \(R\) tiene su superficie circular cargada con una densidad de carga constante.
Determina el campo eléctrico generado a una distancia \(d\) de su eje como muestra la figura.

\medskip
\noindent
El campo eléctrico generado por el cilindro vendrá dado como:
\[
\vec E \;=\; \frac{1}{4\pi\varepsilon_0}\iint \frac{dq}{r^{2}}\;\hat r_{qP}.
\]
Con elemento de carga superficial
\[
dq=\sigma\,dA=\sigma\,R\,d\theta\,dz;
\qquad
\theta\in[0,2\pi],\;\; z\in[0,h].
\]
Calculamos el vector
\[
\vec r_{qP}=\vec r_P-\vec r_q
= (h+d)\,\hat{\mathbf k} - \big(R\,\hat{\mathbf e}_r + z\,\hat{\mathbf k}\big),
\]
\[
|\vec r_{qP}|=\sqrt{R^{2}+(h+d-z)^{2}}.
\]
Sustituyendo:
\[
\vec E
=\frac{\sigma R}{4\pi\varepsilon_0}
\int_{0}^{h}\!\!\int_{0}^{2\pi}
\left[
\frac{-\,R\,\hat{\mathbf e}_r}{\big(R^{2}+(h+d-z)^{2}\big)^{3/2}}
+\frac{(h+d-z)\,\hat{\mathbf k}}{\big(R^{2}+(h+d-z)^{2}\big)^{3/2}}
\right] d\theta\,dz .
\]

\textit{Nota:} hay que cambiar \(\hat{\mathbf e}_r\) por \(\cos\theta\,\hat{\imath}+\sin\theta\,\hat{\jmath}\) ya que \(\hat{\mathbf e}_r\) depende de \(\theta\).

\bigskip
\noindent
\textbf{Ejercicio 3.}
Un recipiente hemisférico de radio \(a\) tiene una carga total \(q\) repartida uniformemente en su superficie.
Encuentre el campo eléctrico en el centro de curvatura (véase la figura).

\medskip
\noindent
El campo eléctrico viene dado como
\[
\vec E \;=\; \frac{1}{4\pi\varepsilon_0}\,\int \frac{dq}{r^{2}}\;\hat r .
\]
\textbf{Sustituimos.} Como por simetría sólo queda la componente sobre el eje,
\[
E_z \;=\; \frac{1}{4\pi\varepsilon_0}
\int_{0}^{2\pi}\!\!\int_{0}^{\pi/2}
\frac{dq}{a^{2}}\;\cos\theta ,
\quad
dq=\sigma\,dA,\quad
dA=a^{2}\sin\theta\,d\theta\,d\varphi,
\quad
\sigma=\frac{q}{2\pi a^{2}} .
\]
Por tanto:
\[
E_z \;=\; \frac{1}{4\pi\varepsilon_0}
\int_{0}^{2\pi}\!\!\int_{0}^{\pi/2}
\frac{\sigma a^{2}\sin\theta\,d\theta\,d\varphi}{a^{2}}\;\cos\theta .
\]
Además,
\[
\hat r_{qP}\;=\;\frac{\vec r_{P}-\vec r_{q}}{\lvert \vec r_{P}-\vec r_{q}\rvert},
\qquad
\lvert \vec r_{P}-\vec r_{q}\rvert = a .
\]

\bigskip
\subsection{Ley de Gauss}
\noindent
La ley de Gauss en electrostática relaciona el flujo de campo eléctrico a través de una
superficie cerrada con la carga neta encerrada en ella, afirmando que el flujo es igual
a la carga neta dividida por la permitividad del vacío. Matemáticamente, se expresa como:
\[
\oint_{\partial V}\vec E\cdot d\vec A \;=\; \frac{q_{\text{enc}}}{\varepsilon_0}.
\]\\
La ley de Gauss proviene de la primera ley de Maxwell; para llegar a ella partimos del
\emph{teorema de la divergencia}:
\[
\oint_{\partial V}\vec E\cdot d\vec A
\;=\;
\iiint_{V} (\nabla\!\cdot\!\vec E)\,dV,
\]
\[
\frac{q_{\text{enc}}}{\varepsilon_0}
=
\iiint_{V} \frac{\rho}{\varepsilon_0}\,dV
\]
\[
\nabla\!\cdot\!\vec E \;=\; \frac{\rho}{\varepsilon_0}
\quad\text{(primera ley de Maxwell en el vacío).}
\]
Si hay divergencia en el campo, es decir, \(\nabla\!\cdot\!\vec E>0\), el campo \emph{diverge} (fuente).
Si por el contrario \(\nabla\!\cdot\!\vec E<0\), tenemos un \emph{sumidero}, es decir, el campo \emph{converge}.\\

\noindent
Para usar la \textbf{ley de Gauss} tengo que crear una \textbf{superficie imaginaria} a mi conveniencia para poder integrar sobre ella.
\subsection*{Ejemplo:}
\noindent 
Se tiene una esfera aislante con una distribución de carga homogénea repartida uniformemente en su volumen.  
La esfera tiene radio \(R\). Determina el campo eléctrico producido por la esfera en todo el espacio.

\medskip
\noindent
Primero calculamos el campo \textbf{fuera de la esfera} (\(r > R\)).  
Para ello, mi \textbf{superficie gaussiana} será una esfera concéntrica de radio \(r\).

\[
\oint_{\partial V}\vec E\cdot d\vec A = \frac{q_{\text{enc}}}{\varepsilon_0}
\]

\[
\oint_{\partial V}\vec E\cdot d\vec A = \int_0^{2\pi}\int_0^{\pi} |\vec E|\, r^{2}\sin\theta\, d\theta\, d\phi = |\vec E|\,r^{2}\,4\pi
\]

\[
\frac{q_{\text{enc}}}{\varepsilon_0} = \frac{1}{\varepsilon_0}\iiint \rho\, dV = \frac{1}{\varepsilon_0}\int_0^{2\pi}\int_0^{\pi}\int_0^{R} \rho\, r^{2}\sin\theta\, dr\, d\theta\, d\phi = \frac{4\pi}{3\varepsilon_0}\rho R^{3}
\]
Igualando ambas ecuaciones:
\[
|\vec E|\,4\pi r^{2} = \frac{4\pi}{3\varepsilon_0}\rho R^{3}
\]

\[
|\vec E| = \frac{\rho R^{3}}{3\varepsilon_0 r^{2}}
\]
Por lo tanto, el campo eléctrico para \(r > R\) es:
\[
\boxed{\vec E = \frac{\rho R^{3}}{3\varepsilon_0 r^{2}}\,\hat r }
\]

\bigskip
\noindent
Veamos lo que pasa ahora para \(r < R\) (\textbf{dentro de la esfera}).  
En este caso, la superficie gaussiana será una esfera concéntrica menor que la esfera original.

\[
\oint_{\partial V}\vec E\cdot d\vec A = \frac{q_{\text{enc}}}{\varepsilon_0}
\]

\[
\oint_{\partial V}\vec E\cdot d\vec A = \int_0^{2\pi}\int_0^{\pi} |\vec E|\, r^{2}\sin\theta\, d\theta\, d\phi = |\vec E|\,r^{2}\,4\pi
\]

\[
\frac{q_{\text{enc}}}{\varepsilon_0} = \frac{1}{\varepsilon_0}\iiint \rho\, dV = \frac{1}{\varepsilon_0}\int_0^{2\pi}\int_0^{\pi}\int_0^{r} \rho\, r^{2}\sin\theta\, dr\, d\theta\, d\phi = \frac{4\pi}{3\varepsilon_0}\rho r^{3}
\]
Igualando ambas ecuaciones:
\[
|\vec E|\,4\pi r^{2} = \frac{4\pi}{3\varepsilon_0}\rho r^{3}
\]

\[
\boxed{\vec E = \frac{\rho r}{3\varepsilon_0}\,\hat r }
\]

\bigskip
\noindent
Resultado final:
\[
\vec E(r) =
\begin{cases}
\dfrac{\rho r}{3\varepsilon_0}\,\hat r, & r < R \\
\dfrac{\rho R^{3}}{3\varepsilon_0 r^{2}}\,\hat r, & r > R
\end{cases}
\]
\newpage
\noindent
Ahora sustituimos en la ley de Gauss:
\[
|\vec E|\,4\pi r^{2} \;=\; \frac{1}{\varepsilon_{0}}\;\frac{4\pi}{3}\,\rho\,r^{3}
\]
\[
|\vec E| \;=\; \frac{\rho\,r}{3\,\varepsilon_{0}}
\]
Por tanto, en forma vectorial quedaría:
\[
\vec E \;=\; \frac{\rho\,r}{3\,\varepsilon_{0}}\;\hat r
\]
Entonces el campo eléctrico será:
\[
\vec E \;=\; \frac{\rho\,r}{3\,\varepsilon_{0}}\;\hat r \quad \text{si } r<R,
\qquad
\vec E \;=\; \frac{\rho\,R^{3}}{3\,\varepsilon_{0}\,r^{2}}\;\hat r \quad \text{si } r>R.
\]
\subsection*{Ejemplo:}
\noindent
Tenemos una línea de carga de densidad lineal \(\lambda\) y longitud infinita.
Determine el campo eléctrico producido en un punto a una distancia \(r\) del cable. Para conocer el campo eléctrico aplicaremos la \textbf{ley de Gauss} sobre la superficie gaussiana mostrada en la figura:
\[
\Phi_{E} \;=\; \iint_{\partial V} \vec E\cdot d\vec A
\;=\; \frac{q_{\text{enc}}}{\varepsilon_{0}}.
\]
% Línea infinita con carga lineal λ: flujo sobre un cilindro de Gauss
\[
\iint_{S}\vec E\!\cdot d\vec A
=\iint_{\text{tapa sup}}\vec E\!\cdot d\vec A
+\iint_{\text{tapa inf}}\vec E\!\cdot d\vec A
+\iint_{\text{pared}}\vec E\!\cdot d\vec A .
\]
Sabemos que \(\vec E=|\!E|\;\hat{\mathbf e}_\rho\) (radial) y es tangencial a las tapas \(\Rightarrow\)
las contribuciones de las tapas son nulas; sólo aporta la pared cilíndrica:
\[
\Phi_E=\iint_{\text{pared}} \vec E\!\cdot d\vec A
=\int_{0}^{h}\!\!\int_{0}^{2\pi} |\!E|\,(\rho\,d\theta)\,dz
=|\!E|\,(2\pi\rho h).
\]
Por otro lado,
\[
\frac{q_{\text{enc}}}{\varepsilon_0}
=\frac{1}{\varepsilon_0}\int \lambda\,dl
=\frac{1}{\varepsilon_0}\int_{0}^{h}\lambda\,dz
=\frac{\lambda h}{\varepsilon_0}.
\]
Igualando términos:
\[
|\!E|\,(2\pi\rho h)=\frac{\lambda h}{\varepsilon_0}
\;\;\Longrightarrow\;\;
|\!E|=\frac{\lambda}{2\pi\varepsilon_0\,\rho},
\qquad
\boxed{\;\vec E=\dfrac{\lambda}{2\pi\varepsilon_0\,\rho}\,\hat{\mathbf e}_\rho\;}
\]

% Plano infinito con densidad superficial σ: “pastilla” gaussiana
\subsection*{Ejemplo:}
\noindent
Caso de un plano cargado con densidad superficial constante \(\sigma\). Determinar el campo eléctrico a una distancia \(d\) del plano. Aplicamos la ley de Gauss con una superficie cilíndrica (“pastilla”) que corta el plano:
\[
\Phi_E=\iint_{S}\vec E\!\cdot d\vec A=\frac{q_{\text{enc}}}{\varepsilon_0},
\qquad
d\vec A=\hat{\mathbf n}\,dA,\quad \vec E=|\!E|\,\hat{\mathbf n}.
\]
La contribución de la pared lateral es nula (\(\vec E\perp d\vec A\) allí). Aportan sólo las tapas:
\[
\Phi_E=|\!E|A_{\text{sup}}+|\!E|A_{\text{inf}}
=2|\!E|\,(\pi R^{2}).
\]
La carga encerrada es
\[
\frac{q_{\text{enc}}}{\varepsilon_0}
=\frac{1}{\varepsilon_0}\iint \sigma\,dA
=\frac{\sigma}{\varepsilon_0}\,\pi R^{2}.
\]
Por tanto,
\[
2|\!E|\,\pi R^{2}=\frac{\sigma}{\varepsilon_0}\,\pi R^{2}
\;\;\Longrightarrow\;\;
\boxed{\;|\!E|=\dfrac{\sigma}{2\varepsilon_0}\;}
\quad\text{(constante, independiente de \(d\)).}
\]
\end{document}