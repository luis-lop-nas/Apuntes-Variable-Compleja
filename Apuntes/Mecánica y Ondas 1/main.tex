\documentclass[a4paper,12pt]{article} 

% --- Idioma y codificación ---
\usepackage[utf8]{inputenc}
\usepackage[T1]{fontenc}
\usepackage[spanish, es-tabla, shorthands=off]{babel} % <-- clave para que TikZ no falle
\usepackage{lmodern}

% --- Matemáticas ---
\usepackage{amsmath}
\usepackage{amssymb}
\usepackage{amsthm}
\usepackage{mathtools}
\usepackage{bm}
\usepackage{esint}

% --- Utilidades ---
\usepackage{graphicx}
\usepackage{xcolor}
\usepackage{enumitem}
\usepackage{titlesec}
\usepackage[hidelinks]{hyperref}
\usepackage{caption}

% --- Esquemas y Dibujos ---
\usepackage{tikz}
\usetikzlibrary{arrows.meta,calc}

% Formato de subtítulos
\titleformat{\subsection}{\normalfont\small\bfseries}{\thesubsection}{1em}{}

\title{Apuntes Mecánica y Ondas 1}
\author{Luis López}
\date{Septiembre 2025}

\begin{document}

\maketitle
\newpage
\tableofcontents
\newpage

\section*{Introducción}

En esta asignatura abordaremos los fundamentos de la \textbf{Mecánica clásica} y la teoría de \textbf{Ondas}, organizados en seis bloques principales. Cada uno introduce nuevas herramientas conceptuales y matemáticas que permiten analizar sistemas físicos de manera más general y profunda:

\begin{itemize}
    \item \textbf{Sistemas de referencia no inerciales.} 
    Estudiaremos qué ocurre cuando analizamos el movimiento desde un sistema que no es inercial, es decir, que está acelerado linealmente o en rotación respecto a un sistema de referencia inercial. En estos casos aparecen las llamadas \emph{fuerzas ficticias} (como la fuerza centrífuga, la de Coriolis o la fuerza de inercia), necesarias para explicar el movimiento observado desde dichos marcos. Veremos cómo se formulan y en qué fenómenos cotidianos aparecen, como en el movimiento de proyectiles, giroscopios o el comportamiento de masas en la Tierra debido a su rotación.

    \item \textbf{Mecánica Lagrangiana.} 
    Introduciremos el \emph{principio de mínima acción}, que reformula la mecánica clásica en términos de un funcional que debe extremarse. A partir de este principio surge el formalismo lagrangiano, donde el \emph{lagrangiano} se define como \(L = T - V\) (energía cinética menos energía potencial). De aquí derivan las ecuaciones de \emph{Euler-Lagrange}, que permiten describir sistemas de partículas y sistemas con restricciones de una manera mucho más general y elegante que las leyes de Newton.

    \item \textbf{Mecánica Hamiltoniana.} 
    Veremos cómo a partir del formalismo lagrangiano se puede construir el \emph{hamiltoniano}, que representa la energía total del sistema en función de coordenadas y momentos generalizados. A través del \emph{principio de Hamilton}, obtendremos las \emph{ecuaciones de Hamilton}, que constituyen un nuevo marco para describir sistemas dinámicos. Este formalismo no solo simplifica algunos problemas de la mecánica clásica, sino que es también la base conceptual de la mecánica estadística y la mecánica cuántica.

    \item \textbf{Oscilaciones de un grado de libertad.} 
    Estudiaremos primero el \emph{oscilador armónico simple}, uno de los sistemas más importantes de la física, pues modela fenómenos tan diversos como vibraciones, circuitos eléctricos o moléculas. Posteriormente veremos el \emph{oscilador amortiguado}, en el que aparece la disipación de energía, y el \emph{oscilador forzado}, donde se introduce una fuerza periódica externa. El análisis de estos sistemas nos llevará a comprender el fenómeno de la \emph{resonancia}, fundamental en física e ingeniería.

    \item \textbf{Sistemas de partículas.} 
    Abordaremos el estudio de conjuntos de partículas, introduciendo el concepto de \emph{centro de masas} y su movimiento como equivalente al de una partícula ficticia que concentra toda la masa. Además, se estudiarán las leyes de conservación del \emph{momento lineal} y del \emph{momento angular}, aplicadas a fenómenos como colisiones elásticas e inelásticas. Estos principios permiten explicar una gran variedad de problemas físicos de forma compacta.

    \item \textbf{Sólido rígido.} 
    Finalmente, se analizará la dinámica de los cuerpos rígidos, es decir, aquellos que no cambian de forma bajo la acción de fuerzas. Estudiaremos la energía cinética de rotación, el \emph{momento angular} de un sólido rígido y la importancia del \emph{tensor de inercia}, que generaliza el concepto de inercia en sistemas tridimensionales. Se verán además las ecuaciones de movimiento para la rotación de cuerpos en torno a ejes fijos o móviles, fundamentales para comprender desde el giro de una peonza hasta la estabilidad de satélites y planetas.
\end{itemize}

Este recorrido proporcionará una visión amplia y profunda de la mecánica clásica, preparando las bases para comprender fenómenos más complejos y para conectar con otras ramas de la física moderna.

\newpage

\section{Sistemas de referencia no inerciales}

En mecánica newtoniana, un \textbf{ sistema de referencia no inercial} es aquel en el que las leyes del movimiento de Newton no se cumplen si solo se consideran las fuerzas reales.  
Un sistema de referencia es no inercial siempre que presente un \emph{movimiento acelerado} respecto a un sistema de referencia inercial.  

La aceleración de un sistema no inercial puede deberse a:
\begin{itemize}
    \item Un cambio en el módulo de su velocidad de traslación (aceleración lineal).
    \item Un cambio en la dirección de su velocidad de traslación (por ejemplo, en un movimiento de giro respecto a un sistema inercial).
    \item Un movimiento de rotación sobre sí mismo.
    \item Una combinación de los casos anteriores.
\end{itemize}

Un ejemplo típico de sistema no inercial es el sistema de coordenadas fijo a la \textbf{Tierra}: los cuerpos son medidos respecto a puntos que, en realidad, se mueven debido a la rotación terrestre.  

En estos marcos de referencia, el observador debe introducir las llamadas \textbf{fuerzas ficticias} (como la fuerza centrífuga o la de Coriolis) para explicar el movimiento según las leyes de Newton. Estas fuerzas no son interacciones reales, pero resultan necesarias para mantener la validez de las ecuaciones de Newton dentro del sistema no inercial.

Por tanto, un sistema de referencia no inercial puede identificarse precisamente porque en él se violan las leyes de Newton. Un ejemplo ilustrativo es el \textbf{péndulo de Foucault}, cuyo plano de oscilación varía debido a la rotación terrestre.

Finalmente, cabe señalar que, siendo estrictos, los sistemas de referencia inerciales \emph{no existen de forma absoluta}, pues incluso la Tierra se mueve alrededor del Sol y este alrededor del centro galáctico. Sin embargo, en la práctica, se consideran como inerciales aquellos sistemas en los que el error cometido es despreciable; por ejemplo, para la mayoría de problemas cotidianos resulta útil considerar la superficie terrestre como un sistema inercial.

\newpage

\subsection{Sistemas no inerciales}

\begin{figure}[h]
\centering
% --- Dibujo a la izquierda ---
\begin{minipage}{0.42\textwidth}
\centering
\begin{tikzpicture}[scale=1.2, every node/.style={inner sep=1pt}]
  % Ejes sistema S
  \draw[->] (0,0) -- (3.2,0) node[below] {$x$};
  \draw[->] (0,0) -- (0,2.1) node[left] {$y$};
  \node[below left] at (0,0) {$S$};

  % Origen desplazado de S' (más a la derecha y arriba)
  \coordinate (Op) at (3.5,1.0);

  % Ejes sistema S'
  \draw[->] (Op) -- ++(2.3,0) node[right] {$x'$};
  \draw[->] (Op) -- ++(0,1.6) node[above] {$y'$};
  \node[below=4pt] at (Op) {$S'$};

  % Vector R (S -> S')
  \draw[->, thick, violet] (0,0) -- (Op) node[pos=0.55, below] {$\vec{R}$};

  % Punto final en S' para r'
  \coordinate (P) at ($(Op)+(1.3,0.7)$);

  % Vector r' (en S')
  \draw[->, thick] (Op) -- (P) node[pos=0.55, below right] {$\vec{r}'$};

  % Vector r (en S)
  \draw[->, thick] (0,0) -- (P) node[pos=0.55, above left] {$\vec{r}$};
\end{tikzpicture}
\end{minipage}
\hspace{4cm} % <-- separación entre dibujo y fórmulas
% --- Fórmulas a la derecha ---
\begin{minipage}{0.28\textwidth}
\[
\vec{r} = \vec{R} + \vec{r}'
\]
\[
\vec{v} = \vec{V} + \vec{v}'
\]
\end{minipage}

\caption{Relaciones vectoriales en sistemas no inerciales.}
\end{figure}

\subsection{Sistemas inerciales}

En un \textbf{sistema de referencia no inercial}, las leyes de Newton no se cumplen directamente, por lo que es necesario introducir fuerzas ficticias para describir correctamente el movimiento.  
En este marco, la posición y velocidad de una partícula se relacionan con las medidas desde un sistema inercial mediante las siguientes expresiones:

\begin{figure}[h]
\centering
% --- Dibujo a la izquierda ---
\begin{minipage}{0.42\textwidth}
\centering
\begin{tikzpicture}[scale=1.2, every node/.style={inner sep=1pt}]
  % Ejes sistema S
  \draw[->] (0,0) -- (3.2,0) node[below] {$x$};
  \draw[->] (0,0) -- (0,2.1) node[left] {$y$};
  \node[below left] at (0,0) {$S$};

  % Origen desplazado de S'
  \coordinate (Op) at (3.5,1.0);

  % Ejes sistema S'
  \draw[->] (Op) -- ++(2.3,0) node[right] {$x'$};
  \draw[->] (Op) -- ++(0,1.6) node[above] {$y'$};
  \node[below=4pt] at (Op) {$S'$};

  % Vector R (S -> S')
  \draw[->, thick, violet] (0,0) -- (Op) node[pos=0.55, below] {$\vec{R}$};

  % Punto final en S' para r'
  \coordinate (P) at ($(Op)+(1.3,0.7)$);

  % Vector r' (en S')
  \draw[->, thick] (Op) -- (P) node[pos=0.55, below right] {$\vec{r}'$};

  % Vector r (en S)
  \draw[->, thick] (0,0) -- (P) node[pos=0.55, above left] {$\vec{r}$};
\end{tikzpicture}
\end{minipage}
\hspace{4cm} % separación entre dibujo y fórmulas
\end{figure}

Son aquellos en los que la \textbf{velocidad relativa es constante}:
\[
\vec{v} = \vec{V} + \vec{v}'.
\]

En el caso particular en que \(\vec{V}=0\), se cumple que:
\[
\vec{r} = \vec{r}', \quad \vec{v} = \vec{v}'.
\]

Además, derivando con respecto al tiempo:
\[
\vec{v} = \frac{d\vec{r}}{dt} = \frac{d\vec{R}}{dt} + \frac{d\vec{r}'}{dt}
\]
\[
\vec{a} = \frac{d\vec{v}}{dt} = \frac{d\vec{V}}{dt} + \frac{d\vec{v}'}{dt}.
\]

De este modo, la segunda ley de Newton se transforma en:
\[
m \vec{a} = \vec{F} \quad \Rightarrow \quad m \vec{a}' = \vec{F}.
\]


\subsection{Sistemas en rotación}

Para un vector unitario:
\[
\vec{u}_1 = \cos\theta\, \vec{i} - \sin\theta\, \vec{j}, 
\qquad 
\vec{u}_2 = \sin\theta\, \vec{i} + \cos\theta\, \vec{j}
\]

\[
\frac{d\vec{u}_1}{dt} = -\dot{\theta}\sin\theta\,\vec{i} - \dot{\theta}\cos\theta\,\vec{j} 
= -\dot{\theta}\vec{u}_2
\]

\[
\frac{d\vec{u}_2}{dt} = \dot{\theta}\cos\theta\,\vec{i} - \dot{\theta}\sin\theta\,\vec{j} 
= \dot{\theta}\vec{u}_1
\]

\[
\omega = \frac{d\theta(t)}{dt}
\]

\[
\frac{d\vec{u}_1}{dt} = -\omega \vec{u}_2, 
\qquad 
\frac{d\vec{u}_2}{dt} = \omega \vec{u}_1
\]

---

\[
\frac{d\vec{A}}{dt}\Big|_{S} = \frac{d\vec{A}}{dt}\Big|_{S'} + \boldsymbol{\omega} \times \vec{A}
\]

\[
\vec{v} = \frac{d\vec{r}}{dt} = \vec{v}' + \boldsymbol{\omega}\times \vec{r}
\]

\[
\vec{v} = \vec{v}' + \boldsymbol{\omega}\times\vec{r}
\]

\[
\vec{v} = \vec{v}' + \boldsymbol{\omega}\times\vec{r}' 
\]

\[
\text{Lo que nos deja } \quad \vec{v} = \vec{v}' + \boldsymbol{\omega}\times \vec{r}'
\]

---

Para la aceleración:

\[
\vec{a} = \frac{d\vec{v}}{dt}\Big|_S = \frac{d\vec{v}'}{dt}\Big|_{S} + \frac{d}{dt}(\boldsymbol{\omega}\times \vec{r}')\Big|_{S}
\]

\[
= \frac{d\vec{v}'}{dt}\Big|_{S'} + \boldsymbol{\omega}\times \vec{v}' + 
\dot{\boldsymbol{\omega}}\times \vec{r}' + \boldsymbol{\omega}\times \vec{v}' 
\]

\[
= \vec{a}' + \dot{\boldsymbol{\omega}}\times \vec{r}' + 2\boldsymbol{\omega}\times \vec{v}' 
+ \boldsymbol{\omega}\times (\boldsymbol{\omega}\times \vec{r}')
\]

\[
\vec{a} = \vec{a}' + 2\boldsymbol{\omega}\times \vec{v}' + \dot{\boldsymbol{\omega}}\times \vec{r}' + \boldsymbol{\omega}\times (\boldsymbol{\omega}\times \vec{r}')
\]

\[
\text{Teorema de Coriolis}
\]

---

Para las fuerzas:

\[
m\vec{a}' = \vec{F} - m\dot{\boldsymbol{\omega}}\times \vec{r}' - m\big(\boldsymbol{\omega}\times (\boldsymbol{\omega}\times \vec{r}')\big) - 2m(\boldsymbol{\omega}\times \vec{v}')
\]

Donde:

\[
\vec{F}_{\text{Euler}} = -m\dot{\boldsymbol{\omega}}\times \vec{r}', 
\quad \vec{F}_{\text{centrífuga}} = -m(\boldsymbol{\omega}\times (\boldsymbol{\omega}\times \vec{r}')), 
\quad \vec{F}_{\text{Coriolis}} = -2m(\boldsymbol{\omega}\times \vec{v}')
\]

\[
\vec{F}' = \vec{F} + \vec{F}_i
\]

---

\subsection{La Tierra como sistema no inercial}

\[
R_T = 6370\, \text{km}, 
\qquad 
\omega = \frac{2\pi}{24h} = 7.3\cdot 10^{-5}\,\text{rad/s}
\]

\[
\vec{F}_{\text{real}} = m\vec{g}
\]

Estudiamos la fuerza centrífuga:

\[
\vec{F}_{\text{cen}} = -m(\boldsymbol{\omega}\times (\boldsymbol{\omega}\times \vec{R}))
\]

\[
= -m\big((\boldsymbol{\omega}\cdot \vec{R})\boldsymbol{\omega} - (\boldsymbol{\omega}\cdot \boldsymbol{\omega})\vec{R}\big)
\]

\[
= m\omega^2 \vec{R}_\perp
\]

\[
\text{Se desprecia ya que } |\vec{F}_{\text{cen}}|\ll |\vec{R}|
\]
\end{document}