\documentclass[11pt,a4paper]{article}
\usepackage[margin=2.2cm]{geometry}
\usepackage{parskip}
\usepackage{enumitem}
\usepackage{titlesec}
\usepackage[hidelinks]{hyperref}
\usepackage{microtype}
\usepackage{amssymb}
\setlist[itemize]{topsep=2pt,itemsep=2pt}
\setlist[enumerate]{topsep=2pt,itemsep=2pt}
\titleformat{\section}{\large\bfseries}{}{0pt}{}
\titleformat{\subsection}{\bfseries}{}{0pt}{}
\title{English Tenses Notes: Present, Past, and Future}
\author{}
\date{}

\begin{document}
\maketitle

\section*{How to Use These Notes}
For each tense: \textbf{When it’s used} $\to$ \textbf{Form (Affirmative / Negative / Interrogative)} $\to$ \textbf{Examples} $\to$ \textbf{Exercises (10 items)}. Fill in the blanks with the correct verb form.

%========================
\section{Present Simple}
\subsection*{When it’s used}
\begin{itemize}
  \item Habits and routines: ``I get up at 7.''
  \item General truths and facts: ``Water boils at 100°C.''
  \item Timetables/scheduled events (future reference): ``The train leaves at 6.''
  \item States, feelings, opinions (non–action verbs): ``She likes coffee.''
\end{itemize}

\subsection*{Form}
\textbf{Affirmative}: \emph{Subject + base verb} (add \emph{-s/-es} with he/she/it).\\
\textbf{Negative}: \emph{Subject + do not (don't) / does not (doesn't) + base verb}.\\
\textbf{Interrogative}: \emph{Do/Does + subject + base verb?}

\subsection*{Examples}
\begin{itemize}
  \item A: \emph{She works} in marketing.
  \item N: \emph{They don't watch} TV on weekdays.
  \item Q: \emph{Does he play} tennis on Sundays?
\end{itemize}

\subsection*{Exercises}
Complete with the Present Simple.
\begin{enumerate}
  \item My brother \underline{\hspace{2.5cm}} (live) near the stadium.
  \item We \underline{\hspace{2.5cm}} (study) English every day.
  \item \underline{\hspace{2.5cm}} (your father / drive) to work?
  \item She \underline{\hspace{2.5cm}} (not like) spicy food.
  \item The museum \underline{\hspace{2.5cm}} (open) at 10 a.m.
  \item Where \underline{\hspace{2.5cm}} (you / come) from?
  \item He \underline{\hspace{2.5cm}} (watch) the news at night.
  \item They \underline{\hspace{2.5cm}} (not believe) that story.
  \item \underline{\hspace{2.5cm}} (it / rain) a lot here in spring?
  \item Anna \underline{\hspace{2.5cm}} (teach) math at a high school.
\end{enumerate}

%========================
\section{Present Continuous (Progressive)}
\subsection*{When it’s used}
\begin{itemize}
  \item Actions happening now or around now: ``She is reading.''
  \item Temporary situations: ``I’m living with friends this month.''
  \item Changing/developing situations: ``The weather is getting colder.''
  \item Planned near-future arrangements: ``We are meeting at 6.''
\end{itemize}

\subsection*{Form}
\textbf{Affirmative}: \emph{Subject + am/is/are + verb-ing}.\\
\textbf{Negative}: \emph{Subject + am/is/are + not + verb-ing}.\\
\textbf{Interrogative}: \emph{Am/Is/Are + subject + verb-ing?}

\subsection*{Examples}
\begin{itemize}
  \item A: \emph{They are working} on a new app.
  \item N: \emph{I am not watching} TV right now.
  \item Q: \emph{Are you studying} for the test?
\end{itemize}

\subsection*{Exercises}
Complete with the Present Continuous.
\begin{enumerate}
  \item Look! The kids \underline{\hspace{2.5cm}} (play) outside.
  \item I \underline{\hspace{2.5cm}} (not work) this week.
  \item \underline{\hspace{2.5cm}} (she / cook) dinner now?
  \item They \underline{\hspace{2.5cm}} (move) to a new flat this weekend.
  \item The company \underline{\hspace{2.5cm}} (grow) quickly.
  \item Why \underline{\hspace{2.5cm}} (you / laugh)?
  \item He \underline{\hspace{2.5cm}} (not use) his phone at the moment.
  \item We \underline{\hspace{2.5cm}} (plan) a surprise party.
  \item \underline{\hspace{2.5cm}} (it / snow) right now?
  \item I \underline{\hspace{2.5cm}} (read) an amazing book these days.
\end{enumerate}

%========================
\section{Present Perfect}
\subsection*{When it’s used}
\begin{itemize}
  \item Life experiences (time not specified): ``I have visited Japan.''
  \item Recent events with present results: ``She has broken her arm.''
  \item Unfinished time period: ``We have worked a lot this week.''
  \item Changes over time / achievements: ``He has improved a lot.''
\end{itemize}

\subsection*{Form}
\textbf{Affirmative}: \emph{Subject + have/has + past participle (V3)}.\\
\textbf{Negative}: \emph{Subject + have/has not (haven’t/hasn’t) + V3}.\\
\textbf{Interrogative}: \emph{Have/Has + subject + V3?}

\subsection*{Examples}
\begin{itemize}
  \item A: \emph{She has finished} her homework.
  \item N: \emph{They haven’t seen} that movie.
  \item Q: \emph{Have you ever tried} sushi?
\end{itemize}

\subsection*{Exercises}
Use the Present Perfect.
\begin{enumerate}
  \item I \underline{\hspace{2.5cm}} (not finish) the report yet.
  \item \underline{\hspace{2.5cm}} (you / ever / be) to Canada?
  \item She \underline{\hspace{2.5cm}} (lose) her keys.
  \item They \underline{\hspace{2.5cm}} (work) here since 2022.
  \item He \underline{\hspace{2.5cm}} (just / call) me.
  \item We \underline{\hspace{2.5cm}} (not decide) the date.
  \item \underline{\hspace{2.5cm}} (the train / leave)?
  \item My English \underline{\hspace{2.5cm}} (improve) a lot.
  \item She \underline{\hspace{2.5cm}} (visit) London three times.
  \item They \underline{\hspace{2.5cm}} (not see) each other lately.
\end{enumerate}

%========================
\section{Past Simple}
\subsection*{When it’s used}
\begin{itemize}
  \item Completed actions at a specific time in the past: ``He left in 2019.''
  \item Past habits and sequences: ``I walked home and cooked dinner.''
\end{itemize}

\subsection*{Form}
\textbf{Affirmative}: \emph{Subject + past form (V2)}.\\
\textbf{Negative}: \emph{Subject + did not (didn’t) + base verb}.\\
\textbf{Interrogative}: \emph{Did + subject + base verb?}

\subsection*{Examples}
\begin{itemize}
  \item A: \emph{We visited} Rome last year.
  \item N: \emph{She didn’t enjoy} the concert.
  \item Q: \emph{Did you finish} the task?
\end{itemize}

\subsection*{Exercises}
Use the Past Simple.
\begin{enumerate}
  \item They \underline{\hspace{2.5cm}} (arrive) late yesterday.
  \item I \underline{\hspace{2.5cm}} (not understand) the question.
  \item \underline{\hspace{2.5cm}} (you / see) the eclipse?
  \item She \underline{\hspace{2.5cm}} (buy) a new laptop.
  \item We \underline{\hspace{2.5cm}} (walk) along the river.
  \item He \underline{\hspace{2.5cm}} (not call) me last night.
  \item Where \underline{\hspace{2.5cm}} (they / go) on holiday?
  \item The meeting \underline{\hspace{2.5cm}} (start) at 9.
  \item I \underline{\hspace{2.5cm}} (lose) my wallet on the bus.
  \item She \underline{\hspace{2.5cm}} (study) French at school.
\end{enumerate}

%========================
\section{Past Continuous}
\subsection*{When it’s used}
\begin{itemize}
  \item Actions in progress at a specific past time: ``At 8 pm, I was studying.''
  \item Background action interrupted by a shorter action (Past Simple): ``I was cooking when he arrived.''
  \item Parallel past actions: ``She was reading while I was writing.''
\end{itemize}

\subsection*{Form}
\textbf{Affirmative}: \emph{Subject + was/were + verb-ing}.\\
\textbf{Negative}: \emph{Subject + was/were + not + verb-ing}.\\
\textbf{Interrogative}: \emph{Was/Were + subject + verb-ing?}

\subsection*{Examples}
\begin{itemize}
  \item A: \emph{They were driving} home at 7.
  \item N: \emph{I wasn’t listening} to the radio.
  \item Q: \emph{Were you waiting} for long?
\end{itemize}

\subsection*{Exercises}
Use the Past Continuous (and Past Simple where needed).
\begin{enumerate}
  \item I \underline{\hspace{2.5cm}} (watch) TV when the phone \underline{\hspace{2.5cm}} (ring).
  \item They \underline{\hspace{2.5cm}} (not sleep) at midnight.
  \item \underline{\hspace{2.5cm}} (she / drive) when it \underline{\hspace{2.5cm}} (start) to rain?
  \item We \underline{\hspace{2.5cm}} (have) dinner while our friends \underline{\hspace{2.5cm}} (talk).
  \item He \underline{\hspace{2.5cm}} (jog) in the park at 6 am.
  \item The kids \underline{\hspace{2.5cm}} (play) when the lights \underline{\hspace{2.5cm}} (go) out.
  \item I \underline{\hspace{2.5cm}} (not pay) attention.
  \item \underline{\hspace{2.5cm}} (they / argue) about the price?
  \item She \underline{\hspace{2.5cm}} (read) while I \underline{\hspace{2.5cm}} (cook).
  \item What \underline{\hspace{2.5cm}} (you / do) at 9 last night?
\end{enumerate}

%========================
\section{Past Perfect}
\subsection*{When it’s used}
\begin{itemize}
  \item The earlier of two past actions: ``He had left before I arrived.''
  \item Cause-and-effect in the past: ``She was tired because she had worked all day.''
\end{itemize}

\subsection*{Form}
\textbf{Affirmative}: \emph{Subject + had + past participle (V3)}.\\
\textbf{Negative}: \emph{Subject + had not (hadn’t) + V3}.\\
\textbf{Interrogative}: \emph{Had + subject + V3?}

\subsection*{Examples}
\begin{itemize}
  \item A: \emph{They had finished} the project before the deadline.
  \item N: \emph{I hadn’t seen} him before that day.
  \item Q: \emph{Had she studied} enough for the exam?
\end{itemize}

\subsection*{Exercises}
Use the Past Perfect (and Past Simple where needed).
\begin{enumerate}
  \item By 8 o’clock, we \underline{\hspace{2.5cm}} (finish) dinner.
  \item She was angry because he \underline{\hspace{2.5cm}} (forget) her birthday.
  \item \underline{\hspace{2.5cm}} (you / ever / try) sushi before that night?
  \item They \underline{\hspace{2.5cm}} (leave) when I arrived.
  \item He \underline{\hspace{2.5cm}} (not read) the book before the class.
  \item After we \underline{\hspace{2.5cm}} (pack), we called a taxi.
  \item I realized I \underline{\hspace{2.5cm}} (lose) my keys.
  \item \underline{\hspace{2.5cm}} (she / finish) the report by Monday?
  \item The match \underline{\hspace{2.5cm}} (already / start) when we got there.
  \item They were happy because they \underline{\hspace{2.5cm}} (win).
\end{enumerate}

%========================
\section{Future with \emph{will}}
\subsection*{When it’s used}
\begin{itemize}
  \item Instant decisions and offers: ``I’ll help you.''
  \item Predictions without present evidence: ``It’ll be sunny tomorrow.''
  \item Promises and future facts: ``I won’t tell anyone.''
\end{itemize}

\subsection*{Form}
\textbf{Affirmative}: \emph{Subject + will + base verb}.\\
\textbf{Negative}: \emph{Subject + will not (won’t) + base verb}.\\
\textbf{Interrogative}: \emph{Will + subject + base verb?}

\subsection*{Examples}
\begin{itemize}
  \item A: \emph{I will call} you later.
  \item N: \emph{She won’t agree} to that plan.
  \item Q: \emph{Will they arrive} on time?
\end{itemize}

\subsection*{Exercises}
Use \emph{will}.
\begin{enumerate}
  \item Don’t worry, I \underline{\hspace{2.5cm}} (send) the email now.
  \item \underline{\hspace{2.5cm}} (they / come) to the meeting?
  \item She \underline{\hspace{2.5cm}} (not accept) the offer.
  \item I think it \underline{\hspace{2.5cm}} (rain) tomorrow.
  \item We \underline{\hspace{2.5cm}} (finish) the task this afternoon.
  \item \underline{\hspace{2.5cm}} (you / help) me with this box?
  \item He \underline{\hspace{2.5cm}} (be) 30 next year.
  \item The team \underline{\hspace{2.5cm}} (not give up).
  \item What time \underline{\hspace{2.5cm}} (the show / start)?
  \item I promise I \underline{\hspace{2.5cm}} (call) you back.
\end{enumerate}

%========================
\section{Future with \emph{be going to}}
\subsection*{When it’s used}
\begin{itemize}
  \item Plans and intentions decided before speaking: ``We’re going to visit Peru.''
  \item Predictions based on present evidence: ``Look at those clouds! It’s going to rain.''
\end{itemize}

\subsection*{Form}
\textbf{Affirmative}: \emph{Subject + am/is/are + going to + base verb}.\\
\textbf{Negative}: \emph{Subject + am/is/are + not + going to + base verb}.\\
\textbf{Interrogative}: \emph{Am/Is/Are + subject + going to + base verb?}

\subsection*{Examples}
\begin{itemize}
  \item A: \emph{They are going to start} a podcast.
  \item N: \emph{I’m not going to buy} that phone.
  \item Q: \emph{Are you going to apply} for the job?
\end{itemize}

\subsection*{Exercises}
Use \emph{be going to}.
\begin{enumerate}
  \item We \underline{\hspace{2.5cm}} (visit) my grandparents this weekend.
  \item \underline{\hspace{2.5cm}} (she / study) abroad next year?
  \item He \underline{\hspace{2.5cm}} (not change) his plan.
  \item Look at the sky! It \underline{\hspace{2.5cm}} (storm).
  \item I \underline{\hspace{2.5cm}} (start) a new course in November.
  \item They \underline{\hspace{2.5cm}} (renovate) their kitchen soon.
  \item \underline{\hspace{2.5cm}} (you / tell) her the news?
  \item The company \underline{\hspace{2.5cm}} (launch) a new product.
  \item I’m tired, I \underline{\hspace{2.5cm}} (not cook) tonight.
  \item When \underline{\hspace{2.5cm}} (we / leave)?
\end{enumerate}

%========================
\section*{Answer Key (Suggested)}
\textit{Optional:} Add your own answer key here once students finish.
\newpage
%========================
\section{Presente Simple (Present Simple)}
\subsection*{Cuándo se usa}
\begin{itemize}
  \item Hábitos y rutinas: ``I get up at 7.'' (Me levanto a las 7.)
  \item Verdades generales: ``Water boils at 100°C.'' (El agua hierve a 100°C.)
  \item Horarios o hechos fijos: ``The train leaves at 6.'' (El tren sale a las 6.)
  \item Estados, gustos y opiniones: ``She likes coffee.'' (A ella le gusta el café.)
\end{itemize}

\subsection*{Cómo se forma}
\textbf{Afirmativa}: \emph{Sujeto + verbo base} (añadir \emph{-s/-es} con he/she/it).\\
\textbf{Negativa}: \emph{Sujeto + do not (don't) / does not (doesn't) + verbo base}.\\
\textbf{Interrogativa}: \emph{Do/Does + sujeto + verbo base?}

\subsection*{Ejemplos}
\begin{itemize}
  \item A: \emph{She works} in marketing.
  \item N: \emph{They don't watch} TV on weekdays.
  \item Q: \emph{Does he play} tennis on Sundays?
\end{itemize}

\subsection*{Ejercicios}
Completa con el \textbf{Presente Simple}.
\begin{enumerate}
  \item My brother \underline{\hspace{2.5cm}} (live) near the stadium.
  \item We \underline{\hspace{2.5cm}} (study) English every day.
  \item \underline{\hspace{2.5cm}} (your father / drive) to work?
  \item She \underline{\hspace{2.5cm}} (not like) spicy food.
  \item The museum \underline{\hspace{2.5cm}} (open) at 10 a.m.
  \item Where \underline{\hspace{2.5cm}} (you / come) from?
  \item He \underline{\hspace{2.5cm}} (watch) the news at night.
  \item They \underline{\hspace{2.5cm}} (not believe) that story.
  \item \underline{\hspace{2.5cm}} (it / rain) a lot here in spring?
  \item Anna \underline{\hspace{2.5cm}} (teach) math at a high school.
\end{enumerate}

%========================
\section{Presente Continuo (Present Continuous)}
\subsection*{Cuándo se usa}
\begin{itemize}
  \item Acciones que ocurren ahora o alrededor de ahora: ``She is reading.''
  \item Situaciones temporales: ``I’m living with friends this month.''
  \item Cambios o desarrollos: ``The weather is getting colder.''
  \item Planes cercanos: ``We are meeting at 6.''
\end{itemize}

\subsection*{Cómo se forma}
\textbf{Afirmativa}: \emph{Sujeto + am/is/are + verbo-ing}.\\
\textbf{Negativa}: \emph{Sujeto + am/is/are + not + verbo-ing}.\\
\textbf{Interrogativa}: \emph{Am/Is/Are + sujeto + verbo-ing?}

\subsection*{Ejemplos}
\begin{itemize}
  \item A: \emph{They are working} on a new app.
  \item N: \emph{I am not watching} TV right now.
  \item Q: \emph{Are you studying} for the test?
\end{itemize}

\subsection*{Ejercicios}
Completa con el \textbf{Presente Continuo}.
\begin{enumerate}
  \item Look! The kids \underline{\hspace{2.5cm}} (play) outside.
  \item I \underline{\hspace{2.5cm}} (not work) this week.
  \item \underline{\hspace{2.5cm}} (she / cook) dinner now?
  \item They \underline{\hspace{2.5cm}} (move) to a new flat this weekend.
  \item The company \underline{\hspace{2.5cm}} (grow) quickly.
  \item Why \underline{\hspace{2.5cm}} (you / laugh)?
  \item He \underline{\hspace{2.5cm}} (not use) his phone at the moment.
  \item We \underline{\hspace{2.5cm}} (plan) a surprise party.
  \item \underline{\hspace{2.5cm}} (it / snow) right now?
  \item I \underline{\hspace{2.5cm}} (read) an amazing book these days.
\end{enumerate}

%========================
\section{Presente Perfecto (Present Perfect)}
\subsection*{Cuándo se usa}
\begin{itemize}
  \item Experiencias de vida (sin tiempo específico): ``I have visited Japan.''
  \item Acciones recientes con resultado presente: ``She has broken her arm.''
  \item Períodos no terminados: ``We have worked a lot this week.''
  \item Cambios o logros: ``He has improved a lot.''
\end{itemize}

\subsection*{Cómo se forma}
\textbf{Afirmativa}: \emph{Sujeto + have/has + participio pasado (V3)}.\\
\textbf{Negativa}: \emph{Sujeto + have/has not (haven’t/hasn’t) + V3}.\\
\textbf{Interrogativa}: \emph{Have/Has + sujeto + V3?}

\subsection*{Ejemplos}
\begin{itemize}
  \item A: \emph{She has finished} her homework.
  \item N: \emph{They haven’t seen} that movie.
  \item Q: \emph{Have you ever tried} sushi?
\end{itemize}

\subsection*{Ejercicios}
Completa con el \textbf{Presente Perfecto}.
\begin{enumerate}
  \item I \underline{\hspace{2.5cm}} (not finish) the report yet.
  \item \underline{\hspace{2.5cm}} (you / ever / be) to Canada?
  \item She \underline{\hspace{2.5cm}} (lose) her keys.
  \item They \underline{\hspace{2.5cm}} (work) here since 2022.
  \item He \underline{\hspace{2.5cm}} (just / call) me.
  \item We \underline{\hspace{2.5cm}} (not decide) the date.
  \item \underline{\hspace{2.5cm}} (the train / leave)?
  \item My English \underline{\hspace{2.5cm}} (improve) a lot.
  \item She \underline{\hspace{2.5cm}} (visit) London three times.
  \item They \underline{\hspace{2.5cm}} (not see) each other lately.
\end{enumerate}

%========================
\section{Pasado Simple (Past Simple)}
\subsection*{Cuándo se usa}
\begin{itemize}
  \item Acciones terminadas en un tiempo específico: ``He left in 2019.''
  \item Hábitos o secuencias en el pasado: ``I walked home and cooked dinner.''
\end{itemize}

\subsection*{Cómo se forma}
\textbf{Afirmativa}: \emph{Sujeto + verbo en pasado (V2)}.\\
\textbf{Negativa}: \emph{Sujeto + did not (didn’t) + verbo base}.\\
\textbf{Interrogativa}: \emph{Did + sujeto + verbo base?}

\subsection*{Ejemplos}
\begin{itemize}
  \item A: \emph{We visited} Rome last year.
  \item N: \emph{She didn’t enjoy} the concert.
  \item Q: \emph{Did you finish} the task?
\end{itemize}

\subsection*{Ejercicios}
Completa con el \textbf{Pasado Simple}.
\begin{enumerate}
  \item They \underline{\hspace{2.5cm}} (arrive) late yesterday.
  \item I \underline{\hspace{2.5cm}} (not understand) the question.
  \item \underline{\hspace{2.5cm}} (you / see) the eclipse?
  \item She \underline{\hspace{2.5cm}} (buy) a new laptop.
  \item We \underline{\hspace{2.5cm}} (walk) along the river.
  \item He \underline{\hspace{2.5cm}} (not call) me last night.
  \item Where \underline{\hspace{2.5cm}} (they / go) on holiday?
  \item The meeting \underline{\hspace{2.5cm}} (start) at 9.
  \item I \underline{\hspace{2.5cm}} (lose) my wallet on the bus.
  \item She \underline{\hspace{2.5cm}} (study) French at school.
\end{enumerate}

%========================
\section{Pasado Continuo (Past Continuous)}
\subsection*{Cuándo se usa}
\begin{itemize}
  \item Acciones en progreso en un momento del pasado: ``At 8 pm, I was studying.''
  \item Acción interrumpida por otra (Past Simple): ``I was cooking when he arrived.''
  \item Acciones paralelas en el pasado: ``She was reading while I was writing.''
\end{itemize}

\subsection*{Cómo se forma}
\textbf{Afirmativa}: \emph{Sujeto + was/were + verbo-ing}.\\
\textbf{Negativa}: \emph{Sujeto + was/were + not + verbo-ing}.\\
\textbf{Interrogativa}: \emph{Was/Were + sujeto + verbo-ing?}

\subsection*{Ejemplos}
\begin{itemize}
  \item A: \emph{They were driving} home at 7.
  \item N: \emph{I wasn’t listening} to the radio.
  \item Q: \emph{Were you waiting} for long?
\end{itemize}

\subsection*{Ejercicios}
Completa con el \textbf{Pasado Continuo}.
\begin{enumerate}
  \item I \underline{\hspace{2.5cm}} (watch) TV when the phone \underline{\hspace{2.5cm}} (ring).
  \item They \underline{\hspace{2.5cm}} (not sleep) at midnight.
  \item \underline{\hspace{2.5cm}} (she / drive) when it \underline{\hspace{2.5cm}} (start) to rain?
  \item We \underline{\hspace{2.5cm}} (have) dinner while our friends \underline{\hspace{2.5cm}} (talk).
  \item He \underline{\hspace{2.5cm}} (jog) in the park at 6 am.
  \item The kids \underline{\hspace{2.5cm}} (play) when the lights \underline{\hspace{2.5cm}} (go) out.
  \item I \underline{\hspace{2.5cm}} (not pay) attention.
  \item \underline{\hspace{2.5cm}} (they / argue) about the price?
  \item She \underline{\hspace{2.5cm}} (read) while I \underline{\hspace{2.5cm}} (cook).
  \item What \underline{\hspace{2.5cm}} (you / do) at 9 last night?
\end{enumerate}

%========================
\section{Pasado Perfecto (Past Perfect)}
\subsection*{Cuándo se usa}
\begin{itemize}
  \item La acción más antigua de dos en el pasado: ``He had left before I arrived.''
  \item Causa o consecuencia en el pasado: ``She was tired because she had worked all day.''
\end{itemize}

\subsection*{Cómo se forma}
\textbf{Afirmativa}: \emph{Sujeto + had + participio pasado (V3)}.\\
\textbf{Negativa}: \emph{Sujeto + had not (hadn’t) + V3}.\\
\textbf{Interrogativa}: \emph{Had + sujeto + V3?}

\subsection*{Ejemplos}
\begin{itemize}
  \item A: \emph{They had finished} the project before the deadline.
  \item N: \emph{I hadn’t seen} him before that day.
  \item Q: \emph{Had she studied} enough for the exam?
\end{itemize}

\subsection*{Ejercicios}
Completa con el \textbf{Pasado Perfecto}.
\begin{enumerate}
  \item By 8 o’clock, we \underline{\hspace{2.5cm}} (finish) dinner.
  \item She was angry because he \underline{\hspace{2.5cm}} (forget) her birthday.
  \item \underline{\hspace{2.5cm}} (you / ever / try) sushi before that night?
  \item They \underline{\hspace{2.5cm}} (leave) when I arrived.
  \item He \underline{\hspace{2.5cm}} (not read) the book before the class.
  \item After we \underline{\hspace{2.5cm}} (pack), we called a taxi.
  \item I realized I \underline{\hspace{2.5cm}} (lose) my keys.
  \item \underline{\hspace{2.5cm}} (she / finish) the report by Monday?
  \item The match \underline{\hspace{2.5cm}} (already / start) when we got there.
  \item They were happy because they \underline{\hspace{2.5cm}} (win).
\end{enumerate}

\section{Futuro con \emph{will} (Futuro simple)}
\subsection*{Cuándo se usa}
\begin{itemize}
  \item Decisiones espontáneas y ofrecimientos en el momento de hablar: \emph{I’ll help you}.
  \item Predicciones/opiniones sin evidencia fuerte: \emph{It’ll be fine}.
  \item Promesas/amenazas/hechos futuros: \emph{I won’t be late}.
\end{itemize}

\subsection*{Cómo se construye}
\textbf{Afirmativa}: \emph{Sujeto + will + verbo base}. (I/You/He/She/It/We/They \emph{will go})\\
\textbf{Negativa}: \emph{Sujeto + will not (won’t) + verbo base}. (She \emph{won’t agree})\\
\textbf{Interrogativa}: \emph{Will + sujeto + verbo base?} (\emph{Will they come}?)

\subsection*{Mini-ejemplos}
\emph{I’ll call you later. / We won’t forget. / Will you join us?}

%========================
\section{\emph{Be going to}}
\subsection*{Cuándo se usa}
\begin{itemize}
  \item Planes/intenciones decididas antes de hablar: \emph{We’re going to travel}.
  \item Predicciones con evidencia presente: \emph{Look at those clouds! It’s going to rain}.
\end{itemize}

\subsection*{Cómo se construye}
\textbf{Afirmativa}: \emph{Sujeto + am/is/are + going to + verbo base}.\\
\textbf{Negativa}: \emph{Sujeto + am/is/are + not + going to + verbo base}.\\
\textbf{Interrogativa}: \emph{Am/Is/Are + sujeto + going to + verbo base?}

\subsection*{Mini-ejemplos}
\emph{I’m going to start a course. / She isn’t going to buy it. / Are you going to apply?}

%========================
\section{Futuro Continuo (\emph{will be + -ing})}
\subsection*{Cuándo se usa}
\begin{itemize}
  \item Acción en progreso en un momento específico del futuro: \emph{At 8, I’ll be studying}.
  \item Situaciones previstas o educadamente preguntar por planes: \emph{Will you be using the car?}
\end{itemize}

\subsection*{Cómo se construye}
\textbf{Afirmativa}: \emph{Sujeto + will be + verbo-ing}.\\
\textbf{Negativa}: \emph{Sujeto + will not (won’t) be + verbo-ing}.\\
\textbf{Interrogativa}: \emph{Will + sujeto + be + verbo-ing?}

\subsection*{Mini-ejemplos}
\emph{They’ll be working at noon. / I won’t be waiting long. / Will she be driving?}

%========================
\section{Futuro Perfecto (\emph{will have + V3})}
\subsection*{Cuándo se usa}
\begin{itemize}
  \item Acción \textbf{completada} antes de un punto del futuro: \emph{By 2030, we will have reduced costs}.
  \item Inferencias sobre el pasado reciente desde el futuro: \emph{He’ll have left by now} (matiz UK).
\end{itemize}

\subsection*{Cómo se construye}
\textbf{Afirmativa}: \emph{Sujeto + will have + participio pasado (V3)}.\\
\textbf{Negativa}: \emph{Sujeto + will not (won’t) have + V3}.\\
\textbf{Interrogativa}: \emph{Will + sujeto + have + V3?}

\subsection*{Mini-ejemplos}
\emph{She’ll have finished by Monday. / We won’t have arrived by 6. / Will you have read it?}

%========================
\section{Nota sobre ``Futuro simple''}
En gramática escolar, \textbf{futuro simple} suele referirse a \emph{will + verbo base}. Aquí lo hemos unificado con la sección de \emph{will}. Para planes/evidencia presente, usa \emph{be going to}; para acción en progreso en el futuro, \emph{futuro continuo}; para acción completada antes de un momento futuro, \emph{futuro perfecto}.

%========================
\section{Práctica global (30 oraciones, tiempos mezclados)}
\textit{Instrucciones:} Completa con la forma verbal correcta \textbf{(sin indicar el tiempo)}. Algunas frases tienen \textbf{más de un hueco} y pueden requerir \textbf{tiempos distintos}.
\begin{enumerate}
  \item Don’t worry, I \underline{\hspace{2.2cm}} (send) the file as soon as I get home.
  \item Look at those clouds! It \underline{\hspace{2.2cm}} (rain) soon.
  \item By the time you arrive, we \underline{\hspace{2.2cm}} (finish) dinner.
  \item What \underline{\hspace{2.2cm}} (you / do) at 9 p.m. tomorrow?
  \item She usually \underline{\hspace{2.2cm}} (take) the bus, but today she \underline{\hspace{2.2cm}} (drive).
  \item I \underline{\hspace{2.2cm}} (not believe) you; \underline{\hspace{2.2cm}} (you / talk) to him already?
  \item When the phone \underline{\hspace{2.2cm}} (ring), I \underline{\hspace{2.2cm}} (cook) pasta.
  \item We \underline{\hspace{2.2cm}} (plan) a trip next month; we \underline{\hspace{2.2cm}} (visit) Lisbon.
  \item He \underline{\hspace{2.2cm}} (work) here since 2023, so he \underline{\hspace{2.2cm}} (know) the process well.
  \item \underline{\hspace{2.2cm}} (the train / leave) at 7 or at 7:30?
  \item If you need help, I \underline{\hspace{2.2cm}} (call) IT right now.
  \item They \underline{\hspace{2.2cm}} (already / start) the meeting when we arrived late.
  \item At noon tomorrow, I \underline{\hspace{2.2cm}} (meet) the new client.
  \item She \underline{\hspace{2.2cm}} (not go) to the party; she \underline{\hspace{2.2cm}} (study) for her exam.
  \item I think prices \underline{\hspace{2.2cm}} (go) up next year.
  \item Listen! The baby \underline{\hspace{2.2cm}} (cry).
  \item We \underline{\hspace{2.2cm}} (not finish) the project yet, but we \underline{\hspace{2.2cm}} (be) close.
  \item \underline{\hspace{2.2cm}} (you / ever / try) Korean food?
  \item While I \underline{\hspace{2.2cm}} (check) the report, the system \underline{\hspace{2.2cm}} (crash).
  \item By next Friday, the team \underline{\hspace{2.2cm}} (complete) the first milestone.
  \item She \underline{\hspace{2.2cm}} (not usually / drink) coffee, but today she \underline{\hspace{2.2cm}} (need) energy.
  \item \underline{\hspace{2.2cm}} (you / join) us later, or \underline{\hspace{2.2cm}} (you / work) all evening?
  \item Look at his suitcase—he \underline{\hspace{2.2cm}} (travel) abroad next week.
  \item When we got to the cinema, the film \underline{\hspace{2.2cm}} (already / start).
  \item This time tomorrow, we \underline{\hspace{2.2cm}} (fly) over the Alps.
  \item I promise I \underline{\hspace{2.2cm}} (not tell) anyone.
  \item The company \underline{\hspace{2.2cm}} (grow) quickly these days.
  \item Where \underline{\hspace{2.2cm}} (you / be) at 6 p.m. yesterday?
  \item As soon as she \underline{\hspace{2.2cm}} (finish) this task, she \underline{\hspace{2.2cm}} (send) the report.
  \item By 2030, many cities \underline{\hspace{2.2cm}} (have) electric buses only.
\end{enumerate}

%========================
\section{Reading: identificar tiempos verbales}
\subsection*{Texto}
\textit{Yesterday was hectic. At 7 a.m., I was leaving the house when I realized I had forgotten my keys. By the time I got back, my neighbor had already left for work. Later, while I was waiting for the bus, I got an email saying our team has finally reached the quarterly target. Tomorrow morning, I’ll be meeting a new client, and by next week we will have closed the deal if everything goes as planned. Look at the sky now—it's going to rain, so I think I’ll take a taxi.}

\subsection*{Tareas}
\begin{enumerate}
  \item Subraya cada forma verbal y \textbf{anota el tiempo} (p.\,ej., Pasado continuo, Pasado perfecto, Presente perfecto, Futuro continuo, Futuro perfecto, Will, Be going to).
  \item Explica brevemente el \textbf{porqué} de dos casos: (a) uso de \emph{had forgotten} y (b) uso de \emph{will have closed}.
  \item Verdadero/Falso + corrige: \\
  a) El email llegó antes de que esperara el bus. \\
  b) La reunión con el cliente ocurrirá ahora mismo. \\
  c) La predicción de lluvia se basa en evidencia presente.
\end{enumerate}

%========================
\section{Writing: historia usando varios tiempos}
\textit{Instrucciones}: Escribe entre 140–180 palabras contando una mini-historia real o inventada que incluya:
\begin{itemize}
  \item \textbf{Pasado simple} (acciones principales) y \textbf{pasado continuo} (fondo).
  \item \textbf{Pasado perfecto} (acción anterior a otra en el pasado).
  \item \textbf{Presente perfecto} (resultado/experiencia reciente relevante).
  \item \textbf{Futuro}: al menos dos formas distintas (\emph{will}, \emph{be going to}, \emph{futuro continuo} o \emph{futuro perfecto}).
\end{itemize}
\textbf{Checklist (márcalo al final):}

\begin{itemize}
  \item $\square$ Usé 6+ tiempos distintos
  \item $\square$ Conectores: \emph{when, while, by the time, after, because, so}
  \item $\square$ Ortografía y puntuación correctas
  \item $\square$ Claridad y coherencia del relato
\end{itemize}
\vspace{0.6em}
\textbf{Modelo (muy breve, orientativo)}:\\
\textit{I was walking to the station when I realized I had left my pass at home. By the time I got back, the train had already departed. I’ve missed a few trains this month, so I’m going to set two alarms. Tomorrow at 8, I’ll be waiting on the platform, and by next Friday I will have arrived on time every day. I think it’ll be fine.}


\end{document}