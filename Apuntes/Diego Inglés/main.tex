\documentclass[a4paper,12pt]{article}
\usepackage[spanish]{babel}
\usepackage[utf8]{inputenc}
\usepackage[T1]{fontenc}
\usepackage{enumitem}
\usepackage{geometry}
\geometry{margin=2.5cm}
\usepackage{multicol}
\usepackage{xcolor}
\usepackage{titlesec}
\usepackage{amssymb}

\titleformat{\section}{\large\bfseries\color{blue}}{}{0em}{}
\titleformat{\subsection}{\bfseries\color{teal}}{}{0em}{}

\begin{document}

\begin{center}
\Large{\textbf{Resumen y Ejercicios – Unidad: Células y Tejidos}} \\
\vspace{0.3cm}
\small{Preparado para repaso con estudiante con TDAH – versión simplificada y dinámica}
\end{center}

---

\section*{🧠 RESUMEN GENERAL}

\subsection*{🔬 El método científico}
\begin{itemize}
    \item Pasos: Observación, Pregunta, Hipótesis, Experimento, Análisis, Conclusión, Comunicación.
    \item Proyecto de investigación: observar, preguntar, elegir variable, experimentar y analizar (ejemplo: degradación de la vitamina C).
\end{itemize}

\subsection*{🧫 La célula}
\begin{itemize}
    \item Es la unidad básica de los seres vivos.
    \item Tipos:
    \begin{itemize}
        \item \textbf{Procariotas}: sin núcleo (bacterias).
        \item \textbf{Eucariotas}: con núcleo (animales y vegetales).
    \end{itemize}
    \item Partes:
    \begin{itemize}
        \item Membrana plasmática: protege y regula.
        \item Citoplasma: contiene orgánulos.
        \item Núcleo: controla la célula y guarda el ADN.
    \end{itemize}
    \item Orgánulos: Mitocondria (energía), Lisosomas (desechos), Ribosomas (proteínas), Retículo endoplasmático (transporte), Aparato de Golgi (empaquetado), Centríolos (división celular).
\end{itemize}

\subsection*{🌱 Células vegetales}
\begin{itemize}
    \item Pared celular, cloroplastos, vacuola grande.
    \item Fotosíntesis: luz + agua + CO$_2$ → glucosa + oxígeno.
\end{itemize}

\subsection*{🧍‍♀️ Células animales}
\begin{itemize}
    \item Sin pared celular ni cloroplastos.
    \item Realizan respiración celular para obtener energía.
\end{itemize}

\subsection*{🧬 Núcleo y ADN}
\begin{itemize}
    \item El núcleo guarda el ADN.
    \item El ADN forma cromosomas.
\end{itemize}

\subsection*{🧍‍♂️ Tejidos y órganos}
\begin{itemize}
    \item Tejido = células parecidas.
    \item Órgano = tejidos con una función.
    \item Sistema = varios órganos.
    \item Organismo = ser vivo completo.
\end{itemize}

Ejemplos:
\begin{itemize}
    \item Tejido epitelial: piel.
    \item Tejido muscular: movimiento.
    \item Tejido nervioso: transmite información.
    \item Tejido óseo: da soporte.
    \item Tejido adiposo: almacena grasa.
\end{itemize}

\subsection*{⚙️ Funciones vitales}
\begin{itemize}
    \item Nutrición → obtener energía.
    \item Relación → reaccionar ante estímulos.
    \item Reproducción → crear nuevos seres.
\end{itemize}

\subsection*{⚗️ El laboratorio}
\begin{itemize}
    \item Materiales: microscopio, pipeta, portaobjetos, lupa, bisturí.
    \item Importancia de normas de seguridad.
\end{itemize}

---

\section*{🎯 EJERCICIOS SIN SOLUCIÓN}

\subsection*{🧩 1. Tipo Test}
\begin{enumerate}[label=\arabic*)]
    \item ¿Qué parte de la célula contiene el ADN?  
    a) Citoplasma \hspace{1cm} b) Núcleo \hspace{1cm} c) Mitocondria
    \item ¿Cuál de estos orgánulos fabrica energía?  
    a) Aparato de Golgi \hspace{1cm} b) Lisosoma \hspace{1cm} c) Mitocondria
    \item ¿Qué tipo de célula tiene pared celular?  
    a) Animal \hspace{1cm} b) Vegetal \hspace{1cm} c) Ninguna
    \item ¿Qué paso NO pertenece al método científico?  
    a) Hipótesis \hspace{1cm} b) Comunicación \hspace{1cm} c) Decoración
\end{enumerate}

\subsection*{🧠 2. Verdadero o Falso}
\begin{enumerate}[label=\arabic*)]
    \item Las células procariotas tienen núcleo.
    \item La fotosíntesis produce oxígeno.
    \item Los tejidos están formados por órganos.
    \item El núcleo controla las funciones de la célula.
\end{enumerate}

\subsection*{🔗 3. Relaciona}
\begin{tabular}{ll}
1. Membrana plasmática → & a) Limpieza de la célula \\
2. Mitocondria → & b) Controla entrada y salida \\
3. Aparato de Golgi → & c) Genera energía \\
4. Lisosomas → & d) Empaqueta sustancias \\
\end{tabular}

\subsection*{✏️ 4. Completa las frases}
\begin{enumerate}[label=\arabic*)]
    \item La unidad básica de los seres vivos es la \_\_\_\_\_\_\_\_\_\_.
    \item Las células vegetales tienen \_\_\_\_\_\_\_\_\_ y \_\_\_\_\_\_\_\_\_.
    \item El ADN se encuentra dentro del \_\_\_\_\_\_\_\_\_\_.
    \item Los tejidos se agrupan en \_\_\_\_\_\_\_\_ y estos en \_\_\_\_\_\_\_\_.
\end{enumerate}

\subsection*{💬 5. Preguntas cortas}
\begin{enumerate}[label=\arabic*)]
    \item ¿Qué diferencia hay entre una célula animal y una vegetal?
    \item Nombra tres orgánulos y su función.
    \item ¿Qué hace el tejido nervioso?
    \item ¿Qué pasos tiene el método científico?
\end{enumerate}

---

\section*{✅ EJERCICIOS CON SOLUCIÓN}

\subsection*{🧩 Tipo Test}
1. b) Núcleo \\
2. c) Mitocondria \\
3. b) Vegetal \\
4. c) Decoración \\

\subsection*{🧠 Verdadero o Falso}
1. ❌ Falso \\
2. ✅ Verdadero \\
3. ❌ Falso \\
4. ✅ Verdadero \\

\subsection*{🔗 Relaciona}
1 → b) Controla entrada y salida \\
2 → c) Genera energía \\
3 → d) Empaqueta sustancias \\
4 → a) Limpieza de la célula \\

\subsection*{✏️ Completa las frases}
1. célula \\
2. pared celular y cloroplastos \\
3. núcleo \\
4. órganos – sistemas \\

\subsection*{💬 Preguntas cortas}
\begin{enumerate}[label=\arabic*)]
    \item La célula vegetal tiene pared celular y cloroplastos; la animal no.
    \item Ejemplo: Mitocondria (energía), Lisosoma (digestión), Núcleo (control).
    \item Transmite información mediante impulsos nerviosos.
    \item Observación, hipótesis, experimento, resultados, conclusión y comunicación.
\end{enumerate}

\end{document}