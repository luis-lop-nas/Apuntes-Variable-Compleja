% --- Idioma y codificación ---
\usepackage[utf8]{inputenc}   % Acentos directos
\usepackage[T1]{fontenc}      % Codificación de salida
\usepackage[spanish]{babel}   % Español
\usepackage{lmodern}          % Fuente moderna

% --- Matemáticas ---
\usepackage{amsmath}          % Entornos matemáticos
\usepackage{amssymb}          % Símbolos
\usepackage{amsthm}           % Teoremas
\usepackage{mathtools}        % Extras de amsmath
\usepackage{bm}               % Negrita en símbolos
\usepackage{esint}            % Integrales de contorno (∮)

% --- Utilidades ---
\usepackage{graphicx}         % Imágenes
\usepackage{xcolor}           % Colores
\usepackage{enumitem}         % Listas
\usepackage{titlesec}         % Títulos
\usepackage[hidelinks]{hyperref} % Enlaces (cargar al final)

% --- Esquemas y Dibujos ---
\usepackage{tikz}
\usetikzlibrary{arrows.meta}

% subsecciones más pequeñas
\titleformat{\subsection}
  {\normalfont\small\bfseries}   % formato más pequeño y en negrita
  {\thesubsection}{1em}{} 

  \title{Teoremario}
\author{Luis López}
\date{September 2025}

\begin{document}

\maketitle

\cleardoublepage
% Introducción sin título
\thispagestyle{plain}

Un \textbf{teorema} es una proposición matemática que puede ser demostrada 
a partir de axiomas, definiciones y otros resultados previamente aceptados. 
En física y matemáticas, los teoremas actúan como herramientas fundamentales: 
permiten deducir propiedades, justificar métodos de resolución y garantizar 
que las soluciones de los problemas se apoyan en bases sólidas y verificadas. 
Más allá de ser simples fórmulas, representan conexiones profundas entre 
conceptos y proporcionan la seguridad de que los procedimientos seguidos
son correctos.\\

Este documento es un recopilatorio de teoremas y resultados fundamentales 
de matemáticas y física que resultan útiles a la hora de resolver ejercicios. 
La idea es disponer de un archivo de referencia al que poder acudir cuando se 
presenta una duda o un bloqueo en un problema, de manera que se pueda consultar 
rápidamente la formulación, el contexto y la demostración de un teorema concreto.\\

Cada entrada de este archivo está organizada de manera uniforme para 
facilitar la lectura y la consulta. La estructura de los teoremas es la siguiente:

\begin{itemize}
  \item \textbf{Nombre del teorema}: identificación breve y clara.  
  \item \textbf{Dónde se usa}: rama de la física o de las matemáticas donde aparece.  
  \item \textbf{Para qué sirve}: el objetivo o la utilidad principal del teorema 
  en la resolución de ejercicios.  
  \item \textbf{Explicación}: descripción sencilla de la idea central detrás del teorema.  
  \item \textbf{Demostración}: desarrollo matemático riguroso que justifica el resultado.  
\end{itemize}

De esta forma, el archivo actúa como una herramienta práctica y ordenada, 
pensada tanto para repasar teoría como para aplicarla directamente a la resolución 
de problemas.

% =====================
% Índice
% =====================
\cleardoublepage
\tableofcontents

% =====================
% Página con título "Teoremas"
% =====================
\cleardoublepage

\subsection*{Teorema de Pitágoras}
\addcontentsline{toc}{subsection}{Teorema de Pitágoras}

\textbf{Dónde se usa:}  
Geometría, trigonometría, matemáticas básicas y física en problemas de movimiento, vectores y distancias.\\

\textbf{Para qué sirve:}  
Permite calcular un lado de un triángulo rectángulo a partir de los otros dos. 
Se aplica para hallar la hipotenusa o cualquiera de los catetos, y aparece de manera recurrente en problemas de física y matemáticas que involucran magnitudes perpendiculares, como fuerzas, desplazamientos o componentes vectoriales.\\

\textbf{Explicación:}  
El teorema establece que, en un triángulo rectángulo, el cuadrado de la longitud de la hipotenusa es igual a la suma de los cuadrados de las longitudes de los catetos.  
Si $a$ y $b$ son los catetos y $c$ la hipotenusa:
\[
c^2 = a^2 + b^2
\]
Este resultado expresa una relación fundamental entre las distancias y constituye la base de muchas fórmulas y métodos de resolución de ejercicios.\\

\textbf{Demostración:}  
Existen múltiples demostraciones del teorema de Pitágoras. Una de las más clásicas se basa en considerar un cuadrado de lado $(a+b)$ y colocar en su interior cuatro copias del triángulo rectángulo. El área del cuadrado puede calcularse de dos formas diferentes:  
- Como $(a+b)^2$.  
- Como suma del área de los cuatro triángulos rectángulos más el área del cuadrado central de lado $c$, es decir: $4\cdot\frac{1}{2}ab + c^2$.  

\begin{figure}[h]
    \centering
    \includegraphics[width=0.6\textwidth]{IMG_0483.jpg}
    \label{fig:pitagoras}
\end{figure}

Igualando ambas expresiones:
\[
(a+b)^2 = 2ab + c^2 + a^2 + b^2
\]

Lo que simplifica a:

\[
c^2 = a^2 + b^2
\]

De esta manera queda demostrada la relación pitagórica.

\cleardoublepage

\subsection*{Teorema fundamental de los límites laterales}
\addcontentsline{toc}{subsection}{Teorema fundamental de los límites laterales}

\textbf{Dónde se usa:}  
Análisis matemático, cálculo diferencial y estudio de funciones. Es esencial en el tratamiento de continuidad, derivadas y en la definición rigurosa de límites.\\

\textbf{Para qué sirve:}  
Permite determinar si una función posee límite en un punto verificando la existencia e igualdad de sus límites laterales. Es una herramienta fundamental para analizar el comportamiento de funciones cerca de un valor específico.\\

\textbf{Explicación:}  
El teorema establece que el límite de una función en un punto existe \emph{si y solo si} los límites laterales tanto por la izquierda como por la derecha en ese punto existe y son iguales.  
Si alguno de los dos límites laterales no existe, o si son diferentes, el límite bilateral no existe.  
En notación matemática:
\[
\exists\lim_{x \to a} f(x)  \iff 
\begin{cases}
\exists \lim\limits_{x \to a^-} f(x), \\[6pt]
\exists \lim\limits_{x \to a^+} f(x), \\[6pt]
\lim\limits_{x \to a^-} f(x) = \lim\limits_{x \to a^+} f(x).
\end{cases}
\]

\textbf{Demostración:}  
El límite bilateral se define como el valor al que se aproxima la función $f(x)$ cuando $x$ se acerca a $a$ sin especificar la dirección.  
Para que esa aproximación sea consistente, debe cumplirse que el valor desde la izquierda y desde la derecha coinciden.  
De manera formal:  
- Si $\lim\limits_{x \to a^-} f(x) = L$ y $\lim\limits_{x \to a^+} f(x) = L$, entonces $\lim\limits_{x \to a} f(x) = L$.  
- Si $\lim\limits_{x \to a^-} f(x) \neq \lim\limits_{x \to a^+} f(x)$, entonces $\nexists \lim\limits_{x \to a} f(x)$.\\

\textbf{Ejemplo:}  
\[
f(x) =
\begin{cases}
1 & x < 0, \\
2 & x \geq 0,
\end{cases}
\]
En $x=0$:  
\[
\lim_{x \to 0^-} f(x) = 1, \quad \lim_{x \to 0^+} f(x) = 2
\]
Como los límites laterales son distintos, no existe \(\lim_{x \to 0} f(x)\).

\end{document}